\documentclass[8pt, a4paper, twocolumn, twoside]{extarticle}

\usepackage{extsizes} % Smaller text
\usepackage{fullpage}
\usepackage[medium]{titlesec} % Smaller section's font
\usepackage{parskip} % To avoid indentation
\usepackage{textcomp} % To have \textcelsius and other symbols
\usepackage[labelformat=empty]{caption}
\usepackage{etoolbox}
\usepackage[showframe=false, top=1cm, bottom=1.5cm, left=1.5cm, right=1.5cm]{geometry}

\usepackage[pdftex,
pdfauthor={Davide Cossu, Stefano D'Agaro},
pdftitle={Formulario di Matematica},
pdfsubject={Matematica},
pdfkeywords={matematica, formulario, formule},
pdfproducer={LaTeX with hyperref},
pdfcreator={pdflatex with hyperref}]{hyperref}
\hypersetup{
  colorlinks,
  citecolor=black,
  filecolor=black,
  linkcolor=blue,
  urlcolor=black
}
\usepackage[all]{hypcap} % To fix caption loading of hyperref

% For better visual in tables
\renewcommand*{\arraystretch}{2.5}

% Footers for date, vesion and copyright
\usepackage[yyyymmdd]{datetime}
\renewcommand{\dateseparator}{-}

\usepackage{fancyhdr}
\pagestyle{fancy}
\fancyhead{} % clear all header fields
\renewcommand{\headrulewidth}{0pt} % no line in header area
\fancyfoot{} % clear all footer fields
\fancyfoot[LE,RO]{
  \thepage
}
\fancyfoot[RE,LO]{
  ``Se l'uomo non sapesse di matematica non si eleverebbe di un sol palmo da terra'' - 
  Galielo Galilei
}
\fancypagestyle{firststyle}{
  \fancyhf{}
  %\fancyfoot[C]{\thepage}
  \fancyfoot[L]{
    Version 2.7182818 \today
  }
  \fancyfoot[R]{
    Copyright \copyright 2017--\the\year$\,$Cossu Davide
  }
}

\input{plot-snippets.tex}
\input{util-snippets.tex}

\begin{document}
\thispagestyle{firststyle}
\newcommand{\LabelText}{}
\NewDocumentCommand{\LabelPoint}{o o m m g}{
  \def\plotcmd{\addplot [#1] coordinates{(#3,#4)} node [#2]}
  \IfNoValueTF{#5}{
    \renewcommand{\LabelText}{ {$(#3,#4)$}} % Label with given coordinates
    }{
    \renewcommand{\LabelText}{ {#5}} % Use given label
  }

  \expandafter\plotcmd\LabelText;

  %\pgfsyssoftpath@flushcurrentpath % -- Syntax error
  %\pgfusepath{fill}                % -- no effect
}

\NewDocumentCommand{\LabelPointX}{o o m m g}{
  \IfNoValueTF{#5}{
    \addplot [#1] coordinates{(#3,#4)} node [#2] {$(#3,#4)$};
    }{
    \addplot [#1] coordinates{(#3,#4)} node [#2] {#5};
  }
}
\onecolumn
\begin{@twocolumnfalse}
  \clearpage
  \vspace*{\stretch{1.5}}
  \begin{center}
    \begin{figure*}[!h]
      \centering
      \captionsetup{justification=centering}
      \includegraphics[scale=1.7]{colophon/octopus}
      \caption{
        \textit{`Mi piace la libertà della matematica. Se studi fisica 
          o chimica devi descrivere il mondo reale. Ma in matematica puoi 
          costruire le tue strutture. Puoi camminare in mondi creati 
          dall'immaginazione delle persone. Non sei legato al 
          mondo reale. È come essere Dio in un certo senso. Puoi creare mondi, 
          e studiarli. Credo sia per una combinazione della bellezza, 
          dell'immaginazione e della libertà. 
        '}\\- Aner Shalev\\ [\baselineskip]
        \textit{`La matematica, ahinoi, si presta ai colpi bassi. C'è un 
          «terrorismo matematico», che consiste nello spaventare l'avversario 
          sparandogli contro raffiche di equazioni, derivate, integrali, logaritmi,
        matrici, teoremi e corollari.'}\\- Sergio Ricossa\\ [\baselineskip]
      \textit{`Più scrivi, più sbagli.'}\\- Paolo Mainardis}
      \label{fig:octopus}
    \end{figure*}
    \vspace{1cm}
    \begin{minipage}{.6\textwidth}				
      \begin{center}
        {\Huge\textbf{Formulario di Matematica}}\\\vspace{0.2cm}
        {\textbf{Davide Cossu}}
      \end{center}
      \vspace{0.5cm}
      Questo è un formulario con le formule di matematica fatte durante tutti e cinque gli anni 
      di un liceo scientifico con alcune spiegazioni teoriche ed esercizi.
    \end{minipage}
  \end{center}
  \vspace{\stretch{3}} % \vfill equivalent to \vspace{\fill}
  \clearpage
\end{@twocolumnfalse}

\twocolumn
{
  \hypersetup{linkcolor=black}
  \tableofcontents
}

\newpage
\textbf{Durante tutto il formulario, si userà il sistema internazionale di notazione, ovvero $.$ per 
separare interi da decimali e $,$ per separare le migliaia se necessario.}
\section{Simboli}
Qui verrano chiariti i simboli che verranno utilizzati nel formulario. Molti di essi si troveranno
principalmente nelle definizioni formali ma ritorneranno utili anche negli esercizi.

\begin{tabular}{c | M{6cm}}
  $\sum\limits_{i = l}^{n} f(i)$ & $f(l) + f(l+1) + \dotsb + f(n)$\\
  $\prod\limits_{i = l}^{n} f(i)$ & $f(l) \cdot f(l+1) \dotsm f(n)$\\
  $\forall$ & Per ogni\\
  $\exists$ & Esiste\\
  $\in$ & Appartiene\\
  $\not\in$ & Non appartiene\\
  $\subset$ & È contenuto\\
  $\not\subset$ & Non è contenuto\\
  $\cup$ & Unito\\
  $\cap$ & Intersecato\\
  $\mid$, $:$ & Tale che\\
  $\Rightarrow$ & Si ha che\\
  $\mapsto$ & Diventa\\
  $\rightarrow$ & Ne segue che, tende\\
  $\iff$ & Se e solo se
\end{tabular}

%!TEX ROOT=formularioMatematica.tex
\theoremstyle{plain}
\newtheorem*{molteplic}{Molteplicità di un polinomio}
\newtheorem*{tfa}{Teorema fondamentale dell'Algebra}
\newtheorem*{tfa-ext}{Toerema fondamentale dell'Algebra esteso}
\newtheorem*{definizioneGeneraleLimite}{Limite generale}
\newtheorem*{definizioneLimiteFinito}{Limite finito}
\newtheorem*{definizioneLimiteInfinito1}{Limite infinito $+\infty$}
\newtheorem*{definizioneLimiteInfinito2}{Limite infinito $-\infty$}
\newtheorem*{definizioneLimiteInfinitoFinito1}{Limite finito per $x\to+\infty$}
\newtheorem*{definizioneLimiteInfinitoFinito2}{Limite finito per $x\to-\infty$}
\newtheorem*{definizioneLimiteInfinitoInfinito1}{Limite a $+\infty$ per $x\to\pm\infty$}
\newtheorem*{definizioneLimiteInfinitoInfinito2}{Limite a $-\infty$ per $x\to\pm\infty$}
\newtheorem*{limiteFinitoFunzRaz}{Limite finito di una funzione razionale}
\newtheorem*{limiteInfinitoFunzRaz}{Limite all'infinito di una funzione razionale}
\newtheorem*{definizioneLimiteFinitoDestro}{Limite finito destro}
\newtheorem*{definizioneLimiteFinitoSinistro}{Limite finito sinistro}
\newtheorem*{uniLim}{Unicità del limite}
\newtheorem*{confrontoLim}{Teorema del confronto}
\newtheorem*{permanenzaSegno}{Teorema della permanenza del segno}
\newtheorem*{sommaLimiti}{Limite di una somma}
\newtheorem*{prodottoLimiti}{Limite di un prodotto}
\newtheorem*{prodottoLimiti1}{Limite di un prodotto (esteso)}
\newtheorem*{prodottoLimiti2}{Limite di un prodotto (esteso ulteriormente)}
\newtheorem*{quozienteLimiti}{Limite di un quoziente}
\newtheorem*{potenzaLimiti}{Limite di una potenza ($a^{f(x)}$)}
\newtheorem*{potenzaLimiti1}{Limite di una potenza ($[f(x)]^a$)}
\newtheorem*{potenzaLimiti2}{Limite di una potenza ($[f(x)]^{g(x)}$)}
\newtheorem*{moduloLimiti}{Limite di un modulo}
\newtheorem*{logLimiti}{Limite di un logaritmo}
\newtheorem*{weistrass}{Teorema di Weistrass}
\newtheorem*{valoriIntermedi}{Teorema dei valori intermedi}
\newtheorem*{zeri}{Teoremi degli zeri}
\newtheorem*{successioniMonotone}{Teorema delle Successioni Monotone}
\newtheorem*{primitiva}{Primitiva di una funzione}
\newtheorem*{rolle}{Teorema di Rolle}
\newtheorem*{lagrangeDef}{Teorema di Lagrange}
\newtheorem*{lagrangeLemma1}{Monotonia $\leftrightarrow$ Crescenza/Decrescenza}
\newtheorem*{lagrangeLemma2}{Costanza}
\newtheorem*{cauchy}{Teorema di Cauchy}
\newtheorem*{hopital}{Teorema de l'Hôpital}
\newtheorem*{derivataSeconda1}{Massimi e minimi e flessi con derivata seconda}
\newtheorem*{derivataSeconda2}{Concavità con derivata seconda}
\newtheorem*{media}{Teorema del valor medio}
\newtheorem*{tfci}{Teorema fondamentale del calcolo integrale}
\newtheorem*{deriv}{Teorema del criterio di derivabilità}


%!TEX ROOT=formularioMatematica.tex

\section{Generale}\label{sec:gen}
In questa sezione verranno trattati alcuni temi utili in tutto il formulario, come i prodotti notevoli
o alcune prprietà dei radicali.\\
Per gli esercizi si vada a pagina~\pageref{ex:generale}.

\subsection{Prodotti notevoli}\label{subsec:gen:prodnot}
I prodotti notevoli sono dei prodotti o delle fattorizzazioni sempre vere per qualunque numero. 
Risultano essere molto utili quando si deve semplificare un'espressione.
\begin{align*}
  a^2 - b^2 &= (a-b)(a+b)\\
  a^3 - b^3 &= (a-b)(a^2+ab+b^2)
\end{align*}
Questi hanno la seguente formula generale
\begin{equation*}
  a^n - b^n = (a-b)(a^{n-1} - a^{n-2}b + a^{n-3}b^2 - \dotsb - ab^{n-2} + b^{n-1})
\end{equation*}
\begin{align*}
  (a\pm b)^2 &= a^2 \pm 2ab + b^2 \\
  (a\pm b)^3 &= a^3 \pm 3a^2b + 3ab^2 \pm b^3
\end{align*}
Questi hanno la seguente formula generale, anche conosciuto come `Binomio di Newton', approfondito
nella sezione dedicata al calcolo combinatorio.
\begin{equation*}
  (a + b)^n = \sum\limits_{k = 0}^{n}\binom{n}{k}a^{n-k}b^k
\end{equation*}

\subsection{Radicali}
I radicali sono delle espressioni spesso irrazionali che contengono almeno una radice. La radice è 
anche pensabile come una potenza
\begin{equation*}
  \sqrt[n]{x^m} = x^{\frac{m}{n}}.
\end{equation*}

\subsubsection{Addizione e Sottrazione}
Due radicali si possono addizionare o sottrarre se e solo se \textbf{sono simili}
\begin{equation*}
  m\cdot\sqrt[n]{a} \pm p\cdot\sqrt[n]{a} = (m\pm p)\cdot\sqrt[n]{a}
\end{equation*}

\subsubsection{Divisione e Moltiplicazione}
Due radicali si possono moltiplicare o dividere se e solo se \textbf{hanno lo stesso indice}
\begin{equation*}
  \sqrt[m]{a}\cdot\sqrt[m]{b} = \sqrt[m]{ab}\qquad\frac{\sqrt[m]{a}}{\sqrt[m]{b}} = \sqrt[m]{\frac{a}{b}}
  ,\quad b \neq 0
\end{equation*}

\subsubsection{Razionalizzare un radicale}
Per razionalizzare un radicale si intende eliminare il radicale dal denominatore di una frazione in
quanto non è possibile dividere un numero per un irrazionale.
\begin{align*}
  \frac{d}{c\cdot\sqrt{b}} & = \frac{d}{c\cdot\sqrt{b}}\cdot\frac{\sqrt{b}}{\sqrt{b}} = 
  \frac{d\cdot\sqrt{b}}{cb} \quad b > 0,\, c \neq 0 \\
  \frac{c}{a\pm\sqrt{b}} &= \frac{c}{a\pm\sqrt{b}}\cdot\frac{a\mp\sqrt{b}}{a\mp\sqrt{b}} = 
  \frac{c\cdot(a\pm\sqrt{b})}{a^2-b} \quad a\pm\sqrt{b} \neq 0 \\
  \frac{c}{\sqrt{a}\pm\sqrt{b}} &=
  \frac{c}{\sqrt{a}\pm\sqrt{b}}\cdot\frac{\sqrt{a}\mp\sqrt{b}}{\sqrt{a}\mp\sqrt{b}} =
  \frac{c(\sqrt{a}\mp\sqrt{b})}{a-b} \quad \sqrt{a}\pm\sqrt{b} \neq 0
\end{align*}
In generale per razionalizzare si deve moltiplicare per un fattore che annulli il radicale stesso, 
generalmente quel fattore è il radicale o il suo reciproco.

\subsubsection{Radicale di un radicale}
Per risolvere o semplificare espressioni come $\sqrt{a\pm\sqrt{b}}$ si può usare questa formula
\begin{equation*}
  \sqrt{a\pm\sqrt{b}} = \sqrt{\frac{a+\sqrt{a^2-b}}{2}}\pm\sqrt{\frac{a-\sqrt{a^2-b}}{2}}
\end{equation*}

\subsubsection{Disequazioni con radicali}
\begin{align*}
  \sqrt{f(x)} > g(x) &\Leftrightarrow \left\{\begin{cases}
      f(x) \geq 0\\
      g(x) < 0
    \end{cases} \cup \begin{cases}
      g(x) \geq 0\\
      f(x) > g^2(x)
  \end{cases}\right\}\\
  \sqrt{f(x)} < g(x) &\Leftrightarrow \begin{cases}
    f(x) \geq 0\\
    g(x) > 0\\
    f(x) < g^2(x)
  \end{cases}\\
  \sqrt{f(x)} \lessgtr \sqrt{g(x)} &\Leftrightarrow \begin{cases}
    f(x) \geq 0\\
    g(x) \geq 0\\
    f(x) \lessgtr g(x)
  \end{cases}
\end{align*}

\subsection{Equazioni particolari}
Ci sono infiniti tipi di equazioni, alcune però hanno delle soluzioni immediate, eccone alcune.

\subsubsection{Equazioni binomie}
Le equazioni binomie sono nella forma $x^n = a$. Le loro soluzioni sono le seguenti
\begin{align*}
  x &= \pm\sqrt[n]{a} &\quad &\text{Se $n$ è pari e $a\geq0$}\\
  x &= \sqrt[n]{a} &\quad &\text{Se $n$ è dispari}
\end{align*}

\subsubsection{Equazioni trinomie e biquadratiche}
Le equazioni trinomie sono nella forma $ax^{2n} + bx^n + c =0$. Se $n=2$ sono definite biquadratiche.\\
Per risolverle si ponga $t = x^n$ e si risolva l'equazione di secondo grado che ne deriva
\begin{equation*}
  at^2 + bt + c = 0
\end{equation*}
e poi si risolva $x^n = y_1$ e $x^n = y_2$.

\subsubsection{Equazioni di secondo grado}
Le equazioni di secondo grado sono tra le più diffuse. Presentano alcune caratteristiche.\\
Sia $ax^2 + bx + c = 0$ la nostra equazione, allora $x_1$ e $x_2$ sono le sue soluzioni. Per trovarle
si usi la seguente formula
\begin{equation*}
  x_{1/2} = \frac{-b\pm\sqrt{b^2-4ac}}{2a}
\end{equation*}
Conoscendo le soluzioni si può semplificare l'equazione in questo modo
\begin{equation*}
  ax^2+bx+c=a(x-x_1)(x-x_2)
\end{equation*}
Vige anche questa particolarità
\begin{equation*}
  x_1+x_2 = -\frac{b}{a} \qquad x_1\cdot x_2 = \frac{c}{a}
\end{equation*}

\subsection{Ruffini}\label{ruffini}
Il metodo di Ruffini permette di ridurre di grado qualsiasi equazione. Prima di usare questo metodo si
dovrebbe però verificare che non sia possibile usare \hyperref[subsec:gen:prodnot]{prodotti notevoli}
in quanto il processo richiede tempo.\\
Per prima cosa si deve trovare uno \textbf{zero} dell'equazione, ovvero una soluzione. Essi sono
da ricercarsi tra le seguenti frazioni
\begin{equation*}
  \text{Zeri} = \frac{\text{Divisori termine noto}}{\text{Divisori $a$}}
\end{equation*}
Per dimostrare l'utilizzo di questa regola, prendiamo come esempio la seguente equazione
\begin{equation}\label{eq:ruffini}
  2x^3 + 3x + 5 = 0
\end{equation}
Lo zero di quest'equazione è $\mathcolor{red}{-1}$, infatti $2\cdot(-1)^3 + 3\cdot(-1) + 5 = 0$. Il 
seguente disegno chiarisce i passaggi da seguire per ridurre di grado un'equazione.
\begin{center}
  \begin{tikzpicture}
    \draw (-0.5,0) -- (3.5,0);
    \draw (0,2) -- (0,-0.5);
    \draw (3,2) -- (3,-0.5);
    \node[red] (z) at (-0.3,0.2) {$-1$};
    \node (a) at (0.5,1.8) {$2$};
    \node (b) at (1.5,1.8) {$0$};
    \node (c) at (2.5,1.8) {$3$};
    \node (r) at (3.3,1.8) {$+5$};
    \node (b1) at (1.5,0.2) {$-2$};
    \node (c1) at (2.5,0.2) {$2$};
    \node (r1) at (3.3,0.2) {$-5$};
    \node (a2) at (0.5,-0.2) {$2$};
    \node (b2) at (1.5,-0.2) {$-2$};
    \node (c2) at (2.5,-0.2) {$5$};
    \node (r2) at (3.3,-0.2) {$0$};
    \draw[-stealth, thick, cyan] (a) -- (a2);
    \draw[-stealth, thick, red] (a2) -- (z);
    \draw[-stealth, thick] (z) -- (b1);
    \draw[-stealth, thick] (b) -- (b1)
      node[pos=0.5, right]{$+$};
    \node (coef) at (1.5,-0.7) {Coefficienti/Quoziente};
    \node (rest) at (3.5,-0.7) {Resto};
  \end{tikzpicture}
\end{center}

Il processo da seguire è il seguente:
\begin{enumerate}
  \item Moltiplicare il coefficiente del grado massimo per lo zero
  \item Aggiungere al grado successivo il risultato
  \item Continuare fino a che non si arriva al termine noto
\end{enumerate}
Così si otterranno i nuovi coefficienti dell'equazione. Nel nostro caso otteniamo
\begin{equation}\label{eq:ruffini1}
  2x^2 -2x + 5
\end{equation}
però questo non basta in quanto le equazioni~\eqref{eq:ruffini} e~\eqref{eq:ruffini1} non sono 
equivalenti. Per renderle equivalenti, si moltiplichi per $(x-x_0)$ dove $x_0$ è lo zero dell'equazione
originale. Quindi ora abbiamo ottenuto che
\begin{equation*}
  2x^3 + 3x + 5 = (2x^2 -2x + 5)(x+1)
\end{equation*}
E che quindi possiamo dire che $P_n(x) = P_{n-1}(x)\cdot(x-x_0)$ dove $P_n(x)$ è un polinomio di grado
$n$ nella variabile $x$.

% Reset the counter
\setcounter{equation}{0}

\subsection{Regola di Cramer}
La regola di Cramer permette di risolvere sistemi lineari a $n$-incognite. Generalmente non è molto
comodo per grandi valori di $n$ in quanto diventa lungo da risolvere però per due o tre equazioni è
molto comodo.\\ [\baselineskip]
Ogni sistema può essere definito come
\begin{equation*}
  Ax = c
\end{equation*}
dove $A$ è una matrice e $x$ e $c$ sono vettori. Le soluzioni $(x_1,\ldots,x_n)$ sono determinabili
\begin{equation*}
  x_i = \frac{\det A_i}{\det A}
\end{equation*}
dove $A_i$ è la matrice costruita con la $i$-esima colonna di $A$ con il vettore $c$.\\
Esempio a 2 equazioni
\begin{equation*}
  \begin{cases}
    ax+by=\mathcolor{red}{e}\\cx+dy=\mathcolor{red}{f}
  \end{cases} \Leftrightarrow
  \begin{bmatrix}[1]
    a&b\\c&d
  \end{bmatrix}
  \begin{bmatrix}[1]
    x\\y
  \end{bmatrix}=
  \begin{bmatrix}[1]
    \mathcolor{red}{e}\\\mathcolor{red}{f}
  \end{bmatrix}
\end{equation*}
quindi
\begin{align*}
  x = \frac{\begin{vmatrix}[1]
      \mathcolor{red}{e}&b\\\mathcolor{red}{f}&d
    \end{vmatrix}}{\begin{vmatrix}[1]
      a&b\\c&d
  \end{vmatrix}}= \frac{\mathcolor{red}{e}d-b\mathcolor{red}{f}}{ad-bc}
\end{align*}
\begin{align*}
  y = \frac{\begin{vmatrix}[1]
      a&\mathcolor{red}{e}\\c&\mathcolor{red}{f}
    \end{vmatrix}}{\begin{vmatrix}[1]
      a&b\\c&d
  \end{vmatrix}}= \frac{a\mathcolor{red}{f}-\mathcolor{red}{e}c}{ad-bc}
\end{align*}

\subsection{Valori assoluti}
Verranno qui elencate alcune caratteristiche dei valori assoluti.

\subsubsection{Definizione}
\begin{equation*}
  \left\lvert x\right\rvert \Leftrightarrow
  \begin{cases}
    x &\text{se } x \geq 0\\
    -x &\text{se } x < 0
  \end{cases}
\end{equation*}

\subsubsection{Proprietà}
Dalla definizione ne derivano alcune proprietà
\begin{alignat*}{2}
  \left\lvert x\right\rvert &= \left\lvert -x\right\rvert &\qquad 
  \left\lvert x^2\right\rvert&=\left\lvert x\right\rvert^2 = x^2\\
  \left\lvert a+b\right\rvert &\leq \left\lvert a\right\rvert+\left\lvert b\right\rvert & 
  \left\lvert a\cdot b\right\rvert&=\left\lvert a\right\rvert\cdot\left\lvert b\right\rvert
\end{alignat*}

\subsubsection{Funzioni e valori assoluti}
Vengono ora riportati i sistemi risolutivi di funzioni con valori assoluti
\begin{align*}
  \left\lvert f(x)\right\rvert \geq g(x) &\Leftrightarrow 
  \begin{cases}
    f(x) \leq -g(x)\\
    f(x) \geq g(x)
  \end{cases}\\
  \left\lvert f(x)\right\rvert \leq g(x) &\Leftrightarrow -g(x)\leq f(x) \leq g(x)
\end{align*}

\subsection{Geometria}
Qui vengono elencate alcune formule particolari che riguardano la geometria euclidea.

\subsubsection{Teoremi di Euclide}
\begin{center}
  \begin{tikzpicture}[scale=1.5]
    \coordinate (A) at (0.5,1);
    \coordinate (B) at (3,0);
    \coordinate (C) at (0,0);
    \coordinate (O) at (0.71,0.44);


    \draw (A) -- (B) 
      node[pos=0.5, above right]{$c$} -- (C) 
      node[pos=0.5, below]{$a$}
      node[pos=0.25, below, cyan]{$m$}
      node[pos=0.9, below, teal]{$n$} -- cycle 
      node[pos=0.3, above left]{$b$};
    \draw (O) circle (0.43);
    \draw[blue, dashed] (A) -- ++(0,-1)
      node[pos=0.5, right]{$h$};
    \draw[red, dashed] (O) -- ++(0,-0.44)
      node[pos=0.5, right]{$r$};
  \end{tikzpicture}
\end{center}
\begin{alignat*}{2}
  &\textbf{Primo teorema} &\qquad &
  \begin{cases*}
    \dfrac{\mathcolor{cyan}{m}}{c} = \dfrac{c}{a}\\
    \dfrac{\mathcolor{teal}{n}}{b} = \dfrac{b}{a}
  \end{cases*}\\
  &\textbf{Secondo teorema} &\qquad & 
  \frac{\mathcolor{cyan}{m}}{\mathcolor{blue}{h}} = \frac{\mathcolor{blue}{h}}{\mathcolor{teal}{n}}\\
  &\textbf{Proprietà} &\qquad &
  \begin{cases*}
    \mathcolor{blue}{h} = \dfrac{bc}{a}\\
    \mathcolor{red}{r} = \dfrac{b+c-a}{2}
  \end{cases*}
\end{alignat*}

\subsubsection{Formula di Erone}
La formula di Erone permette di trovare l'area di un triangolo qualsiasi conoscendo il semi-perimetro.
\begin{center}
  \begin{tikzpicture}[scale=1.5]
    \coordinate (A) at (0.5,1);
    \coordinate (B) at (3,0);
    \coordinate (C) at (0,0);
    \coordinate (O) at (0.71,0.44);

    \draw (A) -- (B) 
      node[pos=0.5, above right]{$c$} -- (C) 
      node[pos=0.5, below]{$a$} -- cycle 
      node[pos=0.3, above left]{$b$};
  \end{tikzpicture}
\end{center}
\begin{equation*}
  S = \sqrt{p(p-a)(p-b)(p-c)}
\end{equation*}

\subsubsection{Raggio di una circonferenza inscritta di un triangolo}
\begin{equation*}
  r = \frac{\mathscr{A}}{p}
\end{equation*}

\subsubsection{Raggio di una circonferenza circosccritta di un triangolo}
\begin{equation*}
  R = \frac{a}{2\sin\alpha} = \frac{abc}{4\mathscr{A}}
\end{equation*}

%!TEX ROOT=formularioMatematica.tex

\section{Geometria analitica}\label{sec:geomanal}
La geometria analitica è la geometria che si occupa di lavorare nel piano cartesiano ($xOy$).\\
Per gli esercizi si vada a pagina~\pageref{ex:geomanal}.

\subsection{Generale}
Le formule qui riportate sono generali a tutto l'ambito della geometria analitica e non si riferiscono
ad una figura particolare.\\
Di seguito viene rappresentato il tipico piano cartesiano con i suoi quattro quadranti.
\begin{center}
  \begin{tikzpicture}
    \begin{axis}[xmin=-3,ymin=-3,xmax=3,ymax=3]
      \coordinate (O) at (0,0);
      \coordinate (P) at (2,1);
      \pgfmathsetmacro{\P}{(2,1)}
      \node[fill=white,circle,inner sep=0pt] (O-label) at ($(O)+(-135:10pt)$) {$O$};
      \LabelPoint[mark=*,color=red][color=blue,below]{2}{1}{$P(x_P,y_P)$}
    \end{axis}
  \end{tikzpicture}
\end{center}
D'ora in poi, si darà per scontata la convenzione di nominare le coordinate di un punto in base al nome
del punto stesso. Ad esempio $P(x_P, y_P)$. Si noti anche che è possibile definire un punto attraverso
un vettore bidimensionale. Ovvero
\begin{equation*}
  P(x_P,y_P) = \begin{bmatrix}
    x_P\\y_P
  \end{bmatrix}
\end{equation*}

\subsubsection{Distanza tra due punti}
\begin{equation*}
  \overline{AB} = \sqrt{(x_B-x_A)^2+(y_B-y_A)^2}
\end{equation*}

\subsubsection{Punto medio}
\begin{equation*}
  M\left(\frac{x_A-x_B}{2}, \frac{y_A-y_B}{2}\right)
\end{equation*}

\subsubsection{Punto su un segmento in un rapporto $\frac{m}{n}$}
Siano $m$ e $n$ due rapporti a cui sta un punto rispetto al segmento. Ovvero il punto $P(x_P, y_P)$
divide il segmento in due parti una lunga $\frac{n}{m+n}$ e l'altra $\frac{m}{m+n}$.
\begin{equation*}
  P\left(\frac{nx_A+mx_B}{m+n},\frac{ny_A+my_B}{m+n}\right)
\end{equation*}

\subsubsection{Baricentro di un triangolo}
\begin{equation*}
  G\left(\frac{x_A+x_B+x_C}{3},\frac{y_A+y_B+y_C}{3}\right)
\end{equation*}

\subsubsection{Area di un poligono qualsiasi}
\begin{equation*}
  \mathscr{A}(P) = \frac{1}{2}\left\lvert 
    \begin{matrix}[1]
      x_1 & y_1 & 1\\
      x_2 & y_2 & 1\\
      x_3 & y_3 & 1\\
      \vdots & \vdots & 1
  \end{matrix}\right\rvert
\end{equation*}
usando la regola di Sarrus. Questa formula è anche chiamata la formula di Gauss per le aree di 
poligoni.\\
Il modo di calcolare il determinante della matrice è il seguente (andare
in diagonale dall'alto per ogni elemento della prima colonna moltiplicando gli elementi e quando si
cambia colonna si sommi. Andare poi dal basso sottraendo).\\
Il determinante della matrice
\begin{equation*}
  \begin{bmatrix}[1]
    a_{11} & a_{12} & a_{13}\\
    a_{21} & a_{22} & a_{23}\\
    a_{31} & a_{32} & a_{33}
  \end{bmatrix}
\end{equation*}
è dato dalla risoluzione come segue
\begin{center}
  \begin{tikzpicture}[baseline=(A.center), scale=0.5]
    \tikzset{node style ge/.style={circle}}
    \tikzset{bar/.style = {opacity=.3,line width=4 mm,line cap=round,color=#1}}
    \tikzset{plus/.style = {above left,,opacity=1,circle,fill=#1!50}}
    \tikzset{minus/.style = {below left,,opacity=1,circle,fill=#1!50}}

    \matrix (A) [matrix of math nodes, nodes = {node style ge},,column sep=0 mm] 
    {a_{11} & a_{12} & a_{13}\\
      a_{21} & a_{22} & a_{23}\\
      a_{31} & a_{32} & a_{33}\\
      a_{11} & a_{12} & a_{13}\\
      a_{21} & a_{22} & a_{13}\\
    };

    \draw [bar=blue] (A-1-1.north west) node[plus=blue] {$+$} to (A-3-3.south east);
    \draw [bar=blue] (A-2-1.north west) node[plus=blue] {$+$} to (A-4-3.south east);
    \draw [bar=blue] (A-3-1.north west) node[plus=blue] {$+$} to (A-5-3.south east);
    \draw [bar=red]  (A-3-1.south west) node[minus=red] {$-$} to (A-1-3.north east);
    \draw [bar=red]  (A-4-1.south west) node[minus=red] {$-$} to (A-2-3.north east);
    \draw [bar=red]  (A-5-1.south west) node[minus=red] {$-$} to (A-3-3.north east);
  \end{tikzpicture}
\end{center}

\subsection{Rette}\label{subsec:geomanal:retta}
\begin{center}
  \begin{tikzpicture}
    \begin{axis}[xmin=-2,ymin=0,xmax=3,ymax=3]
      \coordinate (O) at (0,0);
      \node[fill=white,circle,inner sep=0pt] (O-label) at ($(O)+(-135:10pt)$) {$O$};
      \addplot[blue,thick] {1/pi*x+1};
    \end{axis}
  \end{tikzpicture}
\end{center}
Le rette sono definite da un'equazione che ha due forme equivalenti:
\begin{equation}\label{eq:retta1}
  y = mx + q\\
\end{equation}
\begin{equation}\label{eq:retta2}
  ax + by + c = 0
\end{equation}
La forma~\eqref{eq:retta1} è chiamata \emph{esplicita}, la forma~\eqref{eq:retta2} è chiamata 
\emph{implicita}. Da queste due forme possiamo evincere che
\begin{equation*}
  m = -\frac{a}{b} \qquad q = -\frac{c}{b}
\end{equation*}
% Reset the counter
\setcounter{equation}{0}

\subsubsection{Retta passante per due punti $P_1(x_1,y_1)$ e $P_2(x_2,y_2)$}
\begin{equation*}
  \frac{y-y_1}{y_2-y_1}=\frac{x-x_1}{x_2-x_1} \qquad x_1\neq x_2 \land y_1\neq y_2
\end{equation*}

\subsubsection{Condizione di parallelismo}
Perché due rette siano parallele, \textbf{il loro coefficiente angolare deve essere uguale}, ovvero
\begin{equation*}
  r_1 \| r_2 \iff m_1=m_2
\end{equation*}

\subsubsection{Condizione di perpendicolarità}
Perché due rette siano perpendicolari, \textbf{il prodotto dei coefficienti angolari deve essere $-1$},
ovvero
\begin{equation*}
  r_1 \perp r_2 \iff m_1m_2 = -1
\end{equation*}

\subsubsection{Retta parallela ad una data e passante per un punto $P(x_P,y_P)$}
\begin{equation*}
  y-y_P = m(x-x_P)
\end{equation*}

\subsubsection{Retta perpendicolare ad una data e passante per un punto $P(x_P,y_P)$}
\begin{equation*}
  y - y_P = -\frac{1}{m} (x - x_P)
\end{equation*}

\subsubsection{Distanza $d$ tra un punto $P(x_P,y_P)$ e una retta}
Sia $r$ una retta in forma esplicita $ax+by+c$ e $P$ un punto del piano, allora la distanza minima
tra le due è
\begin{equation*}
  d = \frac{\abs{ax_P + by_P +c}}{\sqrt{a^2+b^2}}
\end{equation*}

\subsubsection{Coefficiente angolare $m$ di una retta passante per due punti $P_1(x_1,y_1)$ e 
$P_2(x_2,y_2)$}
\begin{equation*}
  m=\frac{y_2-y_1}{x_2-x_1}
\end{equation*}

\subsection{Fasci di Rette}\label{subsec:geomanal:fasciorette}
Un fascio di rette è una combinazione lineare di tutte le rette generabili modificando un solo 
parametro di una quantità costante.

\subsubsection{Fascio di rette a due parametri}
Sceglti appropriati $\alpha$ e $\beta$ si possono generare tutte le rette possibili utilizzando questa
forma
\begin{equation*}
  \alpha(ax + by + c) + \beta(a_1x + b_1y + c_1) = 0
\end{equation*}
\begin{equation*}
  (\alpha a + \beta a_1)x + (\alpha b + \beta b_1)y + \alpha c + \beta c_1 = 0
\end{equation*}

\subsubsection{Fascio di rette ad un parametro}
Questa forma esclude una sola retta, per $k = 0$.
\begin{equation*}
  ax + by + c + k(a_1x + b_1y + c_1) = 0
\end{equation*}
Si noti che $k = \frac{\beta}{\alpha}$

\subsubsection{$k$ avendo una retta del fascio, la retta esclusa e un punto su $a_1x + b_1y + c = 0$}
\begin{center}
  \begin{tikzpicture}
    \begin{axis}[xmin=-3,ymin=-3,xmax=3,ymax=3]
      \coordinate (O) at (0,0);
      \node[fill=white,circle,inner sep=0pt] (O-label) at ($(O)+(-135:10pt)$) {$O$};
      \addplot[thick,dashed,teal]{1.7}node[pos=0.3,above,text width=2cm]{Esclusa: $a_1x+b_1y+c=0$};
      \addplot[thick, brown] {3*x}node[pos=0.45,left]{$r:\,ax+by+c=0$};
      \addplot[thick, red] {-4*x+4};
      \LabelPoint[mark=*,color=blue][right]{1.5}{-2}{$Q(x_Q,y_Q)$}
    \end{axis}
    %\draw[teal, dashed] (0,-0.15) -- (3,-0.15)
    %        node[pos=0, above left]{Esclusa: $a_1x + b_1y + c = 0$};
    %\draw[brown] (0,-3) -- (2,1)
    %        node[pos=0, below]{$r: ax + by + c = 0$};
    %\draw[red] (2.5,-3) -- (1,1)
    %        node[pos = 0.5, left]{$k =?$};
    %\filldraw[blue] (2,-1.65) circle(0.05)
    %        node[right]{$Q(x_Q, y_Q)$};
  \end{tikzpicture}
\end{center}
\begin{equation*}
  \mathcolor{red}{k} = -\frac{
    \mathcolor{brown}{a}\mathcolor{blue}{x_Q} + \mathcolor{brown}{b}\mathcolor{blue}{y_Q} +
    \mathcolor{brown}{c}
    }{
    \mathcolor{teal}{a_1}\mathcolor{blue}{x_Q} + \mathcolor{teal}{b_1}\mathcolor{blue}{y_Q} +
    \mathcolor{teal}{c_1}
  }
\end{equation*}

\subsubsection{Retta di un fascio con coefficiente angolare $m$ passante per un punto $P(x_P,y_P)$}
\begin{equation*}
  y-y_P = m(x-x_P)
\end{equation*}

\subsection{Circonferenza}\label{subsec:geomana:circ}
La circonferenza è una conica i cui punti sono tutti equidistanti dal centro $C$.
\begin{center}
  \begin{tikzpicture}
    \begin{axis}[xmin=-1,ymin=-1,xmax=1,ymax=1]
      \coordinate (O) at (0,0);
      \node[fill=white,circle,inner sep=0pt] (O-label) at ($(O)+(-135:10pt)$) {$O$};
      \draw[thick, blue] (0,0) circle (1);
      \draw[thick, red, ->] (0,0) -- (1,0)
        node[pos=0.5, above]{$r$};
    \end{axis}
  \end{tikzpicture}
\end{center}
Le equazioni delle circonferenze hanno 2 forme
\begin{equation*}
  \mathscr{C}:\,(x-\mathcolor{blue}{x_C})^2 + (y-\mathcolor{blue}{y_C})^2 = \mathcolor{red}{r}^2
\end{equation*}
\begin{equation*}
  \mathscr{C}:\,x^2+y^2+ax+by+c =0
\end{equation*}
Da queste due formule derivano le coordinate del centro
\begin{equation*}
  C\left(-\frac{a}{2},-\frac{b}{2}\right)
\end{equation*}
e la misura del raggio
\begin{equation*}
  \mathcolor{red}{r} = \sqrt{\mathcolor{blue}{x_C}^2+\mathcolor{blue}{y_C}^2-c} = 
  \sqrt{\frac{a^2}{4}+\frac{b^2}{4}-c}
\end{equation*}

\subsubsection{Tangente in $P(x_P,y_P)$}
\begin{equation*}
  x\cdot x_P+y\cdot y_P+a\frac{x+x_P}{2}+b\frac{y+y_P}{2}+c = 0
\end{equation*}

\subsubsection{Area del cerchio}
\begin{equation*}
  \mathscr{A}(\mathscr{C}) = \pi\mathcolor{red}{r}^2
\end{equation*}

\subsubsection{Lunghezza della circonferenza}
\begin{equation*}
  C = 2\pi\mathcolor{red}{r}
\end{equation*}

\subsubsection{Lunghezza dell'arco}
\begin{equation*}
  l = \mathcolor{red}{r}\alpha
\end{equation*}
Si noti che $\alpha$ è in radianti.

\subsubsection{Area del settore}
\begin{equation*}
  \mathscr{A}(\mathscr{S}) = \frac{1}{2}\mathcolor{red}{r}^2\alpha
\end{equation*}
Si noti che $\alpha$ è in radianti.

\subsection{Fasci di circonferenze}\label{subsec:geomanal:fasciocirc}
Un fascio di circonferenze è una combinazione lineare di utte le circonferenze generabili modificando
un parametro di una certa quantità costante.

\subsubsection{Fascio di circonferenze ad un parametro}
Scelti appropriati $\alpha$ e $\beta$ si possono generare tute le circonferenze possibili utilizzando 
questa forma
\begin{equation*}
  \alpha(x^2+y^2+a_1x+b_1y+c_1) + \beta(x^2+y^2+a_2x+b_2y+c_2) = 0
\end{equation*}
\begin{equation*}
  (\alpha+\beta)x^2+(\alpha+\beta)y^2+(\alpha a_1+\beta a_2)x+(\alpha b_1+\beta b_2)y+
  \alpha c_1+\beta c_2 = 0
\end{equation*}

\subsubsection{Fascio di circonferenze a due parametri}
Questa forma esclude una circonferenza per $k=0$.
\begin{equation*}
  x^2+y^2+ax+by+c+k(x^2+y^2+a_1x+b_1y+c_1) = 0
\end{equation*}
Si noti che $k=\frac{\alpha}{\beta}$

\subsection{Parabola}\label{subsec:geomanal:parabola}
\begin{center}
  \begin{tikzpicture}
    \begin{axis}[xmin=-2,ymin=-2,xmax=4,ymax=4]
      \coordinate (O) at (0,0);
      \node[fill=white,circle,inner sep=0pt] (O-label) at ($(O)+(-135:10pt)$) {$O$};
      \addplot[red,thick,samples=1000] {x^2};
      \addplot[blue,thick,samples=1000] (x*x,x);
    \end{axis}

    \begin{scope}[shift={(2,-4)}]
      \draw (0,0) -- ++(2,0);
      \draw[dashed] (1,3) -- ++(0,-4);
      \draw[red,thick] plot[domain=0:2] (\x, {(\x-1)^2+1});
      \filldraw[red] (1,1)
        node[below left]{$V$};
      \filldraw[red] (1,1.5)
        node[above right]{$F$};
    \end{scope}
  \end{tikzpicture}
\end{center}
Una parabola può essere descritta con l'asse focale parallelo all'asse $x$ o all'asse $y$.
\begin{equation*}
  \mathcolor{red}{\mathscr{P}}:\,y=ax^2+bx+c
\end{equation*}
\begin{equation*}
  \mathcolor{blue}{\mathscr{P}}:\,x=ay^2+by+c
\end{equation*}
La direttrice di una parabola è quella che ne da l'inclinazione ed è perpendicolare all'asse di
simmetria.\\
Il vertice di una parabola è il punto più vicino alla direttrice.\\
Il fuoco è il punto la cui distanza da qualsiasi punto della parabola è pari a quella della proiezione
sulla direttrice del punto stesso.

\subsubsection{Elementi di una parabola con asse focale parallelo a $x$}
\begin{center}
  \begin{tabular}{c | c}
    \textbf{Fuoco} & $\left(-\dfrac{b}{2a},\dfrac{1-\Delta}{4a}\right)$\\\hline
    \textbf{Vertice} & $\left(-\dfrac{b}{2a}, -\dfrac{\Delta}{4a}\right)$\\\hline
    \textbf{Direttrice} & $y=-\dfrac{1+\Delta}{4a}$\\\hline
    \textbf{Asse di simmetria} & $x=-\dfrac{b}{2a}$\\\hline
    \textbf{Tangente in un punto} & $\dfrac{y+y_0}{2}=axx_0+b\dfrac{x+x_0}{2}+c$
  \end{tabular}
\end{center}

\subsubsection{Elementi di una parabola con asse focale parallelo a $y$}
\begin{center}
  \begin{tabular}{c | c}
    \textbf{Fuoco} & $\left(\dfrac{1-\Delta}{4a},-\dfrac{b}{2a}\right)$\\\hline
    \textbf{Vertice} & $\left(-\dfrac{\Delta}{4a},-\dfrac{b}{2a}\right)$\\\hline
    \textbf{Direttrice} & $x=-\dfrac{1+\Delta}{4a}$\\\hline
    \textbf{Asse di simmetria} & $y=-\dfrac{b}{2a}$\\\hline
    \textbf{Tangente in un punto} & $\dfrac{x+x_0}{2}=ayy_0+b\dfrac{y+y	_0}{2}+c$
  \end{tabular}
\end{center}

\subsubsection{Parabole di vertice $V(x_V,y_V)$}
\begin{equation*}
  y-y_V=a(x-x_V)^2
\end{equation*}

\subsubsection{Area di un segmento parabolico}
\begin{center}
  \begin{tikzpicture}
    \begin{axis}[
      xmin=-1.5,
      ymin=-1,
      xmax=2.5,
      ymax=6,
      % to be able to draw the orange filling on another layer
      set layers,
      ]

      % draw the function and the "intersection lines"
      % (please note that I have changed the number of samples and added
      %  the option `smooth' to avoid some numerical instabilities for `f')
      \addplot [name path=f,red,thick,samples=49,smooth] {x^2};
      \addplot [name path=l1,thick] {1/3*x+1.5};
      \addplot [name path=l2,thick] {1/3*x};
      \addplot [name path=p1,thick,green] {-3*x-2.1};
      \addplot [name path=p2,thick,green] {-3*x+6.2};

      % find the intersection points of the black and green lines
      \path [
        name intersections={
          of=l1 and p1,
          by={A},
        },
        ];
      \path [
        name intersections={
          of=l1 and p2,
          by={B},
        },
        ];
      \path [
        name intersections={
          of=l2 and p2,
          by={C},
        },
        ];
      \path [
        name intersections={
          of=l2 and p1,
          by={H},
        },
        ];

      % create coordinate at origin
      \coordinate (O) at (0,0);

      % create invisible clip paths which are needed for the orange filling
      \path [name path=clippath1] (A) -- (H) -- (C) -- cycle;
      \path [name path=clippath2] (O) -- (C) -- (B) -- cycle;

      % draw the intersection points
      \pgfmathsetlengthmacro{\Radius}{2pt}
      \fill
      (A) circle (\Radius)
      (B) circle (\Radius)
      (C) circle (\Radius)
      (H) circle (\Radius)
      ;

      % label the intersection points
      \node [coordinate,label=below right:$O$] at (O) {};
      \node [coordinate,label=above right:$A$] at (A) {};
      \node [coordinate,label=above right:$B$] at (B) {};
      \node [coordinate,label=below right:$C$] at (C) {};
      \node [coordinate,label=below left:$H$] at (H) {};

      % fill the area between the intersection points on a lower layer
      % so the red function line doesn't have to be plotted twice
      \begin{pgfonlayer}{axis ticks}
        % left half
        \fill [
        orange,
        fill opacity=0.5,
        % (this is the TikZ equivalent to PGFPlots `fill between')
        intersection segments={
          of=f and clippath1,
          % (here we can draw -- in general -- an arbitrary path
          %  between the path elements of intersection points.
          %  Of course here we want to find the path that surrounds
          %  the area that we want to fill.)
          sequence={R1[reverse] -- L2},
        },
        ];
        % right half
        \fill [
        orange,
        fill opacity=0.5,
        intersection segments={
          of=f and clippath2,
          sequence={R{-2} -- L{-2}[reverse]},
        },
        ];
      \end{pgfonlayer}

      % draw the blue filling
      \addplot [
      fill=none,
      ] fill between [
      of=f and l1,
      split,
      every segment no 1/.style={
        fill=blue,
        fill opacity=0.5,
      },
      ];

      %        % ---------------------------------------------------------------------
      %        % for debugging purpose only
      %        % ---------------------------------------------------------------------
      %        % To find the right `sequence' you can play with the elements.
      %        % Just start with one single element like `R1' to see what happens and
      %        % then replace them until you found the right ones and connect them in
      %        % the right order.
      %        \draw [
      %            blue,
      %            very thin,
      %            |->,
      %            intersection segments={
      %                of=f and clippath1,
      %                sequence={
      %                    % Because we know that the "green/black" line is needed
      %                    % from the start to the first intersection point, for sure
      %                    % we need `R1'.
      %                    % And we also know that we need for the "red" line the part
      %                    % from the first (not real) intersection point above (left)
      %                    % of point A (crossing of the green and red line) to the
      %                    % second intersection point (at point O)
      %                    % (There is still some magic left why there is this "not
      %                    %  real" intersection point ...)
      %                    R1[reverse] -- L2
      %%                    % so the reverse path is also fine, which can be done by
      %%                    % reversing the "pathes" ...
      %%                    R1 -- L2[reverse]
      %%                    % ... or the elements of the pathes which offers another
      %%                    % two possibilities to do this
      %%                    L2 -- R1[reverse]
      %%                    L2[reverse] -- R1
      %%                    % Another possibility to avoid the `[reverse] you could
      %%                    % simply reverse the path directly `clippath1' from
      %%                    %     (A) -- (H) -- (C) -- cycle
      %%                    % to
      %%                    %     (C) -- (H) -- (A) -- cycle
      %%                    % in the (above) definition of that path.
      %%                    % Can you imagine how the right elements and sequence is then?
      %%                    % (One tip: It is not as simple as `L2 -- R1')
      %                },
      %            },
      %        ];
      %        \draw [
      %            blue,
      %            very thin,
      %            |->,
      %            intersection segments={
      %                of=f and clippath2,
      %                sequence={
      %%                    % try to find the right elements and orders here yourself
      %                    R{-1}
      %                },
      %            },
      %        ];
      %        % ---------------------------------------------------------------------

    \end{axis}
  \end{tikzpicture}
\end{center}
\begin{equation*}
  \mathscr{A}(\mathscr{F}) = \frac{2}{3}\overline{AB}\cdot\overline{AH}
\end{equation*}
E ovviamente l'area esterna alla curva sarebbe
\begin{equation*}
  \mathscr{A}(\mathscr{F}') = \frac{1}{3}\overline{AB}\cdot\overline{AH}
\end{equation*}

\subsubsection{Formule di sdoppiamento}
Le formule di sdoppiamento servono per determinare le tangenti in un punto $P(x_0,y_0)$.\\
Se $d\|y$
\begin{equation*}
  \frac{y+y_0}{2}=axx_0+b\frac{x+x_0}{2}+c
\end{equation*}
Se $d\|x$
\begin{equation*}
  \frac{x+x_0}{2}=ayy_0+b\frac{y+y_0}{2}+c
\end{equation*}

\subsubsection{Coefficiente angolare della tangente}
\begin{equation*}
  m = \frac{1}{2ay_0+b} = 2ax_0+b
\end{equation*}

\subsection{Ellisse}\label{subsec:geomanal:ellisse}
\begin{center}
  \begin{tikzpicture}
    \begin{axis}[xmin=-2,ymin=-2,xmax=2,ymax=2]
      \coordinate (O) at (0,0);
      \node[fill=white,circle,inner sep=0pt] (O-label) at ($(O)+(-135:10pt)$) {$O$};
      \draw[thick,red] (axis cs:0,0) ellipse[x radius=2, y radius=1];
      \LabelPoint[mark=*][above]{-1}{0}{$F_1$}
      \LabelPoint[mark=*][above]{1}{0}{$F_2$}
    \end{axis}
  \end{tikzpicture}
\end{center}
Un'ellissi ha due assi, uno maggiore uno minore. Le loro semi-lunghezze (quindi i semi-assi) si 
denominano $a$ (che contiene i fuochi) e $b$.\\
I fuochi sono i due punti tali che preso un punto $P\in\mathscr{E}$, 
$\overline{PF_1}+\overline{PF_2} = 2a$.
\begin{equation*}
  \mathscr{E}:\,\frac{x^2}{a^2}+\frac{y^2}{b^2}=1
\end{equation*}
Tra i semi-assi vige la seguente proprietà
\begin{equation*}
  a^2-c^2=b^2
\end{equation*}
e quindi
\begin{equation*}
  c = \begin{cases}
    a^2-b^2,\, &\text{se } a > b\\
    b^2-a^2,\, &\text{se } a < b
  \end{cases}
\end{equation*}

\subsubsection{Eccentricità}
L'eccentricità è lo schiacciamento dell'ellisse sull'asse maggiore. È un valore compreso tra $0$ e 
$1$.\\
Se $a>b$
\begin{equation*}
  e = \frac{c}{a} = \frac{\sqrt{a^2-b^2}}{a}=\sqrt{1-\frac{b^2}{a^2}}
\end{equation*}
Se $b>a$
\begin{equation*}
  e = \frac{c}{b} = \sqrt{1-\frac{a^2}{b^2}}
\end{equation*}

\subsubsection{Area dell'ellisse}
\begin{equation*}
  \mathscr{A}(\mathscr{E}) = ab\pi
\end{equation*}

\subsubsection{Tangenti all'ellisse}
Per trovare la tangente all'esllise abbiamo due modi:
\begin{equation*}
  \frac{xx_0}{a^2}+\frac{yy_0}{b^2}=1
\end{equation*}
oppure fare il sistema tra la retta generica per $P$ e fare in modo che il discriminante si annulli:
\begin{align*}
  \begin{dcases}
    \frac{x^2}{a^2}+\frac{y^2}{b^2}=1\\
    y-y_0=m(x-x_0)
  \end{dcases}\rightarrow\\ \frac{\Delta}{4}=a^4m^2q^2-a^2(q^2-b^2)(b^2+a^2m^2) = 0
\end{align*}
Il vantaggio di questo secondo metodo è che può anche trovare le rette secanti ed esterne all'ellisse
(rispettivamente con $\dfrac{\Delta}{4}>0$ e $\dfrac{\Delta}{4}<0$). È sicuramente più laborioso e
difficile da ricordare.

\subsection{Iperbole}
\begin{center}
  \begin{tikzpicture}
    \begin{axis}[xmin=-5,ymin=-5,xmax=5,ymax=5]
      \coordinate (O) at (0,0);
      \node[fill=white,circle,inner sep=0pt] (O-label) at ($(O)+(-135:10pt)$) {$O$};
      \addplot[red,thick,domain=-2:2] ({cosh(x)}, {sinh(x)});
      \addplot[red,thick,domain=-2:2] ({-cosh(x)}, {sinh(x)});
      \addplot[red,dashed] expression {x};
      \addplot[red,dashed] expression {-x};
    \end{axis}
  \end{tikzpicture}
\end{center}
L'iperbole può essere descritta in più modi.\\
I fuochi sono i due punti tali che per un punto $P\in\mathscr{I}$, 
$\lvert \overline{PF_1}-\overline{PF_1}\rvert=2a$.\\
L'equazione dell'iperbole con i fuochi su $x$ è
\begin{equation*}
  \mathscr{I}:\,\frac{x^2}{a^2}-\frac{y^2}{b^2}=1
\end{equation*}
Quella con i fuochi su $y$ è
\begin{equation*}
  \mathscr{I}:\,\frac{x^2}{a^2}-\frac{y^2}{b^2}=-1
\end{equation*}
Tra i parametri $a$ e $b$ vige che $a < c$ e $c^2 = a^2+b^2$.

\subsubsection{Asintoti}
Gli asintoti sono le rette che l'iperbole tende a raggiungere senza mai toccare
\begin{equation*}
  y=\pm\frac{b}{a}x
\end{equation*}

\subsubsection{Eccentricità}
L'eccentricità dell'iperbole è il rapporto
\begin{equation*}
  e=\frac{c}{a}=\sqrt{1+\frac{b^2}{a^2}}
\end{equation*}
se l'iperbole ha i fuochi su $x$,
\begin{equation*}
  e = \frac{c}{b}=\sqrt{1+\frac{a^2}{b^2}}
\end{equation*}
altrimenti. Si noti anche che $e > 1$ per ogni iperbole.

\subsubsection{Iperbole equilatera}
Se $a=b$, l'iperbole si definisce equilatera e le equazioni diventano 
\begin{equation*}
  x^2-y^2=a^2
\end{equation*}
se $F\in x$,
\begin{equation*}
  y^2-x^2=a^2
\end{equation*}
se $F\in y$.\\
Questo comporta che $c = a\sqrt{2}$ e che $e=\sqrt{2}$.\\\\
Può anche essere descritta l'iperbole da
\begin{equation*}
  xy=k
\end{equation*}

\subsubsection{Formule di sdoppiamento}
Vengono ora riportate le formule di sdoppiamento che cambiano in base all'equazione dell'iperbole
\begin{center}
  \begin{tabular}{c|c}
    Equazione & Tangente\\\hline
    $\dfrac{x^2}{a^2}-\dfrac{y^2}{b^2}=1$ & $\dfrac{xx_0}{a^2}-\dfrac{yy_0}{b^2}=1$\\\hline
    $\dfrac{x^2}{a^2}-\dfrac{y^2}{b^2}=-1$ & $\dfrac{xx_0}{a^2}-\dfrac{yy_0}{b^2}=-1$\\\hline
    $x^2-y^2=a^2$ & $xx_0-yy_0=a^2$\\\hline
    $x^2-y^2=-a^2$ & $xx_0-yy_0=-a^2$\\\hline
    $xy=k$ & $xx_0\cdot yy_0=k$
  \end{tabular}
\end{center}

\subsubsection{Iperbole equilatera traslata}
Si trova molto spesso una versione traslata di un'iperbole. Questa è la sua generale forma
\begin{equation*}
  y=\frac{ax+b}{cx+d}
\end{equation*}
E gli asintoti sono
\begin{equation*}
  x=-\frac{d}{c}\qquad y=\frac{a}{c}
\end{equation*}
con il centro di simmetria
\begin{equation*}
  O\left(-\frac{d}{c},\frac{a}{c}\right)
\end{equation*}

%!TEX ROOT=formularioMatematica.tex

\section{Goniometria}\label{sec:goniometria}
La gonioetria si incentra tutta sulla \emph{circonferenza goniometrica} che non è altro che una
circonferenza di centro $O(0;0)$ e di raggio $r=1$.\\
Per convenzione, gli angoli vengono definiti a partire dall'asse $x$ e si definiscono positivi quando
proseguono in senso antiorario. Si noti che $2\pi = \ang{360}$.
\begin{center}
  \begin{tikzpicture}		
    \begin{axis}[xmin=-2,ymin=-2,xmax=2,ymax=2]
      \coordinate (O) at (0,0);
      \node[fill=white,circle,inner sep=0pt] (O-label) at ($(O)+(-135:10pt)$) {$O$};
      \draw (axis cs:0,0) circle[radius=1];
      \coordinate (A) at (1,0);
      \coordinate (O) at (0,0);
      \coordinate (P) at (0.866,0.5);
      \markangle[blue]{O}{P}{A}{0.5}{1.5}{$\alpha$}
      \draw[dashed, blue] (O) -- (P);	
      \filldraw (0.866,0.5) circle (0.03)
        node[blue, above right]{$P(\cos\alpha, \sin\alpha)$};
    \end{axis}
  \end{tikzpicture}
\end{center}
Già nella figura identifichiamo le due funzioni fondamentali della goniometria: $\cos$ e $\sin$. Esse,
numericamente, rappresentano rispettivamente l'ascissa e l'ordinata del punto $P$ al variare di 
$\alpha$.\\
Seno e coseno non sono le uniche funzioni goniometriche, esistono infatti anche
\begin{align*}
  \tan\alpha &= \frac{\sin\alpha}{\cos\alpha}\\
  \cot\alpha &= \frac{1}{\tan\alpha}\\
  \sec\alpha &= \frac{1}{\cos\alpha}\\
  \csc\alpha &= \frac{1}{\sin\alpha}
\end{align*}
Da notare che spesso $\csc$ si trova anche scritto nella forma più estesa $\mathrm{cosec}$.\\
$\sin$ e $\cos$ rappresentano anche in un triangolo rettangolo
\begin{gather*}
  \cos\alpha = \frac{\text{Lunghezza cateto adiacente}}{\text{Lunghezza ipotenusa}}\\
  \sin\alpha = \frac{\text{Lunghezza cateto opposto}}{\text{Lunghezza ipotenusa}}
\end{gather*}
Per gli esercizi si vada a pagina~\pageref{ex:goniometria}.

\subsection{Angoli particolari}
Seno e coseno sono funzioni periodiche, ovvero che il loro valore sta all'interno di un insieme e si
ripete con un certo periodo.\\
Gli angoli particolari principali sono
\begin{alignat*}{2}
  \sin\frac{\pi}{6} &= \frac{1}{2} &\qquad \cos\frac{\pi}{6} &= \frac{\sqrt{3}}{2}\\
  \sin\frac{\pi}{4} &= \frac{\sqrt{2}}{2} & \cos\frac{\pi}{4} &= \frac{\sqrt{2}}{2}\\
  \sin\frac{\pi}{3} &= \frac{\sqrt{3}}{2} & \cos\frac{\pi}{3} &= \frac{1}{2}
\end{alignat*}
Come si può notare ricordarli è piuttosto semplice: $\dfrac{\pi}{3}$ e $\dfrac{\pi}{6}$ sono gli 
angoli di un triangolo equilatero, $\dfrac{\pi}{4}$ è la diagonale di un quadrato.

\subsection{Relazione fondamentale}
La relazione fondamentale è quella che permetterà di trovare molte delle formule successive. Essa è
\begin{equation*}
  \cos^2\alpha + \sin^2\alpha = 1
\end{equation*}

\subsection{Grafico delle funzioni}
\subsubsection{$\cos\alpha$}
\begin{center}
  \begin{tikzpicture}
    \begin{axis}[xmin=-8,ymin=-1,xmax=8,ymax=1,
      xtick={-6.28318,-4.7123889,-3.14159,-1.5708,0,1.5708,3.14159,4.7123889,6.28318},
      xticklabels={$-2\pi$,$-\frac{3\pi}{2}$,$-\pi$,$-\frac{\pi}{2}$,
      $0$,$\frac{\pi}{2}$,$\pi$,$\frac{3\pi}{2}$,$2\pi$},
      domain=-2*pi:2*pi]
      \coordinate (O) at (0,0);
      \node[fill=white,circle,inner sep=0pt] (O-label) at ($(O)+(-135:10pt)$) {$O$};
      \addplot[red,thick,smooth,samples=500] {cos(deg(x))}; 
    \end{axis}
  \end{tikzpicture}
\end{center}
\subsubsection{$\sin\alpha$}
\begin{center}
  \begin{tikzpicture}
    \begin{axis}[xmin=-8,ymin=-1,xmax=8,ymax=1,
      xtick={-6.28318,-4.7123889,-3.14159,-1.5708,0,1.5708,3.14159,4.7123889,6.28318},
      xticklabels={$-2\pi$,$-\frac{3\pi}{2}$,$-\pi$,$-\frac{\pi}{2}$,
      $0$,$\frac{\pi}{2}$,$\pi$,$\frac{3\pi}{2}$,$2\pi$},
      domain=-2*pi:2*pi]
      \coordinate (O) at (0,0);
      \node[fill=white,circle,inner sep=0pt] (O-label) at ($(O)+(-135:10pt)$) {$O$};
      \addplot[red,thick,smooth,samples=500] {sin(deg(x))}; 
    \end{axis}
  \end{tikzpicture}
\end{center}
\subsubsection{$\tan\alpha$}
\begin{center}
  \begin{tikzpicture}
    \begin{axis}[xmin=-8,ymin=-1,xmax=8,ymax=1,
      xtick={-6.28318,-4.7123889,-3.14159,-1.5708,0,1.5708,3.14159,4.7123889,6.28318},
      xticklabels={$-2\pi$,$-\frac{3\pi}{2}$,$-\pi$,$-\frac{\pi}{2}$,
      $0$,$\frac{\pi}{2}$,$\pi$,$\frac{3\pi}{2}$,$2\pi$},
      domain=-2*pi:2*pi]
      \coordinate (O) at (0,0);
      \node[fill=white,circle,inner sep=0pt] (O-label) at ($(O)+(-135:10pt)$) {$O$};
      \addplot[red,thick,smooth,samples=500,restrict y to domain=-6:6] {tan(deg(x))}; 
    \end{axis}
  \end{tikzpicture}
\end{center}

\subsection{Funzioni inverse}
Ovviamente se da un angolo possiamo ottenere un numero, possiamo fare anche il contrario. Per indicare
le funzioni inverse abbiamo due possibilità
\begin{enumerate}
  \item Scrivere $f^{-1}(x)$
  \item Dare un nuovo nome alla funzione
\end{enumerate}

Nelle calcolatrici è molto più comune trovare $\sin^{-1}$ e gli altri però non sono precisi e quindi
sarebbe da preferire $\arcsin$ o $\mathrm{asin}$ per brevità. Il motivo è che se
\begin{equation*}
  f:\;A\mapsto B,\,f^{-1}:\;B\mapsto A
\end{equation*}
però per le funzioni goniometriche questo non accade infatti
\begin{equation*}
  \sin x:\;\mathbb{R}\mapsto{[{-1},{+1}]} \qquad \cos x:\;\mathbb{R}\mapsto{[{-1},{1}]}
\end{equation*}
quando però
\begin{equation*}
  \arcsin x:\;{[{-1},{1}]}\mapsto{\left[{-\frac{\pi}{2}},{\frac{\pi}{2}}\right]} \quad 
  \arccos:\;{[{-1},{1}]}\mapsto{[0,{\pi}]}
\end{equation*}
I grafici sono i seguenti
\subsubsection{$\arccos x$}
\begin{center}
  \begin{tikzpicture}
    \begin{axis}[xmin=-1,ymin=0,xmax=1,ymax=3,ytick={1.5707},yticklabels={$\frac{\pi}{2}$}]
      \coordinate (O) at (0,0);
      \node[fill=white,circle,inner sep=0pt] (O-label) at ($(O)+(-135:10pt)$) {$O$};
      \addplot[red,thick,domain=-1:1,smooth,samples=500] {rad(acos(x))};
    \end{axis}
  \end{tikzpicture}
\end{center}
\subsubsection{$\arcsin x$}
\begin{center}
  \begin{tikzpicture}
    \begin{axis}[xmin=-1,ymin=-2,xmax=1,ymax=2]
      \coordinate (O) at (0,0);
      \node[fill=white,circle,inner sep=0pt] (O-label) at ($(O)+(-135:10pt)$) {$O$};
      \addplot[red,thick,domain=-1:1,smooth,samples=500] {rad(asin(x))};
    \end{axis}
  \end{tikzpicture}
\end{center}

\subsection{Formule goniometriche}
Le formule goniometriche permettono di pasare da una funzione ad un'altra. Una delle caratteristiche 
più importanti è l'esistenza dei così denominati \textbf{angoli associati}. Essi sono angoli 
particolari che assumono valori facili da scambiare e ricordare. Essi sono
\begin{alignat*}{2}
  \cos(\pi\pm x) & = -\cos x &\qquad \cos(-x) &= \cos x\\
  \sin(\pi\pm x) &= \mp\sin x & \sin(-x) &= -\sin x\\
  \tan(\pi\pm x) &= \pm\tan x & \tan(-x) &= -\tan x
\end{alignat*}
In associazione a questi che sono i più comuni, sono anche presenti i seguenti
\begin{alignat*}{2}
  \cos\left(\frac{\pi}{2}\pm x\right) &= \mp\sin x &\qquad \cos\left(\frac{3}{2}\pi\pm x\right) = 
  \pm\sin x\\
  \sin\left(\frac{\pi}{2}\pm x\right) &= \pm\cos x &\qquad \sin\left(\frac{3}{2}\pi\pm x\right) = 
  -\cos x\\
  \tan\left(\frac{\pi}{2}\pm x\right) &= \mp\cot x &\qquad \tan\left(\frac{3}{2}\pi\pm x\right) = 
  \mp\cot x\\
\end{alignat*}
Si presti molta attenzione ai segni in quanto è molto facile confondersi.

\subsubsection{Addizione e sottrazione}
\begin{align*}
  \cos(\gamma\pm\theta) &= \cos\gamma\cos\theta\mp\sin\gamma\sin\theta\\
  \sin(\gamma\pm\theta) &= \sin\gamma\cos\theta\pm\cos\gamma\sin\theta\\
  \tan(\gamma\pm\theta) &= \frac{\tan\gamma\pm\tan\theta}{1\mp\tan\gamma\tan\theta}
\end{align*}

\subsubsection{Duplicazione}
\begin{alignat*}{2}
  \sin2x & =2\sin x\cos x &\qquad \cos2x &= \begin{cases}
    \cos^2x - \sin^2x\\
    1-2\sin^2x\\
    2\cos^2x-1
  \end{cases}\\
  \tan2x &= \frac{2\tan x}{1-\tan^2x} & \cot2 &= \frac{\cot^2-1}{2\cot x}
\end{alignat*}

\subsubsection{Bisezione}
\begin{alignat*}{2}
  \sin\frac{x}{2}&=\pm\sqrt{\frac{1-\cos x}{2}} &\qquad \cos\frac{x}{2}&=\pm\sqrt{\frac{1+\cos x}{2}}\\
  \tan\frac{x}{2}&=\begin{dcases}
    \frac{\sin x}{1+\cos x}\\
    \frac{1-\cos x}{\sin x}
  \end{dcases} & \cot\frac{x}{2} &= \begin{dcases}
    \frac{1+\cos x}{\sin x}\\
    \frac{\sin x}{1-\cos x}
  \end{dcases}
\end{alignat*}
Il segno nella prima riga è da scegliersi $+$ se 
$\sin\frac{x}{2}\geq0\lor\cos\frac{x}{2}\geq0$, $-$ altrimenti.

\subsubsection{Parametriche}
Per queste formule poniamo
\begin{equation*}
  t = \tan\frac{x}{2}
\end{equation*}
per comodità.
\begin{alignat*}{2}
  \sin x &= \frac{2t}{1+t^2} &\qquad \cos x &= \frac{1-t^2}{1+t^2}\\
  \tan x &= \frac{2t}{1-t^2} & \cot x &= \frac{1-t^2}{2t}
\end{alignat*}

\subsubsection{Prostaferesi}
\begin{align*}
  \sin p + \sin q &= 2\sin\frac{p+q}{2}\cos\frac{p-q}{2}\\
  \sin p-\sin q &=2\cos\frac{p+q}{2}\sin\frac{p-q}{2}\\
  \cos p+\cos q&=2\cos\frac{p+q}{2}\cos\frac{p-q}{2}\\
  \cos p-\cos q&=-2\sin\frac{p+q}{q}\sin\frac{p-q}{2}
\end{align*}

\subsubsection{Werner}
\begin{align*}
  \cos\gamma\cos\theta &= \frac{1}{2}[\cos(\gamma+\theta)+\cos(\gamma-\theta)]\\
  \cos\gamma\sin\theta &= \frac{1}{2}[\sin(\gamma+\theta)-\sin(\gamma-\theta)]\\
  \sin\gamma\sin\theta &= \frac{1}{2}[\cos(\gamma-\theta)-\cos(\gamma+\theta)]
\end{align*}

\subsection{Equazioni goniometriche}
Si definisce un'equazione goniometrica una qualsiasi equazione che abbia almeno una funzione 
goniometrica e che ha soluzioni solo per particolari angoli.
\subsubsection{$\sin x = m$}
\begin{equation*}
  x = \arcsin m + 2k\pi \lor  x = \arcsin m +2k\pi-\pi
\end{equation*}

\subsubsection{$\cos x = m$}
\begin{equation*}
  x = \pm\arccos m + 2k\pi
\end{equation*}

\subsubsection{$\tan x = m$}
\begin{equation*}
  x = \arctan x + k\pi
\end{equation*}

\subsubsection{Equazioni lineari}
Le equazioni lineari vengono così definite perché sono simili alla forma di una retta implicita.
\begin{equation*}
  a\sin x + b\cos x + c = 0
\end{equation*}
La risoluzione di questa può essere semplice per $b = 0 \lor a = 0$ in quanto si ritorna alle forme 
precedenti. Se invece $a\neq0 \land b\neq0$ si hanno due strade:
\begin{enumerate}
  \item metodo algebrico;
  \item metodo grafico;
\end{enumerate}
Il metodo algebrico è molto lungo e generalmente sconsigliato. In generale si sfrutta la 
parametrizzazione di $\sin$ e $\cos$ in $\tan \frac{x}{2}$.\\
Il metodo grafico consiste nel porre
\begin{equation*}
  \cos x = X\,\text{ e } \sin x = Y
\end{equation*}
e poi risolvere il seguente sistema
\begin{equation*}
  \begin{cases}
    aY + bX + c = 0\\
    X^2 + Y^2 = 1
  \end{cases}
\end{equation*}

\subsubsection{Equazioni omogenee}
Si dicono omogenee se tutti i suoi elementi sono dello stesso grado. Per risolvere queste equazioni
in questa forma, abbiamo varie strade
\begin{itemize}
  \item Se \textbf{è presente il termine di grado $n$ in $\sin x$}, si divide tutto per 
    $\cos^nx\neq0$ ottenendo un'equazione di grado $n$ in $\tan x$ equivalente alla data;
  \item Se \textbf{è presente il termine di grado $n$ in $\cos x$}, si divide tutto per
    $\sin^nx\neq0$ ottenendo un'equazione di grado $n$ in $\cot x$ equivalente alla data;
  \item Se \textbf{nessuno dei precedenti è valido}, si raccolga a fattore comune;
\end{itemize}

\subsection{Teoremi sui triangoli}
\subsubsection{Area di un triangolo qualsiasi}
\begin{center}
  \begin{tikzpicture}
    \coordinate (A) at (1,2);
    \coordinate (B) at (-2,0);
    \coordinate (C) at (3,0);
    \draw (A) -- (B)
      node[pos=0.5,left]{$c$}
      node[pos=0,above]{$A$} -- (C)
      node[pos=0.5,below]{$a$}
      node[pos=0,below]{$B$} -- cycle
      node[pos=0.5,right]{$b$}
      node[pos=0,below]{$C$};
    \markangle{B}{A}{C}{0.3}{1.5}{$\beta$}
    \markangle[blue]{C}{B}{A}{0.3}{1.5}{$\gamma$}
    \markangle[orange]{A}{B}{C}{0.3}{2}{$\alpha$}
  \end{tikzpicture}
\end{center}
\begin{equation*}
  \mathscr{A} = \frac{1}{2}ab\sin\gamma = \frac{1}{2}ac\sin\beta = \frac{1}{2}bc\sin\alpha
\end{equation*}

\subsubsection{Teorema della corda}
\begin{center}
  \begin{tikzpicture}
    \coordinate (A) at (-0.5,0.86);
    \coordinate (B) at (0.98,0.17);
    \coordinate (C) at (-0.17,-0.98);

    \draw (0,0) circle (1);
    \filldraw (A) circle (0.05)
      node[above left]{$A$};
    \filldraw (B) circle (0.05)
      node[right] {$B$};
    \draw[dashed] (A) -- (B) -- (C) -- cycle;
    \draw[thick,red] (A) -- (B);
    \draw[dashed] (0,0) -- (A)
      node[pos=0.5,right]{$R$};
    \markangle{C}{A}{B}{0.3}{1.5}{$\alpha$}
  \end{tikzpicture}
\end{center}
\begin{equation*}
  \overline{AB} = 2R\sin\alpha
\end{equation*}

\subsubsection{Teorema dei seni}
\begin{center}
  \begin{tikzpicture}
    \coordinate (A) at (1,2);
    \coordinate (B) at (-2,0);
    \coordinate (C) at (3,0);
    \draw (A) -- (B)
      node[pos=0.5,left]{$c$}
      node[pos=0,above]{$A$} -- (C)
      node[pos=0.5,below]{$a$}
      node[pos=0,below]{$B$} -- cycle
      node[pos=0.5,right]{$b$}
      node[pos=0,below]{$C$};
    \markangle{B}{A}{C}{0.3}{1.5}{$\beta$}
    \markangle[blue]{C}{B}{A}{0.3}{1.5}{$\gamma$}
    \markangle[orange]{A}{B}{C}{0.3}{2}{$\alpha$}
  \end{tikzpicture}
\end{center}
\begin{equation*}
  \frac{a}{\sin\alpha} = \frac{b}{\sin\beta} = \frac{c}{\sin\gamma} = 2R
\end{equation*}

\subsubsection{Teoremi di Carnot}
\begin{center}
  \begin{tikzpicture}
    \coordinate (A) at (1,2);
    \coordinate (B) at (-2,0);
    \coordinate (C) at (3,0);
    \draw (A) -- (B)
      node[pos=0.5,left]{$c$}
      node[pos=0,above]{$A$} -- (C)
      node[pos=0.5,below]{$a$}
      node[pos=0,below]{$B$} -- cycle
      node[pos=0.5,right]{$b$}
      node[pos=0,below]{$C$};
    \markangle{B}{A}{C}{0.3}{1.5}{$\beta$}
    \markangle[blue]{C}{B}{A}{0.3}{1.5}{$\gamma$}
    \markangle[orange]{A}{B}{C}{0.3}{2}{$\alpha$}
  \end{tikzpicture}
\end{center}
\begin{align*}
  a^2 &= b^2+c^2-2bc\cos\alpha\\
  b^2 &= a^2+c^2-2ac\cos\beta\\
  c^2 &= a^2+b^2-2ab\cos\gamma
\end{align*}

%!TEX ROOT=formularioMatematica.tex

\section{Logaritmi}\label{sec:logaritmi}
Il logaritmo è la seconda funzione inversa della potenza, essendo la prima la radice.\\
Presa un'equazione del tipo
\begin{equation*}
  a^x=b
\end{equation*}
le soluzioni di $x$ si esprimono come
\begin{equation*}
  x = \log_a b
\end{equation*}
quindi si ha anche che 
\begin{equation*}
  a^{\log_a b} = b
\end{equation*}
Si legge ``\textit{logaritmo in base $a$ di $b$}''. Perché un logaritmo esista è necessario che
$a>0\land a\neq1\land b > 0$.\\
Quando si vede scritto $\log$ si intende $\log_{10}$, quando invece è presente $\ln$ si intende
$\log_e$.\\
Per gli esercizi si vada a pagina~\pageref{ex:logaritmi}.

\subsection{Teoremi sui logaritmi}
\subsubsection{Logaritmo del prodotto}
\begin{equation*}
  \log_a (b_1\cdot b_2) = \log_a b_1 + \log_a b_2
\end{equation*}

\subsubsection{Logaritmo del quoziente}
\begin{equation*}
  \log_a\frac{b_1}{b_2} = \log_a b_1 - \log_a b_2
\end{equation*}

\subsubsection{Logaritmo di una potenza}
\begin{equation*}
  \log_a b^k = k\log_a b
\end{equation*}

\subsubsection{Cambiamento di base}
\begin{equation*}
  \log_a b = \frac{\log_c b}{\log_c a}
\end{equation*}
Da questa particolare formula si nota anche che
\begin{equation*}
  \log_{\frac{1}{a}} b = -\log_a b
\end{equation*}

\subsection{Grafici dei logaritmi}
I logaritmi hanno due grafici dipendentemente al valore della base
\subsubsection{$\log_a x$ con $a > 1$}
\begin{center}
  \begin{tikzpicture}
    \tkzInit[xmin=-1,ymin=-2.5,xmax=5,ymax=3]
    \tkzGrid
    \tkzAxeXY
    \draw[red, thick, domain=0.1:5, samples=500] plot({\x}, {ln(\x)});
  \end{tikzpicture}
\end{center}

\subsubsection{$\log_a x$ con $0 < a < 1$}
\begin{center}
  \begin{tikzpicture}
    \tkzInit[xmin=-1,ymin=-2.5,xmax=5,ymax=3]
    \tkzGrid
    \tkzAxeXY
    \draw[red, thick, domain=0.1:5, samples=500] plot({\x}, {ln(\x)/ln(0.5)});
  \end{tikzpicture}
\end{center}
Essendo i logaritmi molto correlati alle funzioni esponenziali, riporto di seguito i loro grafici

\subsubsection{$a^x$ con $a > 1$}
\begin{center}
  \begin{tikzpicture}
    \tkzInit[xmin=-3,ymin=-1,xmax=2,ymax=4.5]
    \tkzGrid
    \tkzAxeXY
    \draw[red, thick, domain=-3:1.5, samples=500] plot({\x}, {exp(\x)});
  \end{tikzpicture}
\end{center}

\subsubsection{$a^x$ con $0 < a < 1$}
\begin{center}
  \begin{tikzpicture}
    \tkzInit[xmin=-2,ymin=-1,xmax=3,ymax=4]
    \tkzGrid
    \tkzAxeXY
    \draw[red, thick, domain=-2:3, samples=500] plot({\x}, {0.5^(\x)});
  \end{tikzpicture}
\end{center}

%!TEX ROOT=formularioMatematica.tex

\section{Progressioni}\label{sec:progressioni}
Le progressioni sono una serie di numeri in modo che tra due numeri successivi ci sia una costante
relazione. Si dividono in \textbf{aritmetiche} e \textbf{geometriche}.\\
Per gli esercizi si vada \hyperref[ex:progressioni]{qui}.

\subsection{Progressioni Aritmetiche}
Le progressioni aritmetiche hanno la caratteristica che la differenza tra due termini successivi �
sempre costante. Questa differenza si chiama \emph{ragione}.
\begin{equation*}
a_n - a_{n-1} = d \quad d = \frac{a_n-1}{n-1}
\end{equation*}
dove $d$ � la ragione e $a_n$ � un elemento qualunque di una progressione.

\subsubsection{$n$-esimo elemento}
\begin{equation*}
a_n = a_1 + d(n-1)
\end{equation*}

\subsubsection{$s$-esimo elemento riferito ad un $r$-esimo elemento}
Questa � considerabile una generalizzazione della formula precedente.
\begin{equation*}
a_s = a_r + d(s-r)
\end{equation*}

\subsubsection{Propriet� di simmetria}
\begin{equation*}
a_1+a_n = a_{k+1}+a_{n-k}\qquad\forall k
\end{equation*}

\subsubsection{Somma di una progressione}
\begin{equation*}
S_n = \frac{a_1+a_n}{2}\cdot n
\end{equation*}

\subsection{Progressioni Geometriche}
Le progressioni geometriche hanno la caratteristica che il rapporto tra due successivi elementi �
costante. Questo rapporto si chiama \emph{ragione}
\begin{equation*}
\frac{a_n}{a_{n-1}} = q \quad q:\;\begin{dcases}
q = \sqrt[n-1]{\frac{a_n}{a_1}} &\text{se concordi, per }n>2\\
q = -\sqrt[n-1]{\left\lvert\frac{a_n}{a_1}\right\rvert} &\text{se discordi}
\end{dcases}
\end{equation*}
dove $q$ � la ragione e $a_n$ � un elemento qualunque di una progressione.

\subsubsection{$n$-esimo elemento}
\begin{equation*}
a_n = a_1\cdot q^{n-1}
\end{equation*}

\subsubsection{$s$-esimo elemento riferito ad un $r$-esimo elemento}
Questa � considerabile una generalizzazione della formula precedente.
\begin{equation*}
a_s = a_s\cdot q^{s-r}
\end{equation*}

\subsubsection{Propriet� di simmetria}
\begin{equation*}
a_1\cdot a_n = a_{k-1}\cdot a_{n-k}
\end{equation*}

\subsubsection{Somma di una progressione}
\begin{equation*}
S_n = a_1\frac{1-q^n}{1-q}
\end{equation*}

\section{Calcolo combinatorio}\label{sec:calccomb}
Il calcolo combinatorio descrive i diversi modi di disporre e organizzare un finito numero di oggetti.
\\Per gli esercizi si vada \hyperref[ex:calccomb]{qui}.

\subsection{Fattoriale}
Un concetto fondamentale del calcolo combinatorio � quello di fattoriale. Esso � definito come
\begin{equation*}
n! = n\cdot (n-1) \cdot (n-2) \dotsm 2\cdot1 = \prod\limits_{1}^{n}n
\end{equation*}

\subsection{Disposizioni}
Due disposizioni si considerano distinte se almeno un elemento � diverso e non tutti devono essere 
presenti. L'ordine � importante.

\subsubsection{Semplici}
\begin{equation*}
D_{n,k} = \frac{n!}{(n-k)!} = n(n-1)\dotsm(n-k+1)
\end{equation*}

\subsubsection{Con ripetizione}
\begin{equation*}
D'_{n,k} = n^k
\end{equation*}

\subsection{Permutazioni}
Due permutazioni si considerano distine se almeno un elemento � diverso.

\subsubsection{Semplici}
\begin{equation*}
P_n = D_{n,n} = n!
\end{equation*}

\subsubsection{Con ripetizione}
\begin{equation*}
P_n^{\alpha_1,\alpha_2,\dotsc,\alpha_n} = \frac{n!}{\alpha_1!\alpha_2!\dotsm\alpha_n!}
\end{equation*}
dove $\alpha_n$ identifica il numero di ripetizioni per il relativo oggetto.

\subsection{Combinazioni}
Le combinazioni rappresentano tutti i gruppi che si possono formare da $n$ elementi considerando 
distinti due gruppi se almeno un elemento � diverso.

\subsubsection{Semplici}
\begin{equation*}
C_{n,k} = \frac{\mathscr{D}_{n,k}}{k!} = \binom{n}{k} = \frac{n!}{k!(n-k)!}
\end{equation*}

\subsubsection{Con ripetizione}
\begin{equation*}
C'_{n,k} = \binom{n+k+1}{k}
\end{equation*}

\subsubsection{Propriet� del coefficiente binomiale}
\paragraph{Simmetria}
\begin{equation*}
\binom{n}{k} = \binom{n}{n-k}
\end{equation*}

\paragraph{Per $1\leq k \leq n-1$}
\begin{equation*}
\binom{n}{k} = \binom{n-1}{k-1}+\binom{n-1}{k}
\end{equation*}

\paragraph{Per $1\leq k\leq n$}
\begin{equation*}
\binom{n}{k} = \binom{n-1}{k-1}+\binom{n-2}{k-1}+\dots+\binom{k}{k-1}+\binom{k-1}{k-1}
\end{equation*}

\paragraph{Binomio di Newton}
\begin{equation*}
(a+b)^n = \sum\limits_{k=0}^{n}\binom{n}{k}a^{n-k}b^k
\end{equation*}
da cui deriva
\begin{equation*}
2^n = \sum\limits_{k=0}^{n}\binom{n}{k}
\end{equation*}

\subsection{Schema riassuntivo}
\begin{figure}[h]
	\centering
	\includegraphics[width=8cm]{image/tree}
	\caption{Si risponda a ciascuna domanda per sapere che tipo di situazione il problema pone.}
\end{figure}
%!TEX ROOT=formularioMatematica.tex

\section{Probabilit�}\label{sec:prob}
La probabilit� � una funzione $p(U)$ che ritorna un valore compreso tra $0$ e $1$ che definisce la 
probilit� di un \emph{evento}.
\begin{equation*}
p:\,p(U) = \frac{\text{Casi favorevoli}}{\text{Casi possibili}}\mapsto {[{0},{1}]}
\end{equation*}

\subsection{Evento ed insieme universo}
Per un qualsiasi caso di studio esiste un insieme \emph{Universo} definito $\mathbb{U}$ che contiene
tutte le possibili uscite dell'oservazione. Ciascuna di queste uscite � definito \emph{evento}.
Quindi
\begin{equation*}
\mathbb{E} \subseteq \mathbb{U}
\end{equation*}
e detto in altri termini, un evento � un insieme di possibilit�. Ad esempio
\begin{equation*}
\mathbb{E} = \{2,4,5\}
\end{equation*}
pu� essere un evento nel lancio di un dado.\\
$p(\mathbb{U}) = 1$ per qualsiasi tipo di osservazione. Quindi la probabilit� che \textbf{non} avvenga
un evento � $1-p(\mathbb{E})$

\subsubsection{Eventi incompatibili}
Due eventi si dicono incompatibili quando
\begin{equation*}
\mathbb{E}_1 \cap \mathbb{E}_2 = \emptyset
\end{equation*}

\subsubsection{Eventi indipendenti}
Due eventi si dicono indipendenti quando
\begin{equation*}
p\left(\mathbb{E}_1\mid\mathbb{E}_2\right) = p(\mathbb{E}_1)
\end{equation*}

\subsection{Probabilit� di eventi incompatibili}
\begin{equation*}
p(\mathbb{E}_1\cup\mathbb{E}_2) = p(\mathbb{E}_1) + p(\mathbb{E}_2)
\end{equation*}

\subsection{Probabilit� di eventi compatibili}
\begin{equation*}
p(\mathbb{E}_1\cup\mathbb{E}_2) = p(\mathbb{E}_1)+p(\mathbb{E}_2)-p(\mathbb{E}_1\cap\mathbb{E}_2)
\end{equation*}
Si estenda questa formula in modo che si tolgano tutte le intersezioni fra eventi per non ripetere 
risultati.

\subsection{Probabilit� condizionata}
La probabilit� condizionata indica la probabilit� che si verifichi l'evento $\mathbb{E}_1$ 
verificatosi $\mathbb{E}_2$.
\begin{equation*}
p\left(\mathbb{E}_1\mid\mathbb{E}_2\right) = 
\frac{p\left(\mathbb{E}_1\cap\mathbb{E}_2\right)}{p(\mathbb{E}_2)}
\end{equation*}

\subsection{Probabilit� composta}
Indica la probabilit� che si verifichi un evento intersezione di altri due.
\begin{equation*}
p\left(\mathbb{E}_1\cap\mathbb{E}_2\right) = p(\mathbb{E}_1)\cdot
p\left(\mathbb{E}_1\mid\mathbb{E}_2\right)
\end{equation*}

Per� se sono eventi indipendenti si semplifica in
\begin{equation*}
p\left(\mathbb{E}_1\cap\mathbb{E}_2\right) = p(\mathbb{E}_1)\cdot p(\mathbb{E}_2)
\end{equation*}

\subsubsection{Formule di Bayes}
\subsubsection{Prima formula}
Essendo $\mathbb{F}_1, \mathbb{F}_2,\dotsc,\mathbb{F}_n$ $n$ eventi incompatibili tali che
\begin{equation*}
\mathbb{U} = \mathbb{F}_1\cup\mathbb{F}_2\cup\dotsb\cup\mathbb{F}_n
\end{equation*}
si consideri un evento $\mathbb{E}$ tale che
\begin{equation*}
\mathbb{E} = \left(\mathbb{E}\cap\mathbb{F}_1\right)\cup\left(\mathbb{E}\cap\mathbb{F}_2\right)\cup
\dots\cup\left(\mathbb{E}\cap\mathbb{F}_n\right)
\end{equation*}
si ha
\begin{equation*}
p(\mathbb{E}) = \sum\limits_{i=1}^{n}p\left(\mathbb{E}\cap\mathbb{F}_i\right) =
\sum\limits_{i=1}^{n}\big(p\left(\mathbb{E}\mid\mathbb{F}_i\right)\cdot 
p\left(\mathbb{F}i\right)\big)
\end{equation*}

\subsubsection{Seconda formula}
Essendo $\mathbb{F}_1, \mathbb{F}_2,\dotsc,\mathbb{F}_n$ $n$ eventi incompatibili tali che
\begin{equation*}
\mathbb{U} = \mathbb{F}_1\cup\mathbb{F}_2\cup\dotsb\cup\mathbb{F}_n
\end{equation*}
sia $\mathbb{E}$ un evento tale che $p(\mathbb{E})>0$, per calcolare le probabilit� condizionali si
usi
\begin{equation*}
p\left(\mathbb{F}_i\mid\mathbb{E}\right) =
\frac{p\left(\mathbb{E}\mid\mathbb{F}_i\right)\cdot p(\mathbb{F}_i)}
{\sum\big(p\left(\mathbb{E}\mid\mathbb{F}_i\right)\cdot p(\mathbb{F}_i)\big)}
\end{equation*}
%!TEX ROOT=formularioMatematica.tex

\section{Affinità}\label{sec:aff}
Si definisce un'affinità come una corrispondenza biunivoca tra due piani e tra punti dello stesso 
piano che trasformi rette in rette conservando il parallelismo.\\
Un'affinità generica denominata $T$ può essere espressa nei seguenti modi
\begin{equation*}
T:\,\begin{cases}
x'=ax+by+e\\
y'=cx+dy+f
\end{cases}
\end{equation*}
\begin{equation*}
T:\,\begin{bmatrix}[1]
x'\\y'
\end{bmatrix}=
\begin{bmatrix}[1]
a&b\\
c&d
\end{bmatrix}
\begin{bmatrix}[1]
x\\y
\end{bmatrix}
+\begin{bmatrix}[1]
e\\f
\end{bmatrix}
\end{equation*}
\begin{equation*}
T:\,(x,y)\mapsto(ax+by+e,cx+dy+f)
\end{equation*}
Tutte le affinità hanno il determinante della matrice dei coefficienti è sempre diverso da zero
\begin{equation*}
\begin{vmatrix}[1]
a&b\\
c&d
\end{vmatrix} = ad-cb \neq0
\end{equation*}
Se il determinante è pari a $1$ è un'isometria e quindi mantiene le distanze.\\\\

Si definisce punto unito qualunque punto che si trasforma in sé stesso, ovvero
\begin{equation*}
T(U) \equiv U (\equiv U')
\end{equation*}

Essendo le affinità proprietà binuivoche, esiste anche la trasformazione inversa, generalmente 
indicata con $T^{-1}$.\\
Per gli esercizi si vada a pagina~\pageref{ex:aff}.

\subsection{Prodotto di trasformazioni}
Se si hanno due trasformazioni $T$ e $T'$, il loro prodotto è descritto come $T\ast T'$ e si ottiene
effettuando prima $T'$ e successivamente $T$. Quindi è equivalente a $T(T'(P))$.\\
La matrice dei coefficienti si ottiene moltiplicando le due matrici $A\ast A'$.

\subsection{Traslazione}
\begin{center}
	\begin{tikzpicture}
		\coordinate (A) at (1,2);
		\coordinate (B) at (4,3);
		
		\tkzInit[xmin=-1,ymin=-1,xmax=5,ymax=5]
		\tkzGrid
		\tkzAxeXY
		
		\filldraw (A) circle (0.05);
		\filldraw (B) circle (0.05);
		\draw[red, thick, -stealth] (A) -- (B)
			node[pos=0.5, fill=white, text=red]{$\vec{v}(a,b)$};
	\end{tikzpicture}
\end{center}
\begin{equation*}
\tau_{\begin{bmatrix}[0.7]
	\mathcolor{red}{a}\\\mathcolor{red}{b}
	\end{bmatrix}}:
\begin{cases}
x'= x+\mathcolor{red}{a}\\
y'= y+\mathcolor{red}{b}
\end{cases}
\end{equation*}
\begin{equation*}
\tau_{\begin{bmatrix}[0.7]
	\mathcolor{red}{a}\\\mathcolor{red}{b}
	\end{bmatrix}}^{-1}:
\begin{cases}
x = x'-\mathcolor{red}{a}\\
y = y'-\mathcolor{red}{b}
\end{cases}
\end{equation*}

\subsection{Rotazione}
\begin{center}
	\begin{tikzpicture}
		\coordinate (A) at (1,2);
		\coordinate (B) at (2.23,0);
		\coordinate (O) at (0,0);
		
		\tkzInit[xmin=-1,ymin=-1,xmax=5,ymax=5]
		\tkzGrid
		\tkzAxeXY
		
		\filldraw (A) circle (0.05);
		\filldraw (B) circle (0.05);
		\path[thick, red, -stealth, bend left] (A) edge (B);
		\draw (O) -- (A);
		\markangle{O}{A}{B}{0.5}{1.5}{$\theta$}
	\end{tikzpicture}
\end{center}
\begin{equation*}
\rho_{O,\theta}:\,\begin{cases}
x'=x\cos\theta-y\sin\theta\\
y'=x\sin\theta+y\cos\theta
\end{cases}
\end{equation*}
\begin{equation*}
\rho_{O,\theta}:\,\begin{cases}
x=x'\cos\theta+y'\sin\theta\\
y=-y'\sin\theta+y\cos\theta
\end{cases}
\end{equation*}

\subsection{Simmetria centrale}
\begin{center}
	\begin{tikzpicture}
		\coordinate (A) at (-1,0);
		\coordinate (B) at (2,0);
		\coordinate (C) at (0.5,0);
		
		\tkzInit[xmin=-1.5,ymin=-1,xmax=2.5,ymax=1]
		\tkzGrid
		\tkzAxeXY
		
		\filldraw (A) circle (0.05);
		\filldraw (B) circle (0.05);
		\filldraw (C) circle (0.05);
		\draw (A) -- (B);
		\path[thick, red, bend left, -stealth] (A) edge (B);
	\end{tikzpicture}
\end{center}

\begin{equation*}
\sigma_{C(x_C,y_C)}:\,\begin{cases}
x'= -x+2x_C\\
y'= -y+2y_C
\end{cases}
\end{equation*}
\begin{equation*}
\sigma_{C(x_C,y_C)}^{-1}:\,\begin{cases}
x = -x'+2x_C\\
y = -y'+2y_C
\end{cases}
\end{equation*}

\subsection{Simmetria assiale}
\begin{center}
	\begin{tikzpicture}
		\coordinate (A) at (-2,0);
		\coordinate (B) at (-1,1);
		\coordinate (C) at (1,1);
		\coordinate (D) at (2,0);
		
		\tkzInit[xmin=-3,ymin=-1,xmax=3,ymax=2]
		\tkzGrid
		\tkzAxeXY
		
		\filldraw (A) circle (0.05);
		\filldraw (B) circle (0.05);
		\filldraw (C) circle (0.05);
		\filldraw (D) circle (0.05);
		\draw (A) -- (B);
		\draw (C) -- (D);
		\draw[-stealth, red, thick] (-1.5,0.5) -- (1.5,.5);
	\end{tikzpicture}
\end{center}

\subsubsection{Rispetto a $r:\,y=y_0$}
\begin{equation*}
\sigma_r:\,\begin{cases}
x'= x\\
y'= -y+2y_0
\end{cases}
\end{equation*}

\subsubsection{Rispetto a $r:\,x=x_0$}
\begin{equation*}
\sigma_r:\,\begin{cases}
x'= -x+2x_0\\
y'=y
\end{cases}
\end{equation*}

\subsubsection{Rispetto a $r:\,y=mx+q$}
\begin{equation*}
\sigma_r:\,\begin{cases}
x'=\frac{1}{1+m^2}[(1-m^2)x+2my-2mq]\\
y'=\frac{1}{1+m^2}[2mx+(m^2-1)y+2q]
\end{cases}
\end{equation*}

\begin{equation*}
\sigma_r^{-1}:\,\begin{dcases}
x=\frac{1}{1+m^2}[(1-m^2)x +2my'-2mq]\\
y=\frac{1}{1+m^2}[2mx'+(m^2-1)y'+2q]
\end{dcases}
\end{equation*}

\subsection{Similitudine}
\begin{center}
	\begin{tikzpicture}[scale=0.75]
		\coordinate (A) at (-2,0);
		\coordinate (B) at (0,4);
		\coordinate (O) at (0,0);
		\coordinate (C) at (-4,6);
		\coordinate (D) at (-2,-5);
		\coordinate (E) at (2,-2);
		
		\tkzInit[xmin=-5,ymin=-6,xmax=3,ymax=7]
		\tkzGrid
		\tkzAxeXY
		
		\filldraw (A) circle (0.05);
		\filldraw (B) circle (0.05);
		\filldraw (C) circle (0.05);
		\filldraw (D) circle (0.05);
		\filldraw (E) circle (0.05);
		\filldraw (O) circle (0.05);
		
		\filldraw[thick, cyan, fill opacity = 0.3] (A) -- (B) -- (O) -- cycle;
		\filldraw[thick, red, fill opacity = 0.3] (C) -- (D) -- (E) -- cycle;
		\path[-stealth, red, thick, bend left] (-0.5,2) edge (0.7,-1);
	\end{tikzpicture}
\end{center}

\begin{equation*}
\Sigma:\,
\begin{cases}
\begin{cases}
x'= ax-by+e\\
y'= bx+ay+f
\end{cases}\text{Se diretta, }\det A = \begin{vmatrix}[1]
a&-b\\b&-a
\end{vmatrix}>0\\
\begin{cases}
x'=ax+by+e\\
y'=bx-ay+f
\end{cases}\text{Se indiretta, }\det A = \begin{vmatrix}[1]
a&b\\b&-a
\end{vmatrix}<0
\end{cases}
\end{equation*}
Il rapporto di similitudine è pari a 
\begin{equation*}
k = \sqrt{a^2+b^2}
\end{equation*}

\subsection{Omotetia}
\begin{center}
	\begin{tikzpicture}[scale=0.5]
		\coordinate (A) at (-2,0);
		\coordinate (B) at (-1,1);
		\coordinate (O) at (0,0);
		\coordinate (C) at (-3,-2);
		\coordinate (D) at (4,0);
		\coordinate (E) at (2,-2);
		\coordinate (F) at (6,4);
		
		\tkzInit[xmin=-4,ymin=-3,xmax=7,ymax=5]
		\tkzGrid
		\tkzAxeXY
		
		\filldraw (A) circle (0.05);
		\filldraw (B) circle (0.05);
		\filldraw (C) circle (0.05);
		\filldraw (D) circle (0.05);
		\filldraw (E) circle (0.05);
		\filldraw (F) circle (0.05);
		\filldraw (O) circle (0.05);
		
		\filldraw[thick, cyan, fill opacity = 0.3] (A) -- (B) -- (C) -- cycle;
		\filldraw[thick, red, fill opacity = 0.3] (D) -- (E) -- (F) -- cycle;
		\path[-stealth, red, thick, bend left] (-1,-0.5) edge (3.5,1);
		\draw[dashed] (A) -- (D);
		\draw[dashed] (B) -- (E);
		\draw[dashed] (C) -- (F);
	\end{tikzpicture}
\end{center}

\begin{equation*}
\omega_{C,a}:\,\begin{cases}
x'=a(x-x_C)+x_C\\
y'=a(y-y_C)+y_C
\end{cases} \rightarrow
\begin{cases}
x'= ax+h\\
y'= ay+k
\end{cases}
\end{equation*}

\subsection{Dilatazione}
\begin{center}
	\begin{tikzpicture}
		\tkzInit[xmin=-1,ymin=-1,xmax=4,ymax=3]
		\tkzGrid
		\tkzAxeXY
		\filldraw[thick, cyan, fill opacity = 0.3] (0,0) -- (2,0) -- (2,2) -- (0,2) -- cycle;
		\filldraw[thick, red, fill opacity = 0.3] (2,2) -- (3,2) -- (3,0) -- (2,0);
		\path[-stealth, thick, red] (1,1) edge (2.5,1);
	\end{tikzpicture}
\end{center}

\begin{equation*}
\delta_{x,k}:\,\begin{cases}
x'=kx\\
y'=y
\end{cases}\quad\delta_{y,k}:\,\begin{cases}
x'=x\\
y'=ky
\end{cases}
\end{equation*}

\subsection{Inclinazione}
\begin{center}
	\begin{tikzpicture}
		\tkzInit[xmin=-1,ymin=-1,xmax=6,ymax=3]
		\tkzGrid
		\tkzAxeXY
		\filldraw[thick, cyan, fill opacity = 0.3] (0,0) -- (2,0) -- (2,2) -- (0,2) -- cycle;
		\filldraw[thick, red, fill opacity = 0.3] (0,0) -- (2,0) -- (5,2) -- (3,2) -- cycle;
		\path[-stealth, thick, red] (1,1) edge (2.5,1);
	\end{tikzpicture}
\end{center}
\begin{equation*}
\xi_{x,k}:\,\begin{cases}
x'=x_ky\\
y'=y
\end{cases}\quad\xi_{y,k}:\,\begin{cases}
x'=x\\
y'=y+kx
\end{cases}
\end{equation*}

%!TEX ROOT=formularioMatematica.tex

\section{Numeri complessi}\label{sec:complex}
Fino ad adesso abbiamo sempre lavorato con numeri appartenenti ad $\mathbb{R}$ al di pi�. Ci sono per�
alcune operazioni che non sono possibili da fare in questo insieme numerico. Una di queste �
\begin{equation*}
\sqrt{-r}\qquad\forall r\in\mathbb{R}^+
\end{equation*}
oppure
\begin{equation*}
\log(n)\qquad \forall n\in\mathbb{R}^-
\end{equation*}
Per sopperire a questa mancanza, � stata introdotta l'unit� immaginaria $i$ che � definita come
\begin{equation*}
i=\sqrt{-1}
\end{equation*}
Un numero complesso � un numero composto da una parte reale e una immaginaria. Esso pu� essere scritto
come
\begin{equation*}
z = \overbrace{a}^{\text{Reale}} + \overbrace{ib}^{\text{Immaginaria}}
\end{equation*}
Quindi
\begin{equation*}
\Re(z) = a \quad\text{e}\quad\Im(z)=ib
\end{equation*}
Questo non � l'unico modo di identificare un numero complesso. Pi� avanti vedremo anche gli altri.\\
$\mathbb{C}$ � quindi definito come
\begin{equation*}
\mathbb{C}=\left\{a+ib: a,b\in\mathbb{R}\right\}
\end{equation*}
Il numero complesso $\bar{z}$ � definito il \textit{coniugato} di $z$ quindi
\begin{equation*}
z=a+ib\qquad\text{e}\qquad\bar{z}=a-ib
\end{equation*}
Si noti che se si deve fare ricorso a \hyperref[ruffini]{Ruffini} si ricerchi lo zero anche in 
$\mathbb{C}$.\\
Per gli esercizi si vada \hyperref[ex:complex]{qui}.

\subsection{Rappresentazione cartesiana}\label{subsec:complex:cart}
Essendo questo numero composto da parte reale e immaginaria, il piano cartesiano non basta pi�. Quindi 
si � deciso di estenderlo a quello che viene comunemente denominato il piano di \textbf{Argrand-Gauss}.
Esso � composto dall'asse delle ascisse come parte reale e quello delle ordinate come parte 
immaginaria.

\begin{center}
	\begin{tikzpicture}
		\coordinate (A) at (1,1.5);
		\coordinate (O) at (0,0);
		\coordinate (B) at (2,0);
		
		\tkzInit[xmin=-2,ymin=-2,xmax=2,ymax=2]
		\tkzGrid
		\tkzAxeXY
		\filldraw[red] (A) circle (0.05);
		\draw[-stealth] (O) -- (A);
		\markangle{O}{A}{B}{0.5}{1.5}{$\theta$}
		\node[above right] at (A) {$P(a,b)$};
	\end{tikzpicture}
\end{center}

La distanza $\overline{OP}$ � detta \emph{modulo} del numero immaginario ed � descritto come
\begin{equation*}
\rho=\norm{z} = \sqrt{a^2+b^2}
\end{equation*}

\subsection{Operazioni tra numeri complessi}
\subsubsection{Somma}
La somma tra due numeri complessi richiede solo di sommare le parti simili fra di loro.
\begin{align*}
&z_1=a_1+i+b_1\qquad\text{e}\qquad z_2=a_2+i_b2
\intertext{La loro somma �}
&z=z_1+z_2=a_1+a_2+i(b_1+b_2)
\end{align*}
Nella somma vige la seguente caratteristica
\begin{equation*}
\overline{z_1+z_2} = \overline{z_1}+\overline{z_2}
\end{equation*}

\subsubsection{Differenza}
La differenza � esattamente come la somma, ovvero si opera parte a parte.
\begin{align*}
&z_1=a_1+i+b_1\qquad\text{e}\qquad z_2=a_2+i_b2
\intertext{La loro differenza �}
&z=z_1-z_2=a_1-a_2+i(b_1-b_2)
\end{align*}
Nella somma e differenza vigono le seguenti caratteristiche
\begin{equation*}
z+\overline{z}=2a\qquad\text{e}\qquad z-\overline{z}=2ib
\end{equation*}

\subsubsection{Prodotto}
Il prodotto si effettua moltiplicando fra di loro parti simili.
\begin{align*}
&z_1=a_1+i+b_1\qquad\text{e}\qquad z_2=a_2+i_b2
\intertext{Il loro prodotto �}
&z=z_1\cdot z_2=a_1a_2-b_1b_2+i(b_1a_2+a_1b_2)
\end{align*}	

\subsubsection{Quoziente}
Il quoziente � un'operazione particolare.
\begin{align*}
&z_1=a_1+i+b_1\qquad\text{e}\qquad z_2=a_2+i_b2
\intertext{Il loro quoziente �}
&z=\frac{z_1}{z_2}=\frac{a_1a_2+b_1b_2}{a_2^2+b_2^2}+i\frac{b_1a_2-a_2b_2}{a_2^2+b_2^2}
\end{align*}

\subsection{Rappresentazione trigonometrica di un numero complesso}
Un modo per definire un numero complesso � gi� stato chiarito. Ne esiste un altro per� che fa capo
alla rappresentazione polare del numero (tramite un altro sistema di assi che identifica un punto
tramite l'angolo che compie un segmento dall'asse $x$).
\begin{equation*}
z=\rho(\cos\theta+i\sin\theta)
\end{equation*}
dove $\theta$ rappresenta l'angolo indicato nella sottosezione
\hyperref[subsec:complex:cart]{Rappresentazione cartesiana}.

\subsubsection{Prodotto}
\begin{align*}
&z_1=\rho_1(\cos\theta_1+i\sin\theta_1)\quad\text{e}\quad z_2=\rho_2(\cos\theta_2+i\sin\theta_2)
\intertext{Il loro prodotto �}
&z=z_1\cdot z_2=\rho_1\rho_2[\cos(\theta_1+\theta_2)_i\sin(\theta_1+\theta_2)]
\end{align*}
\subsubsection{Quoziente}
\begin{align*}
&z_1=\rho_1(\cos\theta_1+i\sin\theta_1)\quad\text{e}\quad z_2=\rho_2(\cos\theta_2+i\sin\theta_2)
\intertext{Il loro quoziente �}
&z=\frac{z_1}{z_2}=\frac{\rho_1}{\rho_2}[\cos(\theta_1-\theta_2)_i\sin(\theta_1-\theta_2)]
\end{align*}

\subsection{Elevazione a potenza}
Messa successivamente alle altre operazioni perch� varia in base alla notazione scelta.
\subsubsection{Algebrica}
\begin{equation*}
z^n = (a+ib)^n
\end{equation*}
\subsubsection{Trigonometrica (Formula di De Moivre)}
\begin{equation*}
z^n=\rho^n(\cos n\theta+i\sin n\theta)
\end{equation*}

\subsection{Radici di un numero complesso}
\begin{align*}
&z = \rho(\cos\theta+i\sin\theta)
\intertext{La radice $n$-esima � pari a}
&\sqrt[n]{z}=\sqrt[n]{\rho}\left(\cos\frac{\theta+2k\pi}{n}+i\sin\frac{\theta+2k\pi}{n}\right)
\end{align*}
Una caratteristica interessante delle radici � la loro rappresentazione grafica. Infatti se si prendono
le coordinate e si uniscono fra di loro si costruir� un poligono regolare con $n$ lati inscritto 
all'interno di una circonferenza di raggio $\rho$.\\
Per calcolarle, sostituire $k$ con i numeri che vanno da $0$ a $n-1$.

\subsection{Teroema fondamentale dell'algebra}
Esso cita:
\begin{tfa}
	Ogni polinomio di grado $n\geq1$
	\begin{equation*}
	P(x) = \sum\limits_{i=0}^{n} a_ix^i
	\end{equation*}
	a coefficienti $\mathbb{C}$ ha almeno uno zero in $\mathbb{C}$.
\end{tfa}
Da questo deriva
\begin{tfa-ext}\hypertarget{teor:tfa-ext}{}
	Per ogni polinomio di grado $n\geq1$
	\begin{equation*}
	P(x) = \sum\limits_{i=0}^{n} a_ix^i
	\end{equation*}
	a coefficienti $\mathbb{C}$ ha esattamente $n$ zeri in $\mathbb{C}$, con la convenzione di contare
	$r$ volte uno zero di molteplicit� $r$.
\end{tfa-ext}
Per molteplicit� si intende
\begin{molteplic}
	Se il polinomio $P(x)$ si pu� scomporre nel prodotto
	\begin{equation*}
	P(x) = (x-\alpha)^rP_r(x)
	\end{equation*}
	dove il polinomio $P_r(x)$ � di grado $(n-r)$, si dice che $P(x)$ � divisibile per $(x-\alpha)^r$
	e se $P_r(x)$ non � divisibile per $(x-\alpha)$, si dice che $\alpha$ � uno \textbf{zero di 
	molteplicit�} $r$ per $P(x)$.
\end{molteplic}

\subsection{Esponenziali complessi}
Detta $e$ la costante di Nepero (anche chiamato numero di Eulero)
\begin{equation*}
e \approx 2.71828\ldots
\end{equation*}
si definisce per ogni numero complesso $z = a+ib$ l'esponenziale complesso $e^{a+ib}$ come il numero
complesso 
\begin{equation*}
w=e^x(\cos a+i\sin b)
\end{equation*}
Quindi possiamo dire che 
\begin{equation*}
e^{z} = \rho e^{i\theta}
\end{equation*}
Proprio da questa formula ne viene ricavata una delle pi� famose della storia della matematica
\begin{equation*}
e^{i\pi}+1=0
\end{equation*}
che collega 5 unit� fondamentali della matematica.

\subsection{Formule di Eulero}
\begin{equation*}
\cos\theta=\frac{e^{i\theta}+e^{-i\theta}}{2}\qquad\sin\theta=\frac{e^{i\theta}-e^{-i\theta}}{2}
\end{equation*}
Queste formule permettono di trasferire tutte le caratteristiche della notazione trigonometrica in 
quella esponenziale.
%!TEX ROOT=formularioMatematica.tex

\section{Insiemi numerici}\label{sec:insiemi}
Se durante la nostra carriera scolastica abbiamo sempre lavorato con $\mathbb{R}$ o al massimo in 
$\mathbb{C}$, nulla vieta che noi creiamo nuovi insiemi numerici e li studiamo per trovarne alcune
caratteristiche.\\
Ogni insieme numerico possiede delle caratteristiche che noi possiamo studiare
\begin{enumerate}
	\item � limitato/illimitato
	\item Possiede un $\max$ e un $\min$
	\item Possiede \emph{maggioranti} o \emph{minoranti}
	\item Possiede un $\sup$ o un $\inf$
\end{enumerate}
Per le seguenti definizioni ed esempi, prenderemo in considerazione
\begin{equation*}
A = \left\{\frac{1}{n}\mid n\in\mathbb{N}_0\right\}
\end{equation*}

\subsection{Condizioni di limitazione}
La o le condizioni di limitazione indicani quale pu� essere un limite o i limiti di un insieme. In 
insieme pu� essere illimitato (ovvero che per qualunque numero noi scegliamo, esister� un punto
sulla retta che lo rappresenta), limitato \emph{superiormente}, \emph{inferiormente} o entrambi
contemporaneamente.\\
In generale la condizione di limitazione (superiore ed inferiore per uno stesso valore) �
\begin{equation*}
\exists\,k>0\land k\in\mathbb{R}\mid \forall x \in A, \abs{x} \leq k
\end{equation*}
Generalizzando ancora per due valori diversi
\begin{equation*}
\exists\,k_1,k_2>0\land k_1,k_2\in\mathbb{R}\mid \forall x\in A, k_1\leq x \leq k_2
\end{equation*}

\subsection{Maggioranti e minoranti}
Prendendo le condizioni di limitazione separatamente
\begin{align*}
\exists\,m\in\mathbb{R}\mid\forall x\in A x\geq m
\intertext{e}
\exists\,M\in\mathbb{R}\mid\forall x\in A x\leq M
\end{align*}
$m$ rappresenta un \textbf{minorante} di $A$ e $M$ rappresenta un \textbf{maggiorante} di $A$.

\subsection{Massimi e minimi}
Un numero si definisce \emph{massimo} se
\begin{equation*}
\exists\,L\in\mathbb{A}\mid\forall x\in A, L\geq x
\end{equation*}
quindi � il valore pi� alto che l'insieme contiene.\\
Un numero si definisce \emph{minimo} se
\begin{equation*}
\exists\,l\in\mathbb{A}\mid\forall x\in A, l\leq x
\end{equation*}
quindi � il valore pi� basso che l'insieme contiene.

\subsection{Intervalli}
Un \emph{intervallo} pu� essere aperto (illimitato) o chiuso (limitato). La notazione pi� comune � la 
seguente
\begin{align*}
\textbf{Limitato }&{[{1},{4}]}\coloneqq\{x\in\mathbb{R}\mid1\leq x \leq4\}\\
\textbf{Illimitato }&{]{-\infty},{\pi}[}\coloneq\{x\in\mathbb{R}\mid-\infty<x<\pi\}
\end{align*}
Da notare che $\pm\infty$ � sempre escluso in quanto tecnicamente non appartiene a $\mathbb{R}$.\\
Un \emph{intorno} non � altro che un intervallo che comprende un numero specifico. In simboli
\begin{equation*}
I(x_0)\coloneqq{]{x_0-\delta_1},{x_0+\delta_2}[} \qquad(\delta_1,\delta_2\in\mathbb{R})
\end{equation*}
Ovviamente si pu� definire un intorno che sia limitato con una distanza
\begin{equation*}
I_\varepsilon(x_0)\coloneqq{]{x_0-\varepsilon},{x_0+\varepsilon}[} 
\end{equation*}
Questo � completo e ovviamente possiamo anche fare intorni non completi (quindi solo da un lato). Essi 
sono di conseguenza denominati sinistri o destri.

\subsection{Punti isolati}
Un punto isolato � un punto i cui intorni non contengono alcun elemento dell'insieme.
\begin{equation*}
x_0\in A, \exists\,I(x_0)\mid\forall x\in A\setminus\{x_0\}\not\supset I(x_0)
\end{equation*}

\subsection{Punti di accumulazione}
Un punto di accumulazione � un punto in cui ogni suo intorno cade almeno un elemento distinto 
dell'insieme.
\begin{equation*}
x_0,y\in A,\forall I(x_0), y\in I(x_0)
\end{equation*}

\subsection{Estremi}
L'estremo superiore � quel valore che non viene mai superato. A seconda dei casi pu� essere il 
\emph{pi� grande elemento dell'insieme} o il \emph{pi� piccolo dei maggioranti}.
\begin{equation*}
\forall x\in A\implies x\leq\Lambda\quad\text{e}\quad \forall\varepsilon\in\mathbb{R}_0^+,
\exists\,x\in A\mid x>\Lambda-\varepsilon
\end{equation*}
L'estremo inferiore � quel valore che non non ha valori inferiori. A seconda dei casi pu� essere il
\emph{pi� piccolo elemento dell'insieme} o il \emph{pi� grande dei minoranti}.
\begin{equation*}
\forall x\in A\implies x\geq\lambda\quad\text{e}\quad \forall\varepsilon\in\mathbb{R}_0^+,
\exists\,x\in A\mid x<\lambda+\varepsilon
\end{equation*}
%!TEX ROOT=formularioMatematica.tex

\section{Limiti}\label{sec:limiti}
Per introdurre il concetto di limite, prendiamo ad esempio la funzione
\begin{equation*}
f:\,\mathscr{D}_f\mapsto\mathbb{R}\mid x\mapsto\frac{2x^2-8}{x-2}
\end{equation*}
essendo
\begin{equation*}
\mathscr{D}_f=\mathbb{R}-\{2\}
\end{equation*}
La funzione non � definita per $x = 2$ per� possiamo comunque trovare i valori della funzione per numeri
che si avvicinano sempre pi� a $2$
\begin{center}
	\begin{tabular}{cc}
		$\boldsymbol{x}$ & $\boldsymbol{f(x)}$\\\hline
		$1$ & $6$\\
		$1.5$ & $7$\\
		$1.9$ & $7.8$\\
		$1.9991$ & $7.9982$\\
		$\ldots$ & $\ldots$\\
		$2.0001$ & $8.002$\\
		$2.1$ & $8.2$\\
		$2.5$ & $9$\\
		$3$ & $10$ 
	\end{tabular}
\end{center}
Come notiamo dalla tabella, pi� i ci si avvicina a $2$ pi� i valori si avvicinano a $8$. A questo
comportamento diamo il nome di \textbf{limite finito}.\\
Possiamo quindi dire che per un numero $\varepsilon$ positivo
\begin{equation*}
\left\lvert\frac{2x^2-8}{x-2}-8\right\rvert<\varepsilon
\end{equation*}
Questa disequazione ammette come soluzioni un intervalo opportuno di centro $x=2$. Tenuto conto che
\begin{align*}
&2x^2-8=2(x-2)(x+2)
\intertext{riducendo}
&\abs{2x-4}<\varepsilon\quad x \neq 2
\intertext{ovvero}
&-\varepsilon<2x-4<\varepsilon\quad x \neq 2
\intertext{ossia}
&2-\frac{\varepsilon}{2}<x<2+\frac{\varepsilon}{2}\quad x \neq 2
\intertext{Questo vuol dire che se}
&x\in{\left]{2-\frac{\varepsilon}{2}},{2+\frac{\varepsilon}{2}}\right[}\quad x \neq 2
\intertext{i corrispondenti valori di $f(x)$ distano da $8$ meno di $\varepsilon$}
\end{align*}
Scriveremo allora
\begin{equation*}
\lim\limits_{x\to2}\frac{2x^2-8}{x-2}=8
\end{equation*}
che si legge \emph{il limite per $x$ che tende a $2$ di $\frac{2x^2-8}{x-2}$ � uguale a $8$}.\\\\
Consideriamo ora la funzione
\begin{equation*}
f:\,\mathscr{D}_f\mapsto\mathbb{R}\mid x\mapsto\frac{1}{(x+1)^2}
\end{equation*}
essendo
\begin{equation*}
\mathscr{D}_f = \mathbb{R}-\{-1\}
\end{equation*}
Attribuiamo ora a $x$ valori sempre pi� vicini a $-1$
\begin{center}
	\begin{tabular}{cc}
		$\boldsymbol{x}$ & $\boldsymbol{f(x)}$\\\hline
		$-2$ & $1$\\
		$-1.5$ & $4$\\
		$-1.001$ & $1,000,000$\\
		$\ldots$ & $\ldots$\\
		$0$ & $1$\\
		$-0.5$ & $4$\\
		$-0.99995$ & $400,000,000$
	\end{tabular}
\end{center}
Notiamo che per valori che si avvicinano a $-1$ otteniamo sempre valori molto grandi. A questo 
comportamento si da il nome di \textbf{limite a pi� infinito ($+\infty$)}.\\
Quindi si pu� scrivere
\begin{equation*}
\lim\limits_{x\to-1}\frac{1}{(x+1)^2}=+\infty
\end{equation*}
che sgnifica che comunque si prenda un numero $M$ la disuguaglianza
\begin{equation*}
\frac{1}{(x+1)^2}>M
\end{equation*}
� soddisfatta dai punti di un intorno di $-1$, escluso $-1$ stesso.\\
Supposto che $x\neq-1$ l'equazione equivale a
\begin{align*}
&(x+1)^2<\frac{1}{M}
\intertext{verificata per}
&-\frac{1}{\sqrt{M}}<x+1<\frac{1}{\sqrt{M}}
\intertext{cio�}
&-1-\frac{1}{\sqrt{M}}<x<-1+\frac{1}{\sqrt{M}}\quad x\neq-1
\end{align*}
Tali valori effettivamente rappresentano un intorno di $-1$ escluso $-1$ stesso.\\
Analogamente al limite che tende a $+\infty$, si pu� trovare il limite a $-\infty$.\\\\
Un problema simile a quelli precedenti � quello di un valore che dopo un po' si stabilizza. In altre
parole, un valore che tendendo ad $\infty$ tende ad un numero finito. Questi sono definiti \textbf{
limiti finiti di una funzione all'infinito}. In simboli
\begin{equation*}
\lim\limits_{x\to+\infty}f(x)=l
\end{equation*}
Possiamo ulteriormente estendere il concetto a \textbf{limiti infiniti di una funzione all'infinito}.
Ad esempio
\begin{equation*}
\lim\limits_{x\to-\infty}f(x)=+\infty
\end{equation*}


\subsection{Definizione di limite finito}
\begin{definizioneLimiteFinito}
	Sia $f$ una funzione definita in un intorno $I$ del punto $x_0$, senza che sia necessariamente
	definita in $x_0$.\\
	Si dice che il numero $l$ � il \textbf{limite} della funzione $f$ nel punto $x_0$ e si scrive
	\begin{equation*}
	\lim\limits_{x\to x_0}f(x)=l
	\end{equation*}
	se, fissato comunque un numero $\varepsilon>0$, � possibile determinare in corrispondenza di esso 
	un numero $\delta_\varepsilon>0$ tale che, per ogni $x$ appartenente a $I$ verificante la 
	condizione
	\begin{equation*}
	0<\abs{x-x_0}<\delta_\varepsilon
	\end{equation*}
	risulti
	\begin{equation*}
	\abs{f(x)-l}<\varepsilon
	\end{equation*}
	In simboli
	\begin{align*}
	\lim\limits_{x\to x_0} f(x) &= l\\
	\ArrowBetweenLines
	\forall \varepsilon, \exists\,\delta_\varepsilon>0\mid\forall x: 0<\vert x-x_0\vert 
	&<\delta_\varepsilon \Rightarrow \vert f(x) - l\vert < \varepsilon
	\end{align*}
\end{definizioneLimiteFinito}

\subsection{Definizione di limite infinito}
\subsubsection{A $+\infty$}
\begin{definizioneLimiteInfinito1}
	Sia $f$ una funzione definita in un intorno $I$ di $x_0$, escluso al pi� il punto $x_0$. Si dice
	che
	\begin{equation*}
	\lim\limits_{x\to x_0}f(x)=+\infty
	\end{equation*}
	se, fissato comunque un numero $M$, � possibile determinare in corrispondenza di esso un numero
	$\delta_M>0$ tale che, per ogni $x$ di $I$ verificante la condizione
	\begin{equation*}
	0<\abs{x-x_0}<\delta_M
	\end{equation*}
	risulti
	\begin{equation*}
	f(x)>M
	\end{equation*}
	In simboli
	\begin{align*}
	\lim\limits_{x\to x_0} f(x) &= l\\
	\ArrowBetweenLines
	\forall M>0, \exists\,\delta_M>0\mid\forall x: 0<\vert x-&x_0\vert < \delta_M \Rightarrow
	f(x) > M
	\end{align*}
\end{definizioneLimiteInfinito1}
\subsubsection{A $-\infty$}
\begin{definizioneLimiteInfinito2}
	Sia $f$ una funzione definita in un intorno $I$ di $x_0$, escluso al pi� il punto $x_0$. Si dice
	che
	\begin{equation*}
	\lim\limits_{x\to x_0}f(x)=-\infty
	\end{equation*}
	se, fissato comunque un numero $M$, � possibile determinare in corrispondenza di esso un numero
	$\delta_M>0$ tale che, per ogni $x$ di $I$ verificante la condizione
	\begin{equation*}
	0<\abs{x-x_0}<\delta_M
	\end{equation*}
	risulti
	\begin{equation*}
	f(x)<M
	\end{equation*}
	In simboli
	\begin{align*}
	\lim\limits_{x\to x_0} f(x) &= l\\
	\ArrowBetweenLines
	\forall M, \exists\,\delta_M>0\mid\forall x: 0<\vert x-x_0\vert &< \delta_M \Rightarrow
	f(x) < M
		\end{align*}
\end{definizioneLimiteInfinito2}

\subsection{Definizione di limite finito di una funzione all'infinito}
\subsubsection{Per $x\to+\infty$}
\begin{definizioneLimiteInfinitoFinito1}
	Sia $f$ una funzione definita in un insieme $\mathscr{D}_f$ illimitato superiormente.\\
	Si dice che
	\begin{equation*}
	\lim\limits_{x\to+\infty}f(x)=l
	\end{equation*}
	se, fissato comunque un numero $\varepsilon>0$ � possibile determinare in corrispondenza di esso 
	un numero $k_\varepsilon$ tale che, per ogni $x\in\mathscr{D}_f$ e maggiore di $k_\varepsilon$, 
	risulti
	\begin{equation*}
	\abs{f(x)-l}<\varepsilon
	\end{equation*}
	In simboli
	\begin{align*}
	\lim\limits_{x\to x_0} f(x) &= l\\
	\ArrowBetweenLines
	\forall \varepsilon, \exists\,k_\varepsilon>0\mid\forall x: x>&k_\varepsilon \Rightarrow
	\vert f(x) - l\vert < \varepsilon
	\end{align*}
\end{definizioneLimiteInfinitoFinito1}
\subsubsection{Per $x\to-\infty$}
\begin{definizioneLimiteInfinitoFinito2}
	Sia $f$ una funzione definita in un insieme $\mathscr{D}_f$ illimitato superiormente.\\
	Si dice che
	\begin{equation*}
	\lim\limits_{x\to-\infty}f(x)=l
	\end{equation*}
	se, fissato comunque un numero $\varepsilon>0$ � possibile determinare in corrispondenza di esso 
	un numero $k_\varepsilon$ tale che, per ogni $x\in\mathscr{D}_f$ e minore di $k_\varepsilon$, 
	risulti
	\begin{equation*}
	\abs{f(x)-l}<\varepsilon
	\end{equation*}
	In simboli
	\begin{align*}
	\lim\limits_{x\to x_0} f(x) &= l\\
	\ArrowBetweenLines
	\forall \varepsilon, \exists\,k_\varepsilon>0\mid\forall x: x<&k_\varepsilon \Rightarrow
	\vert f(x) - l\vert < \varepsilon
	\end{align*}
\end{definizioneLimiteInfinitoFinito2}

\subsection{Definizione di limte infinito di una funzione all'infinito}
\subsubsection{A $+\infty$}
\begin{definizioneLimiteInfinitoInfinito1}
	Sia $f$ una funzione definita in un insieme $\mathscr{D}_f$ illimitato superiormente 
	[inferiormente]. Si dice che
	\begin{equation*}
	\lim\limits_{\substack{x\to+\infty\\ [x\to-\infty]}}f(x)=+\infty
	\end{equation*}
	se, fissato comunque un numero $M$, � possibile determinare in corrispondenza di esso un numero 
	$k_M$ tale che, per ogni $x\in\mathscr{D}_f$ che verifichi la condizione $x>k_M\,[x<k_M]$, risulti
	\begin{equation*}
	f(x)>M
	\end{equation*}
	In simboli
	\begin{align*}
	\lim\limits_{\substack{x\to+\infty\\ [x\to-\infty]}}f(x)&=+\infty\\
	\ArrowBetweenLines
	\forall k_M>0, \exists\,M>0\mid\forall x: x>&k_M[<k_M] \Rightarrow f(x) > M
	\end{align*}
\end{definizioneLimiteInfinitoInfinito1}
\subsubsection{A $-\infty$}
\begin{definizioneLimiteInfinitoInfinito2}
	Sia $f$ una funzione definita in un insieme $\mathscr{D}_f$ illimitato superiormente 
	[inferiormente]. Si dice che
	\begin{equation*}
	\lim\limits_{\substack{x\to+\infty\\ [x\to-\infty]}} f(x)=-\infty
	\end{equation*}
	se, fissato comunque un numero $M$, � possibile determinare in corrispondenza di esso un numero 
	$k_M$ tale che, per ogni $x\in\mathscr{D}_f$ che verifichi la condizione $x>k_M\,[x<k_M]$, risulti
	\begin{equation*}
	f(x)<M
	\end{equation*}
	In simboli
	\begin{align*}
	\lim\limits_{\substack{x\to+\infty\\ [x\to-\infty]}}f(x)&=+\infty\\
	\ArrowBetweenLines
	\forall k_M, \exists\,M>0\mid\forall x: x>k_M&[<k_M] \Rightarrow f(x) < M
	\end{align*}
\end{definizioneLimiteInfinitoInfinito2}

\subsection{Limite sinistro e destro}
Avere limite $l$ in un punto $x_0$ significa per una funzione essere regolare, ovvero assumere valori
sempre pi� prossimi a $l$ tanto $x$ � prossimo a $x_0$.\\
Questa regolarit� per� pu� mancare in senso assoluto. Ci� avviene quando la funzione si stabilizza
su due numeri diversi a seconda che ci si avvicini da destra o da sinistra.
\begin{equation*}
\lim\limits_{x\to x_0^+}f(x)
\end{equation*}
indica un limite destro,
\begin{equation*}
\lim\limits_{x\to x_0^-}f(x)
\end{equation*}
un limite sinistro.
\begin{definizioneLimiteFinitoDestro}
	Sia $f$ una funzione definita in un intorno destro $I^+(x_0)$ di $x_0$, privato al pi� del punto
	$x_0$.\\
	Si dice che
	\begin{equation*}
	\lim\limits_{x\to x_0^+}f(x) = l
	\end{equation*}
	se, fissato comunque un numero $\varepsilon>0$, � possibile determinare in corrispondenza di esso
	un numero $\delta_\varepsilon>0$ tale che, per ogni $x\in I^+(x_0)$ verificante la condizione
	\begin{equation*}
	0<x-x_0<\delta_\varepsilon
	\end{equation*}
	risulti
	\begin{equation*}
	\abs{f(x)-l}<\varepsilon
	\end{equation*}
	In simboli
	\begin{align*}
	\lim\limits_{x\to x_0^+} f(x) &= l\\
	\ArrowBetweenLines
	\forall \varepsilon, \exists\,\delta_\varepsilon>0\mid\forall x: x_0< x <x_0+&\delta_\varepsilon 
	\Rightarrow \vert f(x) - l\vert < \varepsilon
	\end{align*}
\end{definizioneLimiteFinitoDestro}
\begin{definizioneLimiteFinitoSinistro}
	Sia $f$ una funzione definita in un intorno sinistro $I^-(x_0)$ di $x_0$, privato al pi� del punto
	$x_0$.\\
	Si dice che
	\begin{equation*}
	\lim\limits_{x\to x_0^-}f(x) = l
	\end{equation*}
	se, fissato comunque un numero $\varepsilon>0$, � possibile determinare in corrispondenza di esso
	un numero $\delta_\varepsilon>0$ tale che, per ogni $x\in I^-(x_0)$ verificante la condizione
	\begin{equation*}
	0<x-x_0<\delta_\varepsilon
	\end{equation*}
	risulti
	\begin{equation*}
	\abs{f(x)-l}<\varepsilon
	\end{equation*}
	In simboli
	\begin{align*}
	\lim\limits_{x\to x_0^-} f(x) &= l\\
	\ArrowBetweenLines
	\forall \varepsilon, \exists\,\delta_\varepsilon>0\mid\forall x: x_0-\delta_\varepsilon<x&<x_0
	\Rightarrow \vert f(x) - l\vert < \varepsilon
	\end{align*}
\end{definizioneLimiteFinitoSinistro}

\subsection{Definizione generale di limite}
Fin'ora abbiamo elencato varie definizioni formali ma ce n'� una generale, che le comprenda tutte? 
Certo che s� e anzi, � anche pi� facile da ricordare in quanto � una unica. Sapendo poi adattarla, si
ricavano tute le altre.
\begin{definizioneGeneraleLimite}
	Siano $V(l)$ e $U(x_0)$ due intorni dei rispettivi parametri. Si ha allora che
	\begin{align*}
	\lim\limits_{x\to x_0} f(x) &= l\\
	\ArrowBetweenLines
	\forall V(l), \exists\,U(x_0)\mid\forall x\in U(x_0)&\setminus\{x_0\}\Rightarrow f(x)\in V(l)
	\end{align*}
\end{definizioneGeneraleLimite}
Questo permette di imparare una sola formula che per�, opportunamente adattata, permette di ricavare
le definizioni formali di ogni limite.

\subsection{Teroemi sui limiti}
\begin{uniLim}\hypertarget{teor:uniLim}{}
	Se una funzione ammette limite per $x\to x_0$, tale limite � unico.
\end{uniLim}
\begin{confrontoLim}\hypertarget{teor:confLim}{}
	Siano $f$, $g$ e $h$ tre funzioni definite in un intorno $I$ di $x_0$, escluso al pi� $x_0$, e tali
	che per ogni $x\in I$ risulti
	\begin{equation*}
	f(x)\leq g(x)\leq h(x)
	\end{equation*}
	Se
	\begin{equation*}
	\lim\limits_{x\to x_0} f(x) = \lim\limits_{x\to x_0} h(x) = l
	\end{equation*}
	allora risulter�
	\begin{equation*}
	\lim\limits_{x\to x_0}g(x)=l
	\end{equation*}
\end{confrontoLim}
\begin{permanenzaSegno}\hypertarget{teor:segno}{}
	Se
	\begin{equation*}
	\lim\limits_{x\to x_0}f(x)=l\neq0
	\end{equation*}
	esiste un intorno $I(x_0)$, privato al pi� del punto $x_0$, in cui la funzione assume lo stesso 
	segno di $l$.\\
	Viceversa, se esiste un intorno $I(x_0)$ di $x_0$ privato al pi� di $x_0$, in cui risulta $f(x)>0$
	[$f(x)<0$], e se esiste $\lim\limits_{x\to x_0}f(x)=l$ si avr�
	\begin{equation*}
	l\geq0\quad[l\leq0]
	\end{equation*}
\end{permanenzaSegno}

\subsection{Operazioni sui limiti}
\subsubsection{Somma}
\begin{sommaLimiti}\hypertarget{teor:sommaLimiti}{}
	Il limite di una somma di funzioni � uguale alla somma dei limiti se questi sono finiti.
	\begin{equation*}
	\lim\limits_{x\to x_0}[f(x)+g(x)] = l_1+l_2
	\end{equation*}
\end{sommaLimiti}

\subsubsection{Prodotto}
\begin{prodottoLimiti}\hypertarget{teor:prodottoLimiti}{}
	Il limite di un prodotto di funzioni � uguale al prodotto dei limiti delle funzioni se questi sono
	finiti.
	\begin{equation*}
	\lim\limits_{x\to x_0}f(x)\cdot g(x)=l_1\cdot l_2
	\end{equation*}
\end{prodottoLimiti}
Dal prodotto si possono ricavare anche i seguenti 2 teoremi
\begin{prodottoLimiti1}
	Se $f(x)$ � una funzione che ammette limite $l$ per $x$ che tende a $x_0$ e $k$ � un numero reale,
	si ha
	\begin{equation*}
	\lim\limits_{x\to x_0}k\,f(x) = k\cdot l
	\end{equation*}
\end{prodottoLimiti1}
\begin{prodottoLimiti2}
	Se $f(x)$ e $g(x)$ sono due funzioni che per $x$ che tende a $x_0$ hanno limiti $l_1$ e $l_2$, e
	$\lambda$ e $\mu$ sono due numeri reali, si ha
	\begin{equation*}
	\lim\limits_{x\to x_0}[\lambda f(x)+\mu g(x)]=\lambda l_1+\mu l_2
	\end{equation*}
\end{prodottoLimiti2}

\subsubsection{Quoziente}
\begin{quozienteLimiti}
	Se $f(x)$ e $g(x)$ sono due funzioni aventi rispettivamente i limiti $l_1$ e $l_2$ per $x$ che 
	tende a $x_0$ e se $l_2\neq0$ si ha
	\begin{equation*}
	\lim\limits_{x\to x_0}\frac{f(x)}{g(x)}=\frac{l_1}{l_2}
	\end{equation*}
\end{quozienteLimiti}

\subsubsection{Potenza}
\begin{potenzaLimiti}
	Se $\lim\limits_{x\to x_0}f(x)=l$ e $a\in\mathbb{R}_0^+$
	\begin{equation*}
	\lim\limits_{x\to x_0}a^{f(x)} = a^l
	\end{equation*}
\end{potenzaLimiti}
\begin{potenzaLimiti1}
	Se $\lim\limits_{x\to x_0}f(x)=l>0$ e $a\in\mathbb{R}$
	\begin{equation*}
	\lim\limits_{x\to x_0}[f(x)]^a = l^a
	\end{equation*}
\end{potenzaLimiti1}
\begin{potenzaLimiti2}
	Se $\lim\limits_{x\to x_0}f(x)=l_1>0$ e $\lim\limits_{x\to x_0}g(x)=l_2$
	\begin{equation*}
	\lim\limits_{x\to x_0}[f(x)]^{g(x)} = l_1^{l_2}
	\end{equation*}
\end{potenzaLimiti2}

\subsubsection{Modulo}
\begin{moduloLimiti}
	Se $\lim\limits_{x\to x_0}=l$
	\begin{equation*}
	\lim\limits_{x\to x_0}\abs{f(x)}=\abs{l}
	\end{equation*}
\end{moduloLimiti}

\subsubsection{Logaritmo}
\begin{logLimiti}
	Se $\lim\limits_{x\to x_0}=l>0$ e $a\in\mathbb{R}_0^+\setminus\{1\}$
	\begin{equation*}
	\lim\limits_{x\to x_0}\log_af(x)=\log_al
	\end{equation*}
\end{logLimiti}

\subsection{Forme indeterminate}
Le forme indeterminate si ottengono quando si cerca di fare operazioni con limiti all'infinito. Le
forme indeterminate indicano che la sola conoscenza dei limiti delle due funzioni non determina la 
conoscenza del limite della loro operazione.\\
Quelle che vengono riquadrate di seguito sono le forme indeterminate nei vari casi.
\subsubsection{Somma}
\begin{center}
	\begin{tabular}{ccc}
		$\boldsymbol{\lim f(x)}$ & $\boldsymbol{\lim g(x)}$ & $\boldsymbol{\lim[f(x)+g(x)]}$\\\hline
		$l$ & $+\infty$ & $+\infty$\\
		$l$ & $-\infty$ & $-\infty$\\
		$\pm\infty$ & $\pm\infty$ & $\pm\infty$\\
		$\pm\infty$ & $\mp\infty$ & $\boxed{+\infty-\infty}$
	\end{tabular}
\end{center}

\subsubsection{Prodotto}
\begin{center}
	\begin{tabular}{ccc}
		$\boldsymbol{\lim f(x)}$ & $\boldsymbol{\lim g(x)}$ & $\boldsymbol{\lim[f(x)\cdot 
			g(x)]}$\\\hline
		$l\neq0$ & $\pm\infty$ & $\pm\infty$\\
		$\pm\infty$ & $\pm\infty$ & $\infty$\\
		$0$ & $\pm\infty$ & $\boxed{0\cdot\infty}$
	\end{tabular}
\end{center}

\subsubsection{Quoziente}
\begin{center}
	\begin{tabular}{ccc}
		$\boldsymbol{\lim f(x)}$ & $\boldsymbol{\lim g(x)}$ &
			$\boldsymbol{\lim\frac{f(x)}{g(x)}}$\\\hline
		$l$ & $\pm\infty$ & $0$\\
		$\pm\infty$ & $l\neq0$ & $\pm\infty$\\
		$\pm\infty$ & $\pm\infty$ & $\boxed{\frac{\pm\infty}{\pm\infty}}$\\
		$0$ & $0$ & $\boxed{\frac{0}{0}}$
	\end{tabular}
\end{center}

\subsubsection{Potenza}
\begin{center}
	\begin{tabular}{ccc}
		$\boldsymbol{\lim f(x)}$ & $\boldsymbol{\lim g(x)}$ &
		$\boldsymbol{\lim[f(x)]^{g(x)}}$\\\hline
		$l$ & $\pm\infty$ & $\pm\infty$\\
		$1$ & $\pm\infty$ & $\boxed{1^{\pm\infty}}$\\
		$+\infty$ & $0$ & $\boxed{+\infty^0}$\\
		$0$ & $0$ & $\boxed{0^0}$
	\end{tabular}
\end{center}

Per la risoluzione delle forme indeterminate, si utilizzino i limiti di una funzione razionale o 
limiti notevoli.

\subsection{Limite finito di una funzione razionale}
\begin{limiteFinitoFunzRaz}
	Quando $x$ tende a $x_0$, il limite di un polinomio coincide con il limite calcolato con 
	sostituzione.
\end{limiteFinitoFunzRaz}

Prendiamo ad esempio
\begin{equation*}
\lim\limits_{x\to3^+}\frac{x^2-5x+6}{(x-3)^2}
\end{equation*}
Se provassimo a sostituire otterremmo
\begin{equation*}
\lim\limits_{x\to3^+}\frac{x^2-5x+6}{(x-3)^2} = \frac{0}{0}
\end{equation*}
che � una forma indeterminata. In questa situazione si usa \hyperref[ruffini]{Ruffini} per ridurlo di
grado e ottenere
\begin{equation*}
\lim\limits_{x\to3^+}\frac{x^2-5x+6}{(x-3)^2} = 
\lim\limits_{x\to3^+}\frac{\cancel{(x-3)}(x-2)}{\cancel{(x-3)}(x-2)}
\end{equation*}
e per i teoremi dei limiti
\begin{equation*}
\lim\limits_{x\to3^+}\frac{(x-2)}{(x-2)} = +\infty
\end{equation*}
In generale quindi
\begin{equation*}
\lim\limits_{x\to x_0}\frac{p(x)}{q(x)} \overset{\left[\frac{0}{0}\right]}{=} =
\lim\limits_{x\to x_0}\frac{p_1(x)}{q_1(x)} = \dotsb = \begin{cases}
\text{Ruffini se }\frac{0}{0}\\
l
\end{cases}
\end{equation*}

\subsection{Limite all'infinito di una funzione razionale}
\hypertarget{teor:limiteInfinitoFunzRaz}{}
\begin{limiteInfinitoFunzRaz}
	Quando $x$ tende a $\pm\infty$, il limte di un polinomio coincide con il limite del suo monomio di
	grado pi� alto.
\end{limiteInfinitoFunzRaz}
Ad esempio consideriamo
\begin{equation*}
\lim\limits_{x\to+\infty}(3x^3-5x^2+4x+1)
\end{equation*}
Poich� per $x\neq0$
\begin{equation*}
3x^3-5x^2+4x+1=x^3\left(3-\frac{5}{x}+\frac{4}{x^2}+\frac{1}{x^3}\right)
\end{equation*}
si ha
\begin{equation*}
\lim\limits_{x\to+\infty}\left(3-\frac{5}{x}+\frac{4}{x^2}+\frac{1}{x^3}\right) = 3
\end{equation*}
allora
\begin{equation*}
\lim\limits_{x\to+\infty}(3x^3-5x^2+4x+1)=\lim\limits_{x\to+\infty}(3x^3) = +\infty
\end{equation*}
Se invece abbiamo una frazione, abbiamo
\begin{align*}
&\lim\limits_{x\to\infty}\frac{a_nx^n+a_{n-1}x^{n-1}+\dotsb+a_0}{b_mx^m+b_{m-1}x^{m-1}+\dotsb+b_0}=
\lim\limits_{x\to\infty}\frac{a_nx^n}{b_mx^m} = \\
&\lim\limits_{x\to\infty}\left(\frac{a_n}{b_m}x^{n-m}\right) =
\begin{dcases}
\infty, &\text{se } n > m\\
\frac{a_n}{b_n}, &\text{se } n = m\\
0, &\text{se } n < m
\end{dcases}
\end{align*}

\subsection{Limiti di funzioni irrazionali}
Creiamo questa sottosezione per la particolarit� che i limiti con radicali possono avere. Prendiamo ad
esempio
\begin{equation*}
\lim\limits_{x\to-\infty}\left(\sqrt{4x^2+1}-x\right)
\end{equation*}
Notiamo subito che se proviamo a sostituire, otteniamo la forma indeterminata
\begin{equation*}
-\infty+\infty
\end{equation*}
Per risolvere questo tipo di limite, isoliamo il termine di grado massimo (quello che cresce pi� 
velocemente). Quindi
\begin{equation*}
\lim\limits_{x\to+\infty}\left(\sqrt{4x^2+1}-x\right) = 
\lim\limits_{x\to+\infty}\left(\sqrt{x^2\left(4+\frac{1}{x^2}\right)}-x\right)
\end{equation*}
Ora possiamo portare fuori $x^2$ dalla radice
\begin{equation*}
\lim\limits_{x\to+\infty}\left(\sqrt{x^2\left(4+\frac{1}{x^2}\right)}-x\right) =
\lim\limits_{x\to+\infty}\left(\abs{x}\sqrt{4+\frac{1}{x^2}}-x\right)
\end{equation*}
Ora possiamo sostituire $\abs{x} = x$ perch� siamo in un intorno di $+\infty$. Questo perch� sono 
numeri sicuramente $>0$ quindi il loro valore assoluto � esattamente lo stesso loro. (Se fossimo in un
$I(-\infty)$ sostituiremmo $\abs{x} = -x$). Proseguendo nella risoluzione
\begin{align*}
&\lim\limits_{x\to+\infty}\left(\abs{x}\sqrt{4+\frac{1}{x^2}}-x\right)=
\lim\limits_{x\to+\infty}\overbrace{x}^{\mathclap{\to+\infty}}
\overbrace{\left(\sqrt{4+\frac{1}{x^2}}-x\right)}^{\mathclap{\to\sqrt{4+0}-1\to2-1\to1}}=\\
&+\infty\cdot1 = \boxed{+\infty}
\end{align*}
In generale, quindi, si deve sempre isolare il termine che cresce pi� rapidamente utilizzando a proprio
favore l'operatore $\lim$.

\subsection{Limiti notevoli}
Ci sono dei limiti particolari che � estremamente comodo conoscere a memoria. Principalmente sono 2
\begin{equation*}
\lim\limits_{x\to0}\frac{\sin x}{x}=1\qquad\lim\limits_{x\to+\infty}\left(1+\frac{1}{x}\right)^x=e
\end{equation*}
Da questi due se ne possono ricavare altri 8. Dal primo
\begin{equation*}
\lim\limits_{x\to0}\frac{\tan x}{x}=1\quad\lim\lim\limits_{x\to0}\frac{1-\cos x}{x^2}=\frac{1}{2}\quad
\lim\limits_{x\to0}\frac{1-\cos x}{x}=0
\end{equation*}
Dal secondo
\begin{equation*}
\lim\limits_{x\to0}(1+x)^{\frac{1}{x}}=e\,\lim\limits_{x\to0}\frac{a^x-1}{x}=\ln a\,
\lim\limits_{x\to0}\frac{\log_a(x+1)}{x}=\log_a e
\end{equation*}
Infine ne esiste un ultimo che � molto simile al primo ma non identico
\begin{equation*}
  \lim_{x \to \infty} \frac{\sin x}{x} = 0 
\end{equation*}


\subsection{Consigli nella risoluzione di limiti deducibili}
Prendiamo ad esempio
\begin{equation*}
\lim\limits_{x\to4^-}\frac{1}{\log_2 x -2}
\end{equation*}
Per trovare questo limite, sostituiamo $x = 4$ nella funzione. Otteniamo
\begin{equation*}
\lim\limits_{x\to4^-}\frac{1}{\log_2 2} \to \lim\limits_{x\to4^-}\frac{1}{1} = 1
\end{equation*}
Abbiamo molto semplicemente trovato il limite sostituendo. Spesso per� bisogna anche stare attenti 
da che parte $x$ tende.\\[\baselineskip]
Prendiamo un altra funzione
\begin{equation*}
\lim\limits_{x\to2^+}\sqrt{\log_2\frac{x+2}{x-2}}
\end{equation*}
Questa pu� far paura ma andando con calma, scopriamo che non � affatto difficile. Dobbiamo un po'
lavorare come con i domini: dall'interno all'esterno. Con valori numerici, sostiuiamo ad $x$ il valore
di $x_0$
\begin{equation*}
\lim\limits_{x\to2^+}x+2 \to 2^++2 \to 4^+
\end{equation*}
(Il segno qui non � obbligatorio da mantenere in quanto stiamo lavorando su $4$, se invece stessimo
usando $0$, � determinante in quanto pu� cambiare il risultato).
\begin{equation*}
\lim\limits_{x\to2^+}x-2 \to 2^+-2 \to 0^+
\end{equation*}
(Il segno invece qui � indispensabile, ora capiremo perch�)
\begin{equation*}
\lim\limits_{x\to2^+}\frac{x+2}{x-2}\to\lim\limits_{x\to2^+}\frac{4^+}{0^+}\to+\infty
\end{equation*}
Se avessimo avuto $0^-$ non ci saremmo pi� avvicinati da destra, ma da sinistra e quindi avremmo 
ottenuto $-\infty$ in quanto il grafico di $\frac{1}{x}$ per $x\to 0^+$ tende all'infinito positivo,
per $x\to0^-$ a quello negativo. Proseguendo
\begin{equation*}
\lim\limits_{x\to2^+}\log_2 +\infty \to +\infty
\end{equation*}
Questo lo si pu� capire molto facilmente dal grafico (si veda sopra). Ecco perch� conoscere i grafici
generali delle funzioni pi� comuni � molto comodo.
\begin{equation*}
\lim\limits_{x\to2^+}\sqrt{+\infty}\to+\infty
\end{equation*}
Quindi, con questo
\begin{equation*}
\lim\limits_{x\to2^+}\sqrt{\log_2\frac{x+2}{x-2}} = +\infty
\end{equation*}
In definitiva quindi il consiglio � di andare con molta calma e ricordarsi le possibilit� che la
funzione $\lim$ offre.

%!TEX ROOT=formularioMatematica.tex

\section{Funzioni continue}\label{sec:funzCont}

Una funzione $f$ si dice continua o in un punto ($x_0$) o in un intervallo ($I$). La continuit�in
un punto si ha se
\begin{equation*}
  \lim\limits_{x\to x_0}f(x) = f(x_0) 
\end{equation*}
La definizione formale quindi diventa
\begin{equation*}
  \forall\varepsilon>0,\exists\,\delta_\varepsilon>0\mid\forall x:\,\abs{x-x_0}<\delta_\varepsilon
  \Rightarrow \abs{f(x)-f(x_0)}<\varepsilon
\end{equation*}
che, se confrontata con la definizione di limite manca di un $0<$ in quanto la funzione � continua,
quindi $x_0\in\mathscr{C}(f)$.\\
La continuit� in un intervallo � una generalizzazione di quella di un punto, ovvero una funzione
� continua in un intervallo se
\begin{equation*}
  \forall x_0\in I\,f\text{ � continua in }x_0\Rightarrow f\,\text{� continua in } I
\end{equation*}

\subsection{Propriet� delle funzioni continue}
Nelle funzioni continue si mantengono i teoremi dei limiti. Ovvero siano $f$ e $g$ due funzioni.
Se entrambe sono continue in $x_0$
\begin{align*}
  f\pm g\,&\text{� continua in }x_0\\
  f\circ g\,&\text{� continua in }x_0\\
  \frac{f}{g}\,&\text{� continua in }x_0\quad \big(g(x_0)\big)\neq0
\end{align*}
Da queste propriet� ricaviamo subito che qualsiasi funzione polinomiale � continua. Questo perch�,
sia $P(x)$ una funzione polinomiale del tipo $a_nx^n+\dotsb+a_0$. La funzione costante ($y=a$) �
continua in quanto non dipende da alcuna variabile. Per $x^n$ possiamo pensarlo come $x\cdot
x\underbrace{\cdot\dotsb\cdot}_{n-\text{volte}}x$. Considerato le precedenti relazioni ed essendo
$y=x$ continua in quanto il suo dominio � tutto $\mathbb{R}$, il prodotto di funzioni
continue � un'altra funzione continua. La somma di funzioni continue � un'altra funzione continua.
Quindi \textbf{ogni funzione polinomiale � continua in ogni $x_0$}.

\subsection{Punti di discontinuit�}
Una funzione pu� essere continua solo se
\begin{equation*}
  \lim\limits_{x\to x_0}f(x)=f(x_0) 
\end{equation*}
e questo limite esiste, la funzione � definita in $x_0$ e $l=f(x_0)$. Ovviamente ci sono casi in
cui queste tre caratteristiche non si verificano. Ecco che si classificano quindi i punti di
discontinuit�, ovvero quei $x_0$ in cui la funzione pecca di queste particolarit�.

\subsubsection{Prima specie}
Si ha quando
\begin{equation*}
  \lim\limits_{x\to x_0^-}f(x) \neq \lim\limits_{x\to x_0^+}f(x) 
\end{equation*}
Questa specie � anche definita "salto" in quanto visualmente si ha un salto della funzione. Ad
esempio
\begin{center}
  \begin{tikzpicture}
    \tkzInit[xmin=-2,ymin=-2,xmax=2,ymax=2]
    \tkzGrid
    \tkzAxeXY
    
    \draw[thick, red] plot[domain=-2:0] (\x, \x+1);
    \draw[thick, red] plot[domain=0:2] (\x,\x-2);
  \end{tikzpicture}
\end{center}
ha una discontinuit� di prima specie in quanto ha un "salto" nella funzione e in $x_0=0$ la 
funzione non � continua.

\subsubsection{Seconda specie}
La seconda specie si ha quando
\begin{equation*}
  \lim\limits_{x\to x_0^-}f(x) = \pm\infty \lor \lim\limits_{x\to x_0^+}f(x) = \pm\infty
\end{equation*}
Un tipico esempio pu� essere la funzione tangente o un'iperbole equilatera.

\subsubsection{Terza specie}
Avviene quando
\begin{equation*}
  \not\exists f(x_0) \lor \lim\limits_{x\to x_0} f(x) \neq f(x_0)
\end{equation*}
Questa specie viene anche definita "eliminabile" in quanto si pu� trovare una funzione 
$\tilde{f}$ che contenga al suo interno anche un punto. Ad esempio
\begin{equation*}
  \tilde{f} =
  \begin{cases}
    x_0,\,& x = x_0\\
    f(x),& x\neq x_0
  \end{cases}
\end{equation*}

%!TEX ROOT=formularioMatematica.tex

\section{Successioni}\label{sec:successioni}
Una successione è una particolare funzione definita in modo
\begin{equation*}
  f:\,\mathbb{N} \rightarrow \mathbb{R} 
\end{equation*}
ovvero
\begin{equation*}
  n \mapsto a_n=f(n)
\end{equation*}
Una successione quindi è una serie di numeri interi relazionati fra di loro. Ci sono generalmente 3
modi per definire una successione:
\begin{enumerate}
  \item \textbf{Algebrica}
    \begin{equation*}
      a_n = 2n^2+1
    \end{equation*}
  \item \textbf{Ricorsiva} 
    \begin{equation*}
      \begin{cases}
        a_0 &= 1\\ a_n &= 2a_{n-1}-1
      \end{cases}
    \end{equation*}
  \item \textbf{Elencativa}
    \begin{equation*}
      a_n=\{1,3,5,7,\ldots\}
    \end{equation*}
\end{enumerate}
Per andare a studiare una successione, calcoliamo il limite. Dato che per polimorfismo $\mathbb{N}$
ha un solo punto di accumulazione che corrisponde a $+\infty$, il solo limite che possiamo calcolare
è
\begin{equation*}
  \lim\limits_{n \to \infty} a_n
\end{equation*}
che può assumere 4 valori e a seconda del valore che ottiene, si definisce la successione in modo
diverso.
\begin{equation*}
  \lim\limits_{n \to \infty} a_n=
  \begin{cases}
    l, &\text{la successione $\{a_n\}$ è convergente}\\
    +\infty, &\text{la succcessone $\{a_n\}$ è divergente positivamente}\\
    -\infty, &\text{la successione $\{a_n\}$ è divergente negativamente}\\
    \not\exists, &\text{la successione $\{a_n\}$ è indeterminata}
  \end{cases}
\end{equation*}
La definizione formale del limite quindi cambia leggermente definizione
\begin{align*}
  \lim\limits_{n \to \infty} a_n=l \Leftrightarrow \forall\varepsilon>0,\,\exists\bar{n}_\varepsilon
  \mid\forall n>\bar{n}_\varepsilon \Rightarrow \abs{a_n-l}<\varepsilon
\end{align*}

\subsection{Teorema sulle successioni}
\begin{successioniMonotone}
  Se $a_n$ è crescente e limitata superiormente, allora ammette limite che coincide con l'estremo
  superiore.
\end{successioniMonotone}

\subsection{Serie numeriche}
Una serie numerica è una somma di una successione. Una somma infinita però. Quindi
\begin{equation*}
  \sum\limits^{\infty}_{i=1} a_i = a_1+a_2+a_3+\dotsb+a_n+\dotsb
\end{equation*}
Per studiare una serie, si studia il limite a ll'infinito ma facendolo così direttamente non è
possibile, quindi si devono creare delle somme parziali. Ad esempio
\begin{align*}
  \text{Sia }a_1,a_2,\ldots,&a_n,\ldots\text{ una successione}\\
  s_1 &= a_1\\
  s_2 &= a_1+a_2\\
  s_n &= \sum\limits^{n}_{i=1} a_i
\end{align*}
Se quindi $\{s_n\}$ è una successione di somme parziali,
\begin{equation*}
  \lim\limits_{n \to \infty} s_n = S
\end{equation*}
dove $s_n$ deve essere convergente e $S$ è la somma della serie. Generalizzando quindi
\begin{equation*}
  \sum\limits^{\infty}_{i=1} = \lim\limits_{n \to \infty} s_n = S
\end{equation*}

\subsubsection{Serie di Mengoli-Cauchy}
Questa serie è forse la più celebre e può far capire come approcciarsi alle serie
\begin{equation*}
  \sum\limits^{\infty}_{i=1} \frac{1}{i(i+1)}
\end{equation*}
Per andare a risolvere questa serie bisogna riscrivere il parametro in quanto altrimenti
il limite all'infinito avrebbe una forma indeterminata del tipo $\infty\cdot\infty$. Quindi 
possiamo scrivere
\begin{equation*}
  \frac{1}{i(i+1)} = \frac{A(i+1)+Bi}{i(i+1)} = \frac{(A+B)i+A}{i(i+1)}
\end{equation*}
dove $A$ e $B$ rappresentano i coefficienti. Per la proprietà d'identità dei polinomi scriviamo
\begin{equation*}
  \begin{cases}
    A+B=0\\A=1
  \end{cases}
  \begin{cases}
    B=-1\\A=1
  \end{cases}
\end{equation*}
Quindi
\begin{equation*}
  \frac{1}{i(i+1)} = \frac{1}{i}-\frac{1}{i+1}
\end{equation*}
Di conseguenza la nostra serie diventa
\begin{equation*}
  \sum\limits^{\infty}_{i=1} \left( \frac{1}{i}-\frac{1}{i+1} \right)
\end{equation*}
Ora quindi abbiamo riscritto la serie in modo che sia di facile verifica. Andando ora ad osservare
le somme parziali
\begin{align*}
  \label{eq:}
  s_1 &= 1-\frac{1}{2}=\frac{1}{2}\\
  s_2 &= 1-\cancel{\frac{1}{2}}+\cancel{\frac{1}{2}}-\frac{1}{3}=\frac{2}{3}\\
  s_3 &= 1-\cancel{\frac{1}{2}}+\cancel{\frac{1}{2}}-\cancel{\frac{1}{3}}+\cancel{\frac{1}{3}}
  -\frac{1}{4}=\frac{3}{4}
\end{align*}
Notiamo che $1$ rimane sempre e che rimane anche il termine $-\frac{1}{i+1}$. Quindi la
somma parziale generalizzata è
\begin{equation*}
  s_i=1-\frac{1}{i+1}
\end{equation*}
Il limite dunque diventa
\begin{equation*}
  \lim\limits_{i\to\infty}  \left(1-\frac{1}{i+1} \right) = 1
\end{equation*}
Di conseguenza
\begin{equation*}
  \sum\limits^{\infty}_{i=1} \frac{1}{i(i+1)} = 1
\end{equation*}

\subsubsection{Progressioni geometriche}
Sia $a_1,a_2,\ldots,a_n,\ldots$ una progressione geometrica i cui elementi sono
\begin{align*}
  a_2&=q_a1\\
  a_3&=q^2a_1\\
  a_4&=q^3a_1\\
  a_n&=q^{n-1}a_1
\end{align*}
Dove le somme parziali sono uguali a 
\begin{equation*}
  s_n = a_1 \frac{1-q^n}{1-q}
\end{equation*}
stando alle formule trattate nelle sezioni precedenti. Quindi la serie di una progressione
geometrica è
\begin{equation*}
  \sum\limits^{\infty}_{i=1} x^{i-1}=\sum\limits^{\infty}_{i=0} x^i
\end{equation*}
e il limite della somma parziale
\begin{equation*}
  \lim\limits_{n \to \infty} a_1 \frac{1-q^n}{1-q}=
  \frac{a_1}{1-q}\lim\limits_{n \to \infty} \left( 1-q^n \right)
\end{equation*}
che può assumere 3 valori a seconda della ragione ($q$) della progressione.
\begin{equation*}
  \frac{a_1}{1-q}\lim\limits_{n \to \infty} \left( 1-q^n \right) =
  \begin{cases}
    0, &\abs{q}<1\\
    \pm\infty, &q>1\\
    \not\exists, &q\leq-1
  \end{cases}
\end{equation*}

%!TEX ROOT=formularioMatematica.tex

\section{Derivate}\label{sec:derivate}

Il concetto di derivata � uno dei concetti fondamentali della matematica. � la base delle equazioni
differenziali, integrali, calcolo infinitesimale e molto altro.\\
Per arrivare ad una definizione di derivata, si possono prendere due strade, le stesse che Newton e 
Leibniz intrapresero. Osserveremo il metodo di Leibniz in quanto � pi� matematico. Quello di Newton
invece ha pi� riferimenti con la fisica.\\
Leibniz si era posto il problema di come trovare la tangente in un punto di una curva. Ad esempio
\begin{center}
  \begin{tikzpicture}
    \coordinate (X0) at (2,0.70);
    \tkzInit[xmin=0,ymin=-1,xmax=5,ymax=3]
    \tkzGrid
    \tkzAxeXY
    \draw[domain=0.35:5, thick, red] plot({\x}, {ln(\x)});
    \filldraw[blue] (X0) circle (0.05)
      node[above left]{$P(x_0,f(x_0))$};
  \end{tikzpicture}
\end{center}
come troviamo la tangente ad $f(x)$ (in rosso) in $P$ (in blu)? L'idea qui � quella di fissare un
altro punto ($Q$) di coordinate ($(x,f(x))$) e trovare la retta che passa tra questi due punti che,
ovviamente, risulter� secante alla curva. Poi si avviciner� sempre di pi� il punto $Q$ a $P$ in modo
che la retta tra i due punti, risulti in definitiva tangente alla curva.
\begin{center}
  \begin{tikzpicture}
    \coordinate (P) at (2,0.70);
    \coordinate (Q) at (4,1.4);
    \tkzInit[xmin=0,ymin=-1,xmax=5,ymax=3]
    \tkzGrid
    \tkzAxeXY

    \draw[domain=0.35:5, thick, red] plot({\x}, {ln(\x)});
    \draw[blue, thick, shorten >= -1.5cm, shorten <= -2cm] (P) -- (Q);

    \filldraw[blue] (Q) circle (0.05)
      node[below right]{$Q(x,f(x))$};
    \filldraw[blue] (X0) circle (0.05)
      node[above left]{$P(x_0,f(x_0))$};
  \end{tikzpicture}
\end{center}
Ora noi vorremmo trovare questa retta. Per prima cosa, troviamo il coefficiente angolare
\begin{equation*}
  m_{PQ} = \frac{y_Q-y_P}{x_Q-x_P} = \frac{f(x)-f(x_0)}{x-x_0}
\end{equation*}
Questa frazione, � definita \textbf{rapporto incrementale nel punto $x_0$}. Spesso � scritta anche
come 
\begin{equation*}
  m = \frac{\Delta f}{\Delta x}
\end{equation*}
o, ponendo $x-x_0 = h$
\begin{equation*}
  m = \frac{f(x_0+h)-f(x_0)}{h}
\end{equation*}
Ora per trovare quello della tangente, dobbiamo avvicinare $Q$ a $P$, quindi
\begin{equation*}
  m_t = \lim\limits_{x\to x_0} \frac{f(x)-f(x_0)}{x-x_0} = 
  \lim\limits_{h\to0} \frac{f(x_0+h)-f(x_0)}{h}
\end{equation*}
Se questo limite esiste ed � finito, la funzione $f$ si dice \emph{derivabile} in $x_0$ e si pone
\begin{equation*}
  f'(x_0) = \lim\limits_{x\to x_0} \frac{f(x)-f(x_0)}{x-x_0} = 
  \lim\limits_{h\to0} \frac{f(x_0+h)-f(x_0)}{h}
\end{equation*}
Altre notazioni sono
\begin{equation*}
  \Dif f(x_0)\quad\text{e}\quad \frac{\dif f}{\dif x}(x_0)
\end{equation*}

\subsection{Teoremi sulle derivate}
Tra i teoremi sulle derivate e le derivate delle funzioni elementari, si potranno calcolare con
estrema facilit� tutte le derivate che si proporranno.

\subsubsection{Somma}
\begin{equation*}
  \Dif \left[ f(x)\pm g(x) \right] = \Dif f(x)\pm\Dif g(x)
\end{equation*}
Si estenda la propriet� al numero di funzioni desiderato. Si faccia comunque la somma algebrica.

\subsubsection{Prodotto}
\begin{equation*}
  \Dif \left[ f(x)\cdot g(x) \right] \left[ \Dif f(x) \right]g(x) + f(x)\left[ \Dif g(x) \right]
\end{equation*}
Si noti che se $g(x) = k\lor f(x) = k$,
\begin{equation*}
  \Dif \left[ k\cdot f(x) \right] = k\Dif f(x)
\end{equation*}
Si noti che per un numero $\alpha\in\mathbb{R}$ di funzioni $f(x)$, il loro prodotto diventerebbe
una pontenza, in particolar modo
\begin{equation*}
  \Dif[f(x)]^\alpha = \alpha f'(x)[f(x)]^{\alpha-1}
\end{equation*}

\subsubsection{Quoziente}
\begin{equation*}
  \Dif \left( \frac{f(x)}{g(x)} \right) = \frac{[\Dif f(x)]g(x)-f(x)\Dif g(x)}{[g(x)]^2}
\end{equation*}

\subsubsection{Derivata di una funzione inversa}
\begin{equation*}
  \Dif f^{-1}(x_0) = \frac{1}{\Dif f(x_0)}
\end{equation*}

\subsubsection{Derivata di una funzone composta}
\begin{equation*}
  \Dif [f(g(x))] = \Dif f(g(x))\cdot\Dif g(x)
\end{equation*}
Se sono presenti pi� funzioni si estenda di conseguenza. Quindi
\begin{equation*}
  \Dif [f(g(h(x)))] = \Dif f(g(h(x)))\cdot\Dif g(h(x))\cdot\Dif h(x)
\end{equation*}
e cos� via per altre funzioni.

\subsection{Derivate fondamentali}
Verranno ora riportate le derivate fondamentali delle funzioni elementari. Con queste e con i
teoremi sar� possibile trovare una qualunque derivata di una qualunque funzione.
\tablefirsthead{\toprule \textbf{Funzione} & \textbf{Derivata}\\ \midrule}
\tablehead{\textbf{Funzione} & \textbf{Derivata}\\ \midrule}
\tablelasttail{\bottomrule}
\begin{center}
  \begin{xtabular}{M{2cm} M{5cm}}
    $c\,(c\in\mathbb{R})$ & $0$\\ \midrule
    $x^\alpha\, (\alpha\in\mathbb{R})$ & $\alpha x^{\alpha-1}$\\ \midrule
    $\sin x$ & $\cos x$\\ \midrule
    $\cos x$ & $-\sin x$\\ \midrule
    $\tan x$ & $1+\tan^2x=\frac{1}{\cos^2x}$\\ \midrule
    $\cot x$ & $-1-\cot^2x=-\frac{1}{\sin^2x}$\\ \midrule
    $\log_ax$ & $\frac{1}{x}\log_ae$\\ \midrule
    $a^x$ & $a^x\ln e$\\ \midrule
    $\arcsin x$ & $\frac{1}{\sqrt{1-x^2}}$\\ \midrule
    $\arccos x$ & $-\frac{1}{\sqrt{1-x^2}}$\\ \midrule
    $\arctan x$ & $\frac{1}{1+x^2}$\\ \midrule
    $\arccot x$ & $-\frac{1}{1+x^2}$\\
  \end{xtabular}
\end{center}
Si noti che � presente anche questa formula
\begin{equation*}
  \Dif[f(x)]^{g(x)}=[f(x)]^{g(x)} \left[ \Dif g(x)\ln f(x)+g(x) \frac{\Dif f(x)}{f(x)} \right]
\end{equation*}
che risulta essere molto difficile da ricordare ma � facilmente ricavabile sfruttando la nota
caratteristica
\begin{equation*}
  f(x)^{g(x)} = e^{g(x)\ln f(x)}
\end{equation*}

\subsection{Derivate successive}
Le derivate successive sono derivate di derivate di derivate \ldots\\
La notazione pu� essere o usando numeri romani sopra la funzione
\begin{equation*}
  f^I(x),\,f^{II}(x),\,f^{III}(x)
\end{equation*}
oppure con numeri arabi come pedice
\begin{equation*}
  f_1(x),\,f_2(x),\,f_3(x)
\end{equation*}
o infine usare gli apostrofi conseguentamente (da preferirsi per numeri piccoli)
\begin{equation*}
  f'(x),\,f''(x),\,f'''(x)
\end{equation*}

\subsection{Rapporto tra continuit� e derivabilit�}
Andiamo a verificare se dalla continuit� di una funzione, possiamo dedurre la derivabilit�.
Prendiamo come funzione
\begin{equation*}
  f(x) = \abs{x}
\end{equation*}
� continua in quanto $\mathbb{R}$ � il suo dominio e il suo codominio. L'unico punto che pu�
essere di qualche difficolt� � $x_0=0$. Quindi andiamo a verificare la derivata sinistra 
e la derivata destra.
\begin{equation*}
  f_+'(x) = \lim\limits_{x\to0^+} \frac{\abs{x}}{x} = 1
\end{equation*}
\begin{equation*}
  f_-'(x) = \lim\limits_{x\to0^-} \frac{\abs{x}}{x} = -1 
\end{equation*}
Dato che le due derivate sono diverse,
\begin{equation*}
  \not\exists\,f'(0)
\end{equation*}
Quindi \textbf{se $f(x)$ � continua in $x_0$ non � detto che sia derivabile in quel punto}.\\
[\baselineskip]
Ora invece proviamo a vedere se
\begin{equation*}
  \lim\limits_{x\to x_0}f(x) = f(x_0)
\end{equation*}
Scriviamo l'identit�
\begin{equation*}
  f(x) = f(x)
\end{equation*}
Addizioniamo e sottraiamo $f(x_0)$ e dividiamo e moltiplichiamo $x-x_0$, otteniamo
\begin{equation*}
  f(x) = f(x_0)+\frac{f(x)-f(x_0)}{x-x_0}(x-x_0)
\end{equation*}
Quindi ora possiamo scrivere
\begin{equation*}
  \lim\limits_{x\to x_0} f(x)=
  \lim\limits_{x\to x_0}\left[ f(x_0)+\overbrace{\frac{f(x)-f(x_0)}{x-x_0}}^{f'(x_0)}
  \overbrace{(x-x_0)}^{0}\right] = f(x_0)
\end{equation*}
Abbiamo quindi dimostrato che \textbf{se $f(x)$ � derivabile in $x_0$, allora $f(x)$ � continua}. 

\subsection{Punti di non derivabilit�}
I punti di non derivabilit� sono quei punti che vengono fuori da
\begin{equation*}
  \mathcal{D}_f - \mathcal{D}_{f'} 
\end{equation*}
Esistono 3 tipi di non derivabilit�.
\subsubsection{Punti angolosi}
$x_0$ � un punto angoloso per $f(x)$ se $f_+'(x)$ e $f_-'(x) $sono finiti (o al massimo uno � 
infinito) e $f_+'(x)\neq f_-'(x)$.\\
Un chiaro esempio si vede in $f(x)=\abs{x^2-c}$.
\begin{center}
  \begin{tikzpicture}
    \tkzInit[xmin=-4,ymin=-1,xmax=4,ymax=6]
    \tkzGrid
    \tkzAxeXY
    \draw[domain=-3:3, red, thick, samples=500] plot(\x,{abs(\x*\x-3)});
    \draw[domain=-0.4:2.1, blue, thick] plot(\x,{-2.8*\x+4.9});
    \draw[domain=1.5:3.3, blue, thick] plot (\x, {4*\x-7});
  \end{tikzpicture}
\end{center}
I punti in cui $f(x)=0$ sono detti angolosi in quanto le due tangenti verso destra e verso sinistra
sono diverse.

\subsubsection{Cuspidi}
$x_0$ si definisce cuspide se
\begin{equation*}
  \lim\limits_{\Delta x\to 0^-} \frac{\Delta f}{\Delta x}=-\infty\quad
  \lim\limits_{\Delta x\to 0^+} \frac{\Delta f}{\Delta x}=+\infty
\end{equation*}
I due tipici cuspidi sono
\begin{center}
  \begin{tikzpicture}
    \clip (-3.5,-1.5) rectangle (6,4);
    \tkzInit[xmin=-3,ymin=-1,xmax=3,ymax=3]
    \tkzGrid
    \tkzAxeXY
    \draw[red, thick] (-3,2) to[out=0,in=90] (0,0);
    \draw[red, thick] (3,2) to[out=180,in=90] (0,0);
  \end{tikzpicture}
\end{center}
e
\begin{center}
  \begin{tikzpicture}
    \clip (-3.5,-1.5) rectangle (6,4);
    \tkzInit[xmin=-3,ymin=-1,xmax=3,ymax=3]
    \tkzGrid
    \tkzAxeXY
    \draw[red, thick] (-3,0) to[out=0,in=270] (0,2);
    \draw[red, thick] (3,0) to[out=180,in=270] (0,2);
  \end{tikzpicture}
\end{center}

\subsubsection{Flessi a tangente verticale}
$x_0$ � un flesso se
\begin{equation*}
  \lim\limits_{\Delta x\to0} \frac{\Delta f}{\Delta x} = \pm\infty
\end{equation*}
I due tipici flessi sono
\begin{center}
  \begin{tikzpicture}
    \clip (-3.5,-1.5) rectangle (6,4);
    \tkzInit[xmin=-3,ymin=-1,xmax=3,ymax=3]
    \tkzGrid
    \tkzAxeXY
    \draw[red, thick] (-3,-1) to[out=0,in=270] (0,1);
    \draw[red, thick] (3,3) to[out=180,in=90] (0,1);
  \end{tikzpicture}
\end{center}
e
\begin{center}
  \begin{tikzpicture}
    \clip (-3.5,-1.5) rectangle (6,4);
    \tkzInit[xmin=-3,ymin=-1,xmax=3,ymax=3]
    \tkzGrid
    \tkzAxeXY
    \draw[red, thick] (-3,3) to[out=0,in=90] (0,1);
    \draw[red, thick] (3,-1) to[out=180,in=270] (0,1);
  \end{tikzpicture}
\end{center}

%!TEX ROOT=formularioMatematica.tex

\section{Integrali}
\subsection{Integrali indefiniti}
Sia $f$ una funzione continua in $[{a,b}]$. Allora $f'(x)$ � la sua derivata, $F(x)$ � una primitiva.
Bisogna quindi definire una primitiva di una funzione
\begin{primitiva}
  Una funzione $F(x)$ si dice primitiva della funzione $f(x)$ continua in $[{a,b}]$ se
  \begin{equation*}
    F'(x) = f(x)
  \end{equation*}
\end{primitiva}
Sia ad esempio $f(x) = \cos x$, $F(x)$ sar� allora quella funzione la cui derivata � $f(x)$. Quindi
$F(x)=\sin x$. Ma � soltanto questa? No, infatti anche $\sin x +1$ o $\sin x -\frac{e}{4}$ o 
qualsiasi altra funzione che abbia una costante. Cos� possiamo definire un insieme denominato
\textbf{totalit� delle primitive} che le raccoglie tutte. L'operatore che permette di trovare questo
insieme � \textbf{l'integrale indefinito}
\begin{equation*}
  \int f(x)\,\dif x = \{F(x)+c\}
\end{equation*}
Scrivere $\dif x$ � necessario perch�, come nella scrittura di Leibniz per le derivate, indica per
quale lettera si deve integrare.

\subsubsection{Propriet� dell'integrale indefinito}
Per la definizione stessa di integrale si ha che
\begin{equation*}
  \Dif\int f(x)\,\dif x = f(x)
\end{equation*}
e
\begin{equation*}
  \int\Dif f(x)\,\dif x = f(x)+c
\end{equation*}
Se $f(x)$ � una funzione continua e $k$ una costante, si ha
\begin{equation*}
  \int kf(x)\,\dif x = k\int f(x)\,\dif x
\end{equation*}
Se si ha una una somma di funzioni $\sum f$,
\begin{equation*}
  \int\sum\limits^{n}_{i=1} f_i(x)\,\dif x = \sum\limits^{n}_{i=1} \int f_i(x)\,\dif x
\end{equation*}

\subsubsection{Integrali indefiniti immediati}
Di seguito verr� riportata una tabella con i principali integrali indefiniti immediati e le 
principali funzioni composte
\tablefirsthead{\midrule}
\tablehead{\midrule}
\tablelasttail{\bottomrule}
\renewcommand*{\arraystretch}{3.3}
\begin{center}
  \begin{xtabular}{M{4cm}|M{4cm}}
    $\int {[f(x)]}^\alpha\,\dif x = \frac{{[f(x)]}^{\alpha+1}}{\alpha+1}+c$ & 
      $\int\frac{f'(x)}{\sin^2 f(x)}\,\dif x = -\cot f(x) +c$\\ 
    $\int \frac{f'(x)}{f(x)}\,\dif x = \ln\abs{f(x)}+c$ &
      $\int f'(x)a^{f(x)}\,\dif x = a^{f(x)}\log_a e + c$\\ 
    $\int f'(x)\sin f(x)\,\dif x = -\cos f(x)+c$ &
      $\int f'(x)\cos f(x)\,\dif x = \sin f(x) + c$\\ 
    $\int \frac{f'(x)}{\sqrt{1-{[f(x)]}^2}}\,\dif x = \arcsin f(x) + c$ &
      $\int \frac{f'(x)}{\cos^2 f(x)}\,\dif x = \tan f(x) + c$\\ 
    $\int \frac{f'(x)}{1+{[f(x)]}^2}\,\dif x = \arctan f(x) + c$ & \\ 
  \end{xtabular}
\end{center}
\renewcommand*{\arraystretch}{2.4}

%!TEX ROOT=formularioMatematica.tex

\section{Studio di funzione}
Lo studio di funzione è una tecnica per ricavare il grafico di una funzione che viene fornita. Questo
si basa su edi teoremi fondamentali e su tutte le conoscenze pregresse.

\subsection{Teoremi fondamentali del calcolo differenziale}
I teoremi fondamentali sono 3. Ciascuno di essi prende una parte più generale del precedente.
\subsubsection{Teorema di Rolle}
\begin{rolle}\hypertarget{teor:rolle}{}
  Sia $f$ una funzione definita e continua in $[a,b]$ e derivabile in $]a,b[$ e inoltre si abbia
  $f(a) = f(b)$. Allora
  \begin{equation*}
    \exists\,x_0\in{]a,b[}\suchthat f'(x_0)=0
  \end{equation*}
\end{rolle}

\subsubsection{Teorema di Lagrange}
\begin{lagrangeDef}\hypertarget{teor:lagrange}{}
  Sia $f$ una funzione definita e continua in $[a,b]$ e derivabile in $]a,b[$. Allora
  \begin{equation*}
    \exists\,x_0\in]a,b[\suchthat f'(x_0) = \frac{f(b)-f(a)}{b-a}
  \end{equation*}
\end{lagrangeDef}
Il teorema di Lagrange contiene al suo interno anche due lemmi (o corollari)

\begin{lagrangeLemma1}\hypertarget{teor:lagrange:1}{}
  Sia $f$ una funzione definita e continua in $I=[a,b]$ e derivabile in $\dot{I}=]a,b[$ e tale che
  \begin{equation*}
    \forall x\in\dot{I} \Rightarrow f'(x)>0
  \end{equation*}
  allora $f$ è \textbf{crescente} in $I$ e tale che
  \begin{equation*}
    \forall x\in\dot{I} \Rightarrow f'(x)<0
  \end{equation*}
  allora $f$ è \textbf{decrescente} in $I$.
\end{lagrangeLemma1}

\begin{lagrangeLemma2}\hypertarget{teor:lagrange:2}{}
  Sia $f$ una funzione definita e continua in $I=[a,b]$ con derivata nulla in $\dot{I}=]a,b[$, 
  allora
  \begin{equation*}
    f(x) = k
  \end{equation*}
\end{lagrangeLemma2}

\subsection{Teoremi sulle derivate seconde}
Il teorema di Lagrange e il suo primo lemma associa la monotonia al segno della derivata prima.
Ci sono però dei teoremi che associano la derivata seconda alla concavità.
\begin{derivataSeconda1}\hypertarget{teor:derSec:1}
  Sia $x_0\in]a,b[$ e $f$ derivabile nello stesso intorno $n$-volte. Se in $x_0$ si ha
  \begin{equation*}
    f'(x_0)=f''(x_0)=\dotsb=f^{(n-1)}(x_0)=0\land f^{(n)}(x_0)\neq0
  \end{equation*}
  si hanno i seguenti casi
  \begin{description}
    \item[$n$ pari] se
      \begin{description}
        \item[$f^{(n)}(x_0)>0$] $x_0$ è un minimo relativo
        \item[$f^{(n)}(x_0)<0$] $x_0$ è un massimo relativo
      \end{description}
    \item[$n$ dispari] in $x_0$ è presente un flesso a tangente orizzontale
  \end{description}
\end{derivataSeconda1}
\begin{derivataSeconda2}\hypertarget{teor:derSec:2}
  Sia $f$ una funzione tale che $\exists\,f''(x)$
  \begin{description}
    \item[Se $f''(x_0)>0$] ha concavità verso l'alto
    \item[Se $f''(x_0)<0$] ha concavità verso il basso
    \item[Se $f''(x_0)=0$ e $f'''(x_0)\neq0$] in $x_0$ ha un flesso
  \end{description}
\end{derivataSeconda2}

\subsection{Esempio}
Avendo ora tutte le cose necessarie per farlo, definiamo i passaggi per studiare una funzione
\begin{enumerate}
  \item Dominio
  \item Intersezione con gli assi
  \item Simmetria
  \item Periodicità
  \item Segno
  \item Asintoti
  \item Continuità e derivabilità
  \item Massimi e minimi
  \item Concavità e flessi
\end{enumerate}
Per chiarire il tutto, prendiamo ad esempio la seguente funzione
\begin{equation*}
  f(x) = \frac{x^2-5x+4}{x-5}
\end{equation*}
Per prima cosa quindi troviamo il dominio
\begin{equation*}
  x^2-x\neq0 \rightarrow x(x-1)\neq 0
\end{equation*}
quindi
\begin{equation*}
  \mathcal{D} = \mathbb{R} \setminus \{0,1\}
\end{equation*}
\begin{center}
  \begin{tikzpicture}
    \coordinate (O) at (4,0);
    \coordinate (A) at (6,0);
    \domainplot{O}{A}{{5}}
    %\drawSign{(\x^2-5*\x+4)/(\x-5+0.001)}{-2}{7}{0.25}{5}
  \end{tikzpicture}
\end{center}
Trovato il dominio, troviamo le intersezioni con gli assi
\begin{equation*}
  \begin{cases}
    y=0\\
    y = \frac{x^2-5x+4}{x-5}
  \end{cases}
  \begin{cases}
    y = 0\\
    x-5x+4=0
  \end{cases}
  \begin{cases}
    x_{1/2} =
    \begin{cases}
      x_1 = 4\\ x_2=1
    \end{cases}
  \end{cases}
\end{equation*}
Quindi i due punti di intersezione con $x$ sono
\begin{equation*}
  A(4,0)\quad B(1,0)
\end{equation*}
Ora con $y$
\begin{equation*}
  \begin{cases}
    x=0\\
    y=-\frac{4}{5}
  \end{cases}
\end{equation*}
e quindi il punto è immediato
\begin{equation*}
  C \left( -\frac{4}{5},0 \right)
\end{equation*}
Prima di andare a disegnare queste informazioni, troviamone altre due che miglioreranno enormemente
il disegno. La prima sono gli asintoti.\\
Troviamo gli eventuali asintoti verticali. Dato che $5\notin\mathcal{D}$,
\begin{equation*}
  \lim\limits_{x\to5^-} \frac{x^2-5x+4}{x-5} = \frac{4}{0^-} = -\infty
\end{equation*}
e
\begin{equation*}
  \lim\limits_{x\to5^+} \frac{x^2-5x+4}{x-5} = \frac{4}{0^+}= +\infty
\end{equation*}
Sappiamo quindi che $x=5$ è \textbf{asintoto verticale}.\\
Andiamo alla ricerca di un asintoto orizzontale (anche se vedendo la funzione possiamo subito
vedere che non è presente in quanto il grado del numeratore è maggiore di quello del denominatore)
\begin{equation*}
  \lim\limits_{x\to\infty} \frac{x^2-5x+4}{x-5} = 
  \lim\limits_{x\to\infty} 
  \frac{x^2\left(-\frac{5}{x}+\frac{4}{x^2}\right)}{x\left(1-\frac{5}{x}\right)}=
  \infty
\end{equation*}
Quindi non ci sono asintoti orizzontali e dobbiamo andare a cercarne di obliqui quindi
\begin{equation*}
  q = \lim\limits_{x\to\infty} \frac{x^2-5x+4}{x-5}\cdot \frac{1}{x} =
  \lim\limits_{x\to\infty} \frac{x^2-5x+4}{x^2-5x} = 1
\end{equation*}
\begin{align*}
  m &= \lim\limits_{x\to\infty} \left[\frac{x^2-5x+4}{x-5}-x\right]=
  \lim\limits_{x\to\infty} \frac{x^2-5x+4-5x^2+5x}{x-5}=\\
  &\lim\limits_{x\to\infty} \frac{4}{x-5}=0
\end{align*}
Questo significa che $y=x$ è \textbf{asintoto obliquo}.\\
Prima di disegnare, verifichiamo che i conti siano corretti andando a calcolare il segno della 
funzione
\begin{center}
  \begin{tikzpicture}
    \drawSign{(\x^2-5*\x+4)/(\x-5+0.001)}{-1}{7}{0.25}{5}
    \end{tikzpicture}
  \end{center}
  Vediamo che effettivamente le informazioni degli asintoti e quella del segno combaciano. Prima
  di $5$ la funzione tende a $-\infty$ quindi è negativa, poi diventa positiva. Vediamo anche che
  $1$ e $4$ sono punti d'intersezione e quindi che il segno cambia. Disegnamo ciò che sappiamo
  \begin{center}
    \begin{tikzpicture}
      \begin{axis}[xmin=-3,ymin=-5,xmax=10,ymax=20]
        \coordinate (O) at (0,0);
        \node[fill=white,circle,inner sep=0pt] (O-label) at ($(O)+(-135:10pt)$) {$O$};
        \addplot[green,domain=-5:18] {x};
        \addplot[red,thick,samples=51,unbounded coords=jump,domain=-5:-2.5] {(x^2-5*x+4)/(x-5)};
        \addplot[red,thick,samples=51,unbounded coords=jump,domain=5.1:5.4] {(x^2-5*x+4)/(x-5)};
        \addplot[red,thick,samples=51,unbounded coords=jump,domain=15:17.7] {(x^2-5*x+4)/(x-5)};
        \addplot[red,thick,samples=51,unbounded coords=jump,domain=4.5:4.9] {(x^2-5*x+4)/(x-5)};
        \addplot+[blue,mark=none] coordinates {(5,-5) (5,20)};
        \LabelPoint[mark=*,mark size=1.5][above]{4}{0}{$A$}
        \LabelPoint[mark=*,mark size=1.5][above]{1}{0}{$B$}
        \LabelPoint[mark=*,mark size=1.5][below right]{0}{-4/5}{$C$}
      \end{axis}
    \end{tikzpicture}
  \end{center}
  Questo rapido grafico contiene le informazioni che abbiamo già trovato, i due asintoti e il segno.
  Sono anche riporati i punti di intersezione precedentemente trovati.\\
  A questo punto vediamo che la funzione è continua in quanto è polinomiale. Andiamo a vedere se
  è derivabile e qual è la sua derivata prima.
  \begin{align*}
    f'(x) &= \frac{\Dif(x^2-5x+4)(x-5)-(x^2-5x+4)\Dif(x-5)}{(x-5)^2} =\\
          &= \frac{(2x-5)(x-5)-(x^2-5x+4)}{(x-5)^2} = \frac{x^2-10x+21}{(x-5)^2}
  \end{align*}
  Trovata la derivata, studiando il segno si vede che
  \begin{center}
    \begin{tikzpicture}
      \drawSign[scale=0.75]{(\x^2-10*\x+21)/(\x+5.001)^2}{0}{10}{0.25}{}
      \end{tikzpicture}
    \end{center}
    Questo significa che in $x=3$ e in $x=7$ cambia segno. La funzione da crescente diventa decrescente
    e viceversa. Questo significa che se la derivata calcolata in quel punto è pari a $0$, sono
    un punto di minimo e uno di massimo.
    \begin{align*}
      f'(3) &= \frac{9-30+21}{(3-5)^2} = 0\\
      f'(7) &= \frac{49-70+21}{(7-5)^2} = 0
    \end{align*}
    E quindi in $3$ e $7$ si hanno rispettivamente un punto di massimo e uno di minimo. Calcoliamone il
    valore della $y$ e completiamo lo studio
    \begin{equation*}
      m \left( 3,1 \right)\quad M \left( 7,9 \right)
    \end{equation*}
    Quindi il disegno ora completo è
    \begin{center}
      \begin{tikzpicture}
        \begin{axis}[xmin=-3,ymin=-5,xmax=10,ymax=20]
          \coordinate (O) at (0,0);
          \node[fill=white,circle,inner sep=0pt] (O-label) at ($(O)+(-135:10pt)$) {$O$};
          \addplot[green,domain=-5:18] {x};
          \addplot[red,thick,samples=91,smooth,unbounded coords=jump,domain=-5:4.90] 
            {(x^2-5*x+4)/(x-5)};
          \addplot[red,thick,samples=91,smooth,unbounded coords=jump,domain=5.10:18] 
            {(x^2-5*x+4)/(x-5)};
          \addplot+[blue,mark=none] coordinates {(5,-5) (5,20)};
          \LabelPoint[mark=*,mark size=1.5][above]{4}{0}{$A$}
          \LabelPoint[mark=*,mark size=1.5][above]{1}{0}{$B$}
          \LabelPoint[mark=*,mark size=1.5][below right]{0}{-4/5}{$C$}
          \LabelPoint[mark=*,mark size=1.5][above]{3}{1}{$m$}
          \LabelPoint[mark=*,mark size=1.5][below left]{7}{9}{$M$}
        \end{axis}
      \end{tikzpicture}
    \end{center}

%!TEX ROOT=formularioMatematica.tex

\section{Equazioni differenziali}
Se nell'algebra tradizionale si lavora con variabiili come $x$ che rappresentanoun numero, non è
l'unica possibilità. Se si mettono in relazione una variabile $x$, una funzione $f(x)$ e sue
derivate successive, si ottiene un'equazione differenziale. Essa è spesso espressa nella forma
\begin{equation*}
  F\left(x,y,y',y'',\ldots,y^{(n)}\right)=0
\end{equation*}
Si definisce \textbf{ordine} di un'equazione differenziale il massimo ordine di derivazione. Si
definisce \textbf{grado} il grado della funzione di ordine massimo.\\
Ci sono due tipi di soluzioni: \textbf{generale} ovvero una funzione che soddisfa la relazione con la
presenza di una costante (essa sarà nella forma $y=\varphi(x,c_1,c_2,\ldots,c_n)$) e 
\textbf{particolare} ovvero una funzione che soddisfa la relazione senza costante. Per questo secondo
tipo, sono necessarie ulteriori condizioni.\\
Se si riesce a scrivere un'equazione differenziale nella forma
\begin{equation*}
  y^{(n)}=G(x,y,y',\ldots,y^{(n-1)})
\end{equation*}
si dice che è scritta in \textbf{forma normale}.

\subsection{Equazioni differenziali a variabili separabili}
Se nell'equazione
\begin{equation*}
  y^{(n)}=G(x,y) 
\end{equation*}
è possibile scrivere $G(x,y)$ come $M(x)\cdot N(y)$, allora si definisce un'equazione differenziale a
variabili separabili. Per risolverla quindi si ha che
\begin{align*}
  y' &= M(x)\cdot N(y)\\
  \int \frac{y'}{N(y)}\,\dif x &= \int M(x)\,\dif x
\end{align*}

\subsection{Problema di Cauchy}
Dato il sistema
\begin{equation*}
  \begin{cases}
    y' = G(x,y)\\
    y(x_0) = y_0
  \end{cases}
\end{equation*}
Esiste ed è unica la soluzione. Ovviamente per un maggiore ordine, sono necessarie maggiori 
condizioni se si vogliono eliminare tutte le costanti.

\subsection{Equazioni differenziali lineari di primo ordine}
Sia data
\begin{equation*}
  y'+a(x)y=b(x)
\end{equation*}
Se $b(x)=0$ si definisce lineare \textbf{omogenea}. Per risolvere questo tipo di equazioni si deve
moltiplicare tutto per un fattore. Esso è
\begin{equation*}
  e^{\int a(x)\,\dif x}
\end{equation*}
Quindi l'equazione diventa
\begin{equation*}
  \underbrace{y'e^{\int a(x)\,\dif x}+a(x)e^{\int a(x)\,dif x}}_
  {\frac{\dif}{\dif x}\left[ ye^{\int a(x)\,\dif x} \right]}=b(x)e^{\int a(x)\,\dif x}
\end{equation*}
E quindi possiamo scrivere semplicemente
\begin{equation*}
  \int \frac{\dif}{\dif x}\left[ ye^{\int a(x)\,\dif x} \right]\,\dif x = 
  \int b(x)e^{\int a(x)\,\dif x}\,\dif x
\end{equation*}
e infine, isolando
\begin{equation*}
  y = e^{-\int a(x)\,\dif x}\int b(x)e^{\int a(x)\,\dif x}\,\dif x
\end{equation*}

\subsection{Equazioni differenziali lineari di secondo ordine}
Sia data
\begin{equation*}
  ay''+by'+cy=g(x) 
\end{equation*}
Se $g(x)=0$ si definisce lineare \textbf{omogenea}. Dato che $a$, $b$, $c$ sono coefficienti 
costanti, si può dividere tutto per $a$ e ottenere
\begin{equation*}
  y''+\frac{b}{a}y'+\frac{c}{a}y=0
\end{equation*}
che si riscrive in
\begin{equation*}
  y''+by'+c=0
\end{equation*}
Le soluziooni sono del tipo $y=e^{\lambda x}$ e quindi si ha che
\begin{equation*}
  y = e^{\lambda x}\quad y'=\lambda e^{\lambda x}\quad y''=\lambda^2 e^{\lambda x}
\end{equation*}
E quindi
\begin{equation*}
  \lambda^2 e^{\lambda x}+b\lambda e^{\lambda x}+ce^{\lambda x}=0
\end{equation*}
e raccogliendo
\begin{equation*}
  e^{\lambda x}(\lambda^2+b\lambda+c)=0
\end{equation*}
Questo accade solo se $\lambda^2+b\lambda+c=0$ visto che 
$\forall \lambda\in\mathbb{R}, e^{\lambda x}>\neq 0$. $\lambda^2+b\lambda+c=0$ si definisce
\textbf{equazione caratteristica}. Per risolvere queste equazioni differenziali, si distingue in base
al delta dell'equazione caratteristica.
\subsubsection{$\Delta > 0$}
Se $\Delta>0$ allora si avranno due soluzioni $e^{\lambda_1 x}\land e^{\lambda_2 x}$. La soluzione
generale quindi sarà $y=Ae^{\lambda_1 x}+Be^{\lambda_2 x}$ per qualche $A$ e $B$.
\subsubsection{$\Delta = 0$}
Se $\Delta=0$ allora si avranno due soluzioni coincidenti. Quindi la soluzione generale sarà nella 
forma $y=Ae^{\lambda x}+Bxe^{\lambda x}$ per qualche $A$ e $B$.
\subsubsection{$\Delta<0$}
Se $\Delta<0$ si avranno due soluzioni distinte, complesse coniugate 
$\lambda_{1/2}=\alpha+ i\beta$. Utilizzando la formula di Eulero per i numeri complessi, la 
soluzione generale diventa $y=Ae^{\alpha+i\beta}+Be^{\alpha+i\beta}$ che diventa, prendendo le parti
reali $y=Ae^{\alpha x}\cos\beta x +Be^{\alpha x}\sin\beta x$ per qualche $A$ e $B$.\\ [\baselineskip]
Se l'equazione non è omogenea, si avrà la soluzione generale come
\begin{equation*}
  y = y_{\text{omogenea}}+y_{\text{particolare}}
\end{equation*}
dove $y_\text{particolare}$ si ricava tramite tabelle.

%!TEX ROOT=formularioMatematica.tex

\section{Geometria analitica nello spazio}



%!TEX ROOT=formularioMatematica.tex

\section{Esercizi}
Questa sezione � dedicata ad alcuni esercizi con relativa risoluzione e spiegazione. Il suo scopo
� quello di chiarire i concetti teorici con esempi pratici.

\subsection*{\hyperref[sec:gen]{Generale}}\label{ex:generale}

\subsubsection*{\hyperref[subsec:gen:prodnot]{Prodotti notevoli}}
\paragraph{Esercizio 1}
Si scompongano il seguenti polinomi usando i prodotti notevoli.
\begin{equation}\label{eq:ex:prodnot1}
18x^3 - 4 -8x + 9x^2
\end{equation}
\begin{equation}\label{eq:ex:prodnot2}
a^2x^2 - a^2y^2 - b^2x^2 + b^2y^2
\end{equation}
% Reset the counter
\setcounter{equation}{0}
\divisor

Per semplificare \eqref{eq:ex:prodnot1} innanzitutto riscriviamo il polinomio in modo decrescente
\begin{equation*}
18x^3 + 9x^2- 8x - 4
\end{equation*}
Ora possiamo notare che i primi due elementi sono semplificabili, cos� come anche i secondi due per 
uno stesso fattore.
\begin{equation*}
\underbrace{18x^3 + 9x^2}_{9x^2(2x + 1)} \overbrace{- 8x - 4}^{-4(2x + 1)} = 
(9x^2 - 4)(2x + 1)
\end{equation*}
Ora abbiamo solo un altro prodotto da semplificare. Ricordando che $a^2-b^2 = (a-b)(a+b)$ possiamo
espandere la prima parentesi
\begin{equation*}
\boxed{(3x-2)(3x+2)(2x+1)}
\end{equation*}
Per semplificare \eqref{eq:ex:prodnot2} possiamo raccogliere i coefficienti di $x$ e $y$
\begin{equation*}
a^2x^2 - a^2y^2 - b^2x^2 + b^2y^2 = x^2(a^2-b^2) + y^2(b^2-a^2)
\end{equation*}
Ora per�, se si guardano attentamente i coefficienti, si vede che sono semplicemente opposti di segno,
quindi possiamo portare fuori il meno dal secondo e renderli uguali
\begin{align*}
x^2(a^2-b^2) + y^2(b^2-a^2) &= x^2(a^2-b^2) -y^2(a^2-b^2) =\\ &(x^2 - y^2)(a^2-b^2)
\end{align*}
Ricordando che $a^2-b^2 = (a-b)(a+b)$ possiamo espandere e concludere
\begin{equation*}
(x^2 - y^2)(a^2-b^2) = \boxed{(x-y)(x+y)(a-b)(a+b)}
\end{equation*}

\subsection*{\hyperref[sec:geomanal]{Geometria Analitica}}\label{ex:geomanal}
\subsubsection*{\hyperref[subsec:geomanal:retta]{Rette}}
\paragraph{Esercizio 1}
Dato il triangolo di vertici $A(-2,3)$, $B(-2,-1)$ e $C(3,4)$, determinare:
\begin{enumerate}
	\item le equazioni dei lati; \label{enum:ex:retta:1:1}
	\item il perimetro e l'area del triangolo \label{enum:ex:retta:1:2}
	\item detta $t$ la retta passante per $C$ e perpendicolare alla retta $BC$ e detto $D$ il punto
	d'intersezione di $t$ con l'asse $x$, l'area del quadrilatero ACDB; \label{enum:ex:retta:1:3}
	\item i punti della retta $y = 2x$ che hanno distanza uguale a $3$ dalla retta AB.
	\label{enum:ex:retta:1:4}
\end{enumerate}
\divisor

Come in ogni esercizio di geometria, partiamo dal disegno. Lo miglioreremo man mano che andiamo 
avanti.
\begin{center}
	\begin{tikzpicture}[scale=0.75]
		\coordinate (A) at (-2,3);
		\coordinate (B) at (-2,-1);
		\coordinate (C) at (3,4);
		
		\tkzInit[xmin=-5,ymin=-5,xmax=5,ymax=5]
		\tkzGrid
		\tkzAxeXY
		
		% Triangle
		\draw[red, thick] (A) -- (B) -- (C) -- cycle;
		\filldraw (A) circle (0.05)
			node[above left]{$A$};
		\filldraw (B) circle (0.05)
			node[below left]{$B$};
		\filldraw (C) circle (0.05)
			node[above right]{$C$};
	\end{tikzpicture}
\end{center}

Per i primi due punti, questo � tutto quello che ci serve.\\
\textbf{Per il punto \ref{enum:ex:retta:1:1}}, possiamo semplicemente usare la formula per la retta 
passante per due punti. Per convenienza, denominiamo le rette in base ai vertici che attraversano.\\
Per la retta $AB$ � immediato: si nota che hanno la stessa ascissa, quindi la retta passante per i due
punti � solo $\boxed{AB: x = -2}$.\\\\
Per $AC$:
\begin{align*}
\frac{y-y_1}{y_2-y_1} = \frac{x-x_1}{x_2-x_1} &\rightarrow
\frac{y-3}{4-3} = \frac{x-(-2)}{3-(-2)} \rightarrow\\
\frac{y-3}{1} = \frac{x+2}{5} &\rightarrow y = \frac{x+2}{5} + 3\\
y = \frac{1}{5}x + \frac{2}{5} + \frac{15}{5} &\rightarrow \boxed{AC: y = \frac{1}{5}x + \frac{17}{5}}
\end{align*}
Infine per $BC$
\begin{align*}
\frac{y-y_1}{y_2-y_1} = \frac{x-x_1}{x_2-x_1} &\rightarrow
\frac{y+1}{4+1} = \frac{x-(-2)}{3+2}\\
\frac{y+1}{5} = \frac{x+2}{5} &\rightarrow \boxed{BC: y = x + 1}
\end{align*}

\textbf{Ci avviamo ora al punto \ref{enum:ex:retta:1:2}} e per l'area possiamo usare la matrice
\begin{equation*}
\mathscr{A}(ABC) = \frac{1}{2}\left\lvert 
\begin{matrix}[1]
x_1 & y_1 & 1\\
x_2 & y_2 & 1\\
x_3 & y_3 & 1
\end{matrix}\right\rvert
\end{equation*}
Che poi si semplifica usando Sarrus in
\begin{equation*}
\mathscr{A}(ABC) = \frac{1}{2}\left\lvert x_1y_2 + y_1x_3 + x_2y_3 -x_3y_2 -y_3x_1 -x_2y_1\right\rvert
\end{equation*}
E sostituendo otteniamo
\begin{align*}
\mathscr{A}(ABC) &= \frac{1}{2}\left\lvert -2\cdot1 + 3\cdot3 -2\cdot4 -3\cdot1 - 4\cdot(-2) +
2\cdot3\right\rvert\\
\mathscr{A}(ABC) &= \frac{1}{2}\left\lvert -2+9-8-3+8+6\right\rvert\\
\Aboxed{\mathscr{A}(ABC) &= 10}
\end{align*}
Per trovare il perimetro, possiamo usare la distanza tra due punti e trovare tutte le lunghezze.\\
$AB$ � immediato in quanto hanno la stessa ascissa. $\boxed{AB = 3 + 1 = 4}$.\\\\
Per trovare $AC$
\begin{align*}
AC &= \sqrt{(x_C-x_A)^2+(y_C-y_A)^2} \rightarrow \\
AB &= \sqrt{(3+2)^2 + (4-3)^2} = \sqrt{5^2+1^2} \\
 &= \sqrt{25+1} = \boxed{\sqrt{26}}
\end{align*}
Per trovare $BC$
\begin{align*}
BC &= \sqrt{(x_C-x_B)^2+(y_C-y_B)^2} \rightarrow \\
BC &= \sqrt{(3+2)^2+(4+1)^2} = \sqrt{5^2+5^2} =
\sqrt{25+25} \\
&= \sqrt{50} = \sqrt{5^2\cdot2} = \boxed{5\sqrt{2}}
\end{align*}
E ora non resta che sommare
\begin{equation*}
2p = AB + AC + BC = 4 + \sqrt{26} + 5\sqrt{2}
\end{equation*}

\textbf{Per il punto \ref{enum:ex:retta:1:3}} aggiorniamo il disegno
\begin{center}
	\begin{tikzpicture}[scale=0.75]
	\coordinate (A) at (-2,3);
	\coordinate (B) at (-2,-1);
	\coordinate (C) at (3,4);
	\coordinate (D) at (0,7);
	
	
	%\draw[step=1,very thin,darkgray] (-5,-3) grid (5,8);
	%\draw[thick,-stealth] (-5,0) -- (5,0)
	%	node[pos=1,above]{$x$};
	%\draw[thick,-stealth] (0,-3) -- (0,8)
	%	node[pos=1,right]{$y$};
	\tkzInit[xmax=5,ymax=8,xmin=-5,ymin=-3]
	\tkzGrid
	\tkzAxeXY
	
	\filldraw[red, fill opacity = 0.1](A) -- (B) -- (C) -- cycle;
	\filldraw[blue, fill opacity = 0.3] (A) -- (B) -- (C) -- (D) -- cycle;
	
	% Triangle
	\filldraw (A) circle (0.05)
		node[above left]{$A$};
	\filldraw (B) circle (0.05)
		node[below left]{$B$};
	\filldraw (C) circle (0.05)
		node[above right]{$C$};
	\filldraw (D) circle (0.05)
		node[above right]{$D$};
	
	\draw[dashed, blue, thick] (5,2) -- (-1,8)
		node[pos=0.25, above]{$t$};
	\end{tikzpicture}
\end{center}
Noi dobbiamo calcolare l'area di $ABCD$. Abbiamo varie strade che possiamo seguire. Ne propongo una
che pu� essere usata per praticamente ogni figura. Il tutto si basa su trovare l'area del rettangolo 
che contiene la figura e togliere dei triangoli che possiamo individuare. Nel nostro caso

\begin{center}
	\begin{tikzpicture}[scale=0.75]
	\coordinate (A) at (-2,3);
	\coordinate (B) at (-2,-1);
	\coordinate (C) at (3,4);
	\coordinate (D) at (0,7);
	
	\tkzInit[xmin=-5,ymin=-3,xmax=5,ymax=8]
	\tkzGrid
	\tkzAxeXY
	
	%\filldraw[red, fill opacity = 0.1](A) -- (B) -- (C) -- cycle;
	\filldraw[blue, fill opacity = 0.2] (A) -- (B) -- (C) -- (D) -- cycle;
	\filldraw[magenta, fill opacity = 0.3] (B) -- (C) -- (3,-1) -- cycle;
	\filldraw[orange, fill opacity = 0.3] (A) -- (D) -- (-2,7) -- cycle;
	\filldraw[cyan, fill opacity = 0.3] (D) -- (C) -- (3,7) -- cycle;
	
	% Triangle
	\filldraw (A) circle (0.05)
		node[above left]{$A$};
	\filldraw (B) circle (0.05)
		node[below left]{$B$};
	\filldraw (C) circle (0.05)
		node[above right]{$C$};
	\filldraw (D) circle (0.05)
		node[above right]{$D$};
	\filldraw (3,-1)  circle (0.05)
		node[below right]{$X$};
	\filldraw (-2,7)  circle (0.05)
		node[above left]{$Y$};
	\filldraw (3,7) circle (0.05)
		node[above right]{$Z$};
	
	\draw[dashed, blue, thick] (5,2) -- (-1,8)
		node[pos=0.25, above]{$t$};
	\node[magenta] (F1) at (2,1){$F1$};
	\node[red] (F2) at (-1.5,6.5){$F2$};
	\node[blue] (F3) at (2,6){$F3$};
	\end{tikzpicture}
\end{center}

vediamo che possiamo trovare l'area facendo
\begin{equation*}
\mathscr{A}(YZXB) - \mathscr{A}(F1) - \mathscr{A}(F2) - \mathscr{A}(F3)
\end{equation*}
o pi� semplicemente, sostituendo
\begin{align*}
\mathscr{A}(ABCD) &= BX\cdot BY - \overbrace{\frac{5^2}{2}}^{\mathscr{A}(\mathcolor{magenta}{F1})} - 
\overbrace{\frac{2\cdot4}{2}}^{\mathscr{A}(\mathcolor{orange}{F2})} -
\overbrace{\frac{3^2}{2}}^{\mathscr{A}(\mathcolor{blue}{F3})} \\
&= 5\cdot8 - 12.5 - 4 -4.5 = 40 - 21 = 19
\end{align*}

Ora per \textbf{l'ultimo punto} possiamo semplificare il disegno e pulirlo un po'.
\begin{center}
	\begin{tikzpicture}[scale=0.75]
	
	\tkzInit[xmin=-5,ymin=-11,xmax=5,ymax=3]
	\tkzGrid
	\tkzAxeXY

	\draw[red, thick] (-2,-11) -- (-2,3);
	\draw[blue, thick] (-5,-10) -- (1.5,3)
		node[pos=0.75, right]{$y = 2x$};
	\filldraw (-5,-10) circle (0.05);
	\filldraw (1,2) circle (0.05);
	
	\draw[dashed, orange, very thick] (1,2) -- ++(-3,0);
	\draw[dashed, orange, very thick] (-5,-10) -- ++(3,0);
	\end{tikzpicture}
\end{center}

Per prima cosa dobbiamo trasformare in forma esplicita la retta $x = -2$ per poter usare la formula
della distanza Punto-Retta.
\begin{equation*}
r: x + 2 = 0
\end{equation*}
E ora possiamo scrivere la formula della distanza
\begin{align*}
d &= \frac{\left\lvert ax_P + by_P+c\right\rvert}{\sqrt{a^2+b^2}} \rightarrow
3 = \frac{\left\lvert x + 2\right\rvert}{\sqrt{1^2}} \\
3 &= \frac{\left\lvert x + 2\right\rvert}{\sqrt{1}} \rightarrow
3\cdot1 = \left\lvert x + 2\right\rvert\\
\pm3 &= x + 2 \rightarrow \begin{dcases}
x + 5 = 0\\
x - 1 = 0
\end{dcases} \rightarrow
\begin{dcases}
x = -5\\
x = 1
\end{dcases}
\end{align*}
Abbiamo le ascisse di intersezione con la retta $y = 2x$. Ora possiamo sostituire e trovare $y$.
\begin{equation*}
\begin{dcases}
y = -5\cdot2\\
y = 1\cdot2
\end{dcases} \rightarrow
\boxed{\begin{dcases}
P_1(-5,-10)\\
P_2(1,2)
\end{dcases}}
\end{equation*}

\subsubsection*{\hyperref[subsec:geomanal:fasciorette]{Fasci di rette}}
\paragraph{Esercizio 1}
Dopo aver verificato che l'equazione
\begin{equation*}
(2k+1)x -4ky + 3 + 2k = 0 \qquad (k\in\mathbb{R})
\end{equation*}
rappresenta un fascio proprio di rette, determinare:
\begin{enumerate}
	\item il centro $C$ del fascio; \label{enum:ex:retta:2:1}
	\item la retta $r_1$ del fascio perpendicolare alla bisettrice del $\ang{2}$ e $\ang{3}$ quadrante;
	detto $H$ il loro punto di incontro, trovare poi l'area del triangolo $CHO$, essendo $O$ l'orgine
	degli assi;\label{enum:ex:retta:2:2}
	\item le rette del fascio che intersecano il segmento $OH$;\label{enum:ex:retta:2:3}
	\item le bisettrici degli angoli formati dalle rette $CO$ e $CH$.\label{enum:ex:retta:2:4}
\end{enumerate}
\divisor

Prima di avere il disegno, dobbiamo avere qualcosa da disegnare. Se disegnassimo l'intero fascio 
sarebbe come colorare tutto il piano.\\
\textbf{Per il punto \ref{enum:ex:retta:2:1}} dobbiamo mettere a sistema le due rette generatrici.
Nella forma attuale, le due equazioni non sono facilmente riconoscibili, quindi raccogliamo $k$ cos�
da isolare le due rette
\begin{align*}
(2k+1)x -4ky + 3 + 2k = 0 &\rightarrow 2kx+x-4ky+3+2k = 0 \rightarrow\\
k\underbrace{(2x-4y+2)}_{\text{Generatrice 1}}&+\underbrace{x+3}_{\text{Generatrice 2}} = 0
\end{align*}
Avendo ora questa forma, possiamo evidentemente vedere che effettivamente si tratta di un fascio 
proprio di rette.\\
Come trovare il centro del fascio? Avendo le due generatrici, le mettiamo a sistema e troviamo la loro
intersezione
\begin{align*}
\begin{dcases}
2x-4y+2=0\\
x=-3
\end{dcases} &\rightarrow
\begin{dcases}
-x-4y+2=0\\
x=-3
\end{dcases} \rightarrow\\
\begin{dcases}
\cancel{-4}y = \cancelto{-1}{4}\\
x=-3
\end{dcases} &\rightarrow
\boxed{\begin{dcases}
y = -1\\
x = -3
\end{dcases}}
\end{align*}

\textbf{Per il punto \ref{enum:ex:retta:2:2}} facciamo il disegno
\begin{center}
	\begin{tikzpicture}
		\coordinate (C) at (-3,-1);
		\coordinate (H) at (-1,1);
		
		\tkzInit[xmin=-4,ymin=-2,xmax=3,ymax=3]
		\tkzGrid
		\tkzAxeXY
		
		\filldraw[orange, fill opacity = 0.3] (C) -- (H) -- (0,0) -- cycle;
		
		\draw[blue] (-3,3) -- (2,-2)
			node[pos=0.25, left]{$y = -x$};
		\draw[red] (-4,-2) -- (1,3)
			node[pos=0.75, right]{$y = x+2$};
			
		\filldraw (C) circle (0.05)
			node[below]{$C$};
		\filldraw (H) circle (0.05)
			node[above]{$H$};
	\end{tikzpicture}
\end{center}
Ho gi� inserito le cose che ora andiamo a trovare.\\
Innanzitutto sappiamo che la bisettrice del $\ang{2}$ e $\ang{3}$ quadrante � $y=-x$, quindi sappiamo
che la $m$ della perpendicolare deve essere uguale a $1$. Sappiamo anche che fa parte del fascio 
quindi passa per $C(-3,-1)$.
\begin{equation*}
y-y_0 = m(x-x_0) \rightarrow y+1 = x+3 \rightarrow \boxed{y = x+2}
\end{equation*}
E ora ci troviamo $H$, ovvero il punto di intersezione
\begin{equation*}
\begin{dcases}
y = x+2\\
y = -x
\end{dcases} \rightarrow
\begin{dcases}
-x = x+2\\
y = -x
\end{dcases} \rightarrow
\boxed{\begin{dcases}
x = -1\\
y = 1
\end{dcases}}
\end{equation*}
Ora possiamo trovare l'area del triangolo
\begin{align*}
\mathscr{A}(CHO) &= \frac{1}{2}\left\lvert x_1y_2 + y_1x_3 + x_2y_3 -x_3y_2 -y_3x_1-x_2y_1\right\rvert
\rightarrow\\
\mathscr{A}(CHO) &= \frac{1}{2}\left\lvert x_1y_2 + \cancel{y_1x_3} + \cancel{x_2y_3} 
-\cancel{x_3y_2} -\cancel{y_3x_1} -x_2y_1\right\rvert \rightarrow \\
\mathscr{A}(CHO) &= \frac{1}{2}\left\lvert x_1y_2 -x_2y_1\right\rvert \rightarrow
\mathscr{A}(CHO) = \frac{1}{2}\left\lvert -3 -1\right\rvert =\\
\frac{1}{2}&\left\lvert-4\right\rvert \rightarrow \mathscr{A}(CHO) = \frac{1}{2}\cdot4 = \boxed{2}
\end{align*}

\textbf{Il punto \ref{enum:ex:retta:2:3}}, richiede di trovare i $k$ per cui una retta del fascio 
passi in mezzo al segmento $OH$. La prima cosa da fare � quindi trovare i $k$ degli "estremi" $O$ e
$H$.
\begin{equation*}
k_O = -\frac{a_1x_O+b_1y_O+c_1}{ax_O+by_O+c} \rightarrow k_O = -\frac{c_1}{c} = -\frac{3}{2}
\end{equation*}
\begin{equation*}
k_H = -\frac{a_1x_H+b_1y_H+c_1}{ax_H+by_H+c} \rightarrow k_H = -\frac{-1+0+3}{-2-4+2} = \frac{1}{2}
\end{equation*}

Ora sapendo che la retta esclusa attraversa anch'essa il segmento (per dimostrarlo basta 
semplicemente disegnarla), deduciamo che ai lati della esclusa ci siano le rette per $k\to\pm\infty$,
ovvero man mano che ci si avvicina alla retta esclusa pi� ci si avvicina all'infinito. Questo ci porta
a trovare l'intervallo che �
\begin{equation*}
\boxed{k\leq-\frac{3}{2} \vee k\geq\frac{1}{2}}
\end{equation*}

Infine, il \textbf{punto \ref{enum:ex:retta:2:4}} richiede un po' di ragionamento. Una bisettrice � la
retta passante per due punti equidistanti alle rette dell'angolo. Per prima cosa quindi, definiamo
$P(x,y)$ un punto del piano in modo che sia $d_{P,CO} = d_{P,CH}$. Per prima cosa dunque dobbiamo 
trovare le rette che passano per $CO$ e $CH$.
\begin{align*}
\frac{y-y_1}{y_2-y_1} &= \frac{x-x_1}{x_2-x_1} \rightarrow
\frac{y+1}{1+1} = \frac{x+3}{-1+3} \rightarrow\\
y+1 &= x+3 \rightarrow CO:\, x-y+2 = 0
\end{align*}
\begin{align*}
\frac{y-y_1}{y_2-y_1} &= \frac{x-x_1}{x_2-x_1} \rightarrow
\frac{y+1}{1}= \frac{x+3}{3} \rightarrow\\
3y+3 &= x+3 \rightarrow CH:\, -x+3y=0
\end{align*}
E ora possiamo scrivere le formule per le distanze
\begin{align*}
\frac{\lvert x-y+2\rvert}{1} = \frac{\lvert -x+3y\rvert}{\sqrt{10}} &\rightarrow\\
\sqrt{10}(\lvert x-y+2\rvert) = \lvert -x+3y\rvert &\rightarrow\\
\sqrt{10}x-\sqrt{10}y+2\sqrt{10} = \pm(-x+3y) &\rightarrow\\
\begin{cases}
\sqrt{10}x-\sqrt{10}y+2\sqrt{10} = -x+3y\\
\sqrt{10}x-\sqrt{10}y+2\sqrt{10} = x-3y
\end{cases} &\rightarrow\\
\begin{cases}
x(\sqrt{10}+1)-y(\sqrt{10}+3) + 2\sqrt{10}\\
x(\sqrt{10}-1)-y(\sqrt{10}-3) + 2\sqrt{10}
\end{cases} &\rightarrow\\
\boxed{x(\sqrt{10}\pm1)-y(\sqrt{10}\pm3)+2\sqrt{10}}
\end{align*}

\subsubsection*{\hyperref[subsec:geomana:circ]{Circonferanza}}\label{ex:circ}
\paragraph{Esercizio 1}
Determinare l'equazione della circonferenza passante per $A(-2,2)$ e $B(4,-4)$ e avente il centro
sulla retta $x+2y-8=0$, e le equazioni delle rette $t_1$ e $t_2$ passanti per $H(0,8)$ e tangenti 
alla circonferenza. detta poi $t_1$ la tangente con coefficiente angolare positivo, determinare le 
rette ad essa perpendicolari che formano con gli assi cartesiani un triangolo di area $\dfrac{54}{5}$.
Determinare, inoltre, i punti di $t_1$ che hanno distanza uguale a $\sqrt{2}$ dalla retta $x+y-1=0$.
\divisor

Per prima cosa dobbiamo trovare l'equazione della circonferenza $\mathscr{C}$. Come fare? Sappiamo che
$A$ e $B$ appartengono alla circonferenza e che il centro appartiene a $x+2y-8=0$. Mettiamo queste
informazioni a sistema e risolviamo
\begin{align*}
&\begin{dcases}
4+4-2a+2b+c=0\\
16+16+4a-4b+c=0\\
-\frac{a}{2}-b-8=0
\end{dcases}\rightarrow\\
&\begin{dcases}
-8+2a-2b=c\\
32+4a-4b-8+2a-2b=0\\
-\frac{a}{2}-b-8=0
\end{dcases}\rightarrow\\
&\begin{dcases}
-8+2a-2b=c\\
\cancelto{4}{24}+\cancelto{1}{6}a-\cancelto{1}{6}b=0\\
-\frac{a}{2}-b-8=0
\end{dcases}\rightarrow
\begin{dcases}
-8+2a-2b=c\\
4+a+\frac{a}{2}+8=0\\
b=-\frac{a}{2}-8
\end{dcases}\rightarrow\\
&\begin{dcases}
-8+2a-2b=c\\
\frac{\cancel{3}}{\cancel{2}}a = -\cancelto{8}{12}\\
b = -\frac{a}{2}-8
\end{dcases}\rightarrow
\begin{dcases}
\cancel{-8}-16\cancel{+8}=c\\
a = -8\\
b = -\frac{8}{2}-8
\end{dcases}\rightarrow\\
&\begin{dcases}
c = -16\\
a = -8\\
b = -4
\end{dcases} \rightarrow \boxed{\mathscr{C}:\,x^2+y^2-8x-4y-16=0}
\end{align*}
Avendo ora l'equazione possiamo disegnarla.
\begin{center}
	\begin{tikzpicture}[scale=0.5]
		\coordinate (C) at (4,2);
		\coordinate (A) at (-2,2);
		\coordinate (B) at (4,-4);
		\coordinate (H) at (0,8);
		\def\R{6};
		
		\tkzInit[xmin=-5,ymin=-5,xmax=11,ymax=9]
		\tkzGrid
		\tkzAxeXY
		
		\draw (C) circle(\R);
		\draw[orange, thick] (-5,8) -- (11,8)
			node[pos=0.25,above]{$t_2$};
		\draw[red,thick] (-5,-4) -- (0.41,9)
			node[pos=0.25,above left]{$t_1$};
		
		\filldraw (C) circle (0.1)
			node[above]{$C(4,2)$};
		\filldraw (A) circle (0.1)
			node[above]{$A(-2,2)$};
		\filldraw (B) circle (0.1)
			node[above]{$B(4,-4)$};
		\filldraw[magenta] (H) circle (0.1)
			node[below right]{$H(0,8)$};
	\end{tikzpicture}
\end{center}
Per trovare le due tangenti alla circonferenza che passano per $H$, ci troviamo il fascio di rette 
che ha $H$ come centro
\begin{equation*}
y-y_0=m(x-x_0) \rightarrow y-mx-8=0
\end{equation*}
e sappiamo che le tangenti hanno il loro punto di tangenza che dista dal centro esattamente $r$, 
quindi
\begin{align*}
\frac{\lvert ax_P + y_P + c_P\rvert}{\sqrt{a^2+b^2}} = d &\rightarrow
\frac{\lvert-4m+2-8\rvert}{\sqrt{1^2+m^2}} = 6 \rightarrow\\
\lvert-4m+2-8\rvert^2 &= (6\sqrt{1^2+m^2}) \rightarrow\\
16m^2\cancel{+36}+48m&=\cancel{36} + 36m^2 \rightarrow\\
\cancelto{-5}{-20}m^2+\cancelto{12}{48}m = 0 &\rightarrow 5m^2-12m =0 \\
m_{1/2} = \frac{12\pm\sqrt{144+0}}{10} &\rightarrow \boxed{\begin{dcases}
m_1 = \frac{12}{5}\\
m_2 = 0
\end{dcases}}
\end{align*}
e le tangenti sono
\begin{equation*}
\boxed{t_1:\,y = \frac{12}{5}x+8 \qquad t_2:\, y=8}
\end{equation*}

Ora dobbiamo trovare tutte le perpendicolari a $t_1$ che, con l'interesezione degli assi forma un
triangolo di area $\dfrac{54}{5}$. Per farlo, intanto troviamo le perpendicolari.
\begin{equation*}
\mathscr{F_\perp}:\,y=-\frac{5}{12}x + q
\end{equation*}
E ora possiamo trovare le intersezioni con gli assi
\begin{equation*}
\begin{dcases}
x = 0\\
y = q
\end{dcases}\qquad
\begin{dcases}
x = -\frac{12}{5}q\\
y = 0
\end{dcases}
\end{equation*}

E ora imponiamo che l'area del triangolo formato con gli assi sia uguale a $\dfrac{54}{5}$
\begin{equation*}
\frac{54}{5} = \frac{1}{2}\left\lvert q\cdot-\frac{12}{5}q\right\rvert \rightarrow
\frac{54}{5} = \frac{6}{5}\lvert q^2\rvert \rightarrow 9 = q^2 \rightarrow q = \pm3
\end{equation*}
quindi le rette cercate sono
\begin{equation*}
\boxed{y = -\frac{5}{12}x\pm3}
\end{equation*}

Finalmente possiamo avviarci alla conclusione. Dobbiamo cercare i punti di $t_1$ che distano 
$\sqrt{2}$ da $x+y-1$. Per prima cosa quindi, troviamo le rette che distano $\sqrt{2}$ dalla data
\begin{align*}
\frac{\lvert x+y-1\rvert}{\sqrt{2}} = \sqrt{2} \rightarrow \lvert x+y-1\rvert = 2 \rightarrow\\
\begin{cases}
x+y-1=2\\
x+y-1=-2
\end{cases} \rightarrow
\begin{cases}
x+y-3=0\\
x+y+1=0
\end{cases}
\end{align*}
e ora non resta che trovare le intersezioni con $t_1$
\begin{equation*}
\begin{dcases}
x+y+1=0\\
y=\frac{12}{5}x+8
\end{dcases}\rightarrow
\begin{dcases}
x+\frac{12}{5}x+8+1=0\\
y=\frac{12}{5}x+8
\end{dcases}\rightarrow
\boxed{\begin{dcases}
x = -\frac{45}{17}\\
y = \frac{76}{17}
\end{dcases}}
\end{equation*}
\begin{equation*}
\begin{dcases}
x+y-3=0\\
y=\frac{12}{5}x+8
\end{dcases}\rightarrow
\begin{dcases}
x+\frac{12}{5}x+5=0\\
y=\frac{12}{5}x+8
\end{dcases}\rightarrow
\boxed{\begin{dcases}
x = -\frac{25}{17}\\
y = \frac{76}{17}
\end{dcases}}
\end{equation*}

\subsubsection*{\hyperref[subsec:geomanal:fasciocirc]{Fasci di circonferenze}}
\paragraph{Esecizio 1}
Avendo il fascio
\begin{equation*}
x^2+y^2-2(k+1)x-2ky-4k+1=0
\end{equation*}
indicare con $\gamma_1$ quella il cui centro $C$ appartiene
alla retta $3x-y+5=0$. Detti $E$ ed $F$ i punti di intersezione di $\gamma_1$ con l'asse $y$, trovare
le equazioni delle tangenti a $\gamma_1$ in $E$ ed $F$; detto inoltre $T$ il loro punto di 
intersezione, dopo aver dimostrato che il quadrilatero $CETF$ � un quadrato, calcolarne l'area. 
Determinare inoltre l'equazione della circonfereza d centro $T$ e tangente esternamente a $\gamma_1$.
\divisor

Per prima cosa riordiniamo l'equazione per avere tutti i coefficienti
\begin{align*}
x^2+y^2-2(k+1)x-2ky-4k+1=0 \rightarrow\\
x^2+y^2+x(-2k-2)+y(-2k)+4k+1=0
\end{align*}
avendo ora i coefficienti, possiamo imporre la condizione che il centro sia un punto della retta
\begin{align*}
&3\left(-\frac{a}{2}\right)-\left(-\frac{b}{2}\right)-5=0 \rightarrow
3\frac{2k+2}{2}-k-5=0 \rightarrow\\ &6k+6-2k-10=0\rightarrow k=1
\end{align*}
e sostituire per ottenere
\begin{equation*}
\boxed{\gamma_1:\, x^2+y^2-4x-2y-3=0}
\end{equation*}
Prima di proseguire, disegnamo la circonferenza
\begin{center}
	\begin{tikzpicture}[scale=0.75]
		\coordinate (C) at (2,1);
		\coordinate (E) at (0,3);
		\coordinate (F) at (0,-1);
		\coordinate (T) at (-2,1);
		\def\R{2.8284};
		
		\tkzInit[xmin=-5,ymin=-5,xmax=5,ymax=5]
		\tkzGrid
		\tkzAxeXY
		
		\draw (C) circle(\R);
		\draw[thick, red, domain=-5:2] plot(\x,\x+3);
		\draw[thick,red, domain=-5:4] plot(\x,-\x-1);
		
		\filldraw (C) circle (0.05)
			node[below]{$C(2,1)$};
		\filldraw[red] (E) circle (0.05)
			node[left]{$E(0,3)$};
		\filldraw[red] (F) circle (0.05)
			node[left]{$F(0,-1)$};
		\filldraw[blue] (T) circle (0.05)
			node[left]{$T(-2,1)$};
	\end{tikzpicture}
\end{center}
Sono gi� segnati i punti che ora andremo a trovare: $E$ e $F$ ovvero le intersezioni con $y$.
\begin{equation*}
y^2-2y-3=0\rightarrow
y_{1/2} = \frac{2\pm\sqrt{4+12}}{2} = \frac{2\pm4}{2} \rightarrow \begin{cases}
y_1 = 3\\
y_2 = -1
\end{cases}
\end{equation*}
Quindi i due punti sono
\begin{equation*}
\boxed{E(0,3)\qquad F(0,-1)}
\end{equation*}

Ora troviamo le tangenti in $E$ ed $F$.
\begin{align*}
&x\cdot x_P+y\cdot y_P+a\frac{x+x_P}{2}+b\frac{y+y_P}{2}+c = 0\rightarrow\\
&t_E:\, x\cdot0+y\cdot3-4\frac{x+0}{2}-2\frac{y+3}{2}-3=0 \rightarrow\\
&\boxed{-x+y-3=0 \rightarrow y = x+3}
\end{align*}
e
\begin{align*}
&x\cdot x_P+y\cdot y_P+a\frac{x+x_P}{2}+b\frac{y+y_P}{2}+c = 0\rightarrow\\
&t_F:\, x\cdot0+y\cdot(-1)-4\frac{x+0}{2}-2\frac{y-1}{2}-3=0\rightarrow\\
&\boxed{-x-y-1=0\rightarrow y=-x-1}
\end{align*}
E ora possiamo trovare il punto di intersezione
\begin{align*}
\begin{cases}
y=x+3\\
y=-x-1
\end{cases}\rightarrow
\begin{cases}
-x-1=x+3\\
y=-x-1
\end{cases}\rightarrow
\begin{cases}
x = -2\\
y=1
\end{cases}
\end{align*}
Aggiorniamo ora il disegno per mettere in luce il quadrato $CETF$
\begin{center}
	\begin{tikzpicture}[scale=0.75]
	\coordinate (C) at (2,1);
	\coordinate (E) at (0,3);
	\coordinate (F) at (0,-1);
	\coordinate (T) at (-2,1);
	\def\R{2.8284};
	
	\tkzInit[xmin=-5,ymin=-5,xmax=5,ymax=5]
	\tkzGrid
	\tkzAxeXY
	
	\filldraw[orange, fill opacity=0.3] (C) -- (E) -- (T) -- (F) -- cycle;
		
	\draw (C) circle(\R);
	\draw[thick, red, domain=-5:2] plot(\x,\x+3);
	\draw[thick,red, domain=-5:4] plot(\x,-\x-1);
	
	\filldraw (C) circle (0.05)
	node[below]{$C(2,1)$};
	\filldraw[red] (E) circle (0.05)
	node[left]{$E(0,3)$};
	\filldraw[red] (F) circle (0.05)
	node[left]{$F(0,-1)$};
	\filldraw[blue] (T) circle (0.05)
	node[left]{$T(-2,1)$};
	\end{tikzpicture}
\end{center}

Come possiamo dimostrare che � un quadrato? Proviamo a guardare gli angoli: l'angolo $C\widehat{F}T$
e l'angolo $C\widehat{E}T$ sono sicuramente retti in quanto sono angoli formati da un raggio e una 
tangente e per definizione stessa di tangente sono retti. Anche l'angolo $F\widehat{T}E$ � retto in
quanto i coefficienti angolari delle tangenti sono reciprocamente opposti ($m_1m_2=-1$). Ora manca
solo l'angolo $E\widehat{C}F$ da dimostrare. Possiamo semplicemente guardare il coefficiente angolare
della retta che passa tra $F$ e $C$ e vedere che risulta pari a 
\begin{equation*}
m = \frac{y_2-y_1}{x_2-x_1} \rightarrow m= \frac{1+1}{2-0} = 1
\end{equation*}
che � esattamente uguale a quello di $t_E$ quindi le due rette sono parallele. Se $t_F$ incide su 
$t_E$ con un angolo retto, deve per forza incidere con lo stesso angolo anche nelle sue parallele.\\
Abbiamo dimostrato che ha quattro angoli retti, per dimostrare che � un quadrato basta vedere che due
dei lati (che formano un angolo retto) sono uguali in quanto sono raggi. Quindi $CETF$ � un 
quadrato.\\
Per trovarne l'area basta elevare alla seconda la lunghezza del raggio
\begin{equation*}
r = \sqrt{x_C^2+y_C^2-c} \rightarrow r = \sqrt{8} = 2\sqrt{2}
\end{equation*}
E quindi l'area vale
\begin{equation*}
\mathscr{A}(CETF) = r^2 \rightarrow \mathscr{A}(CETF) = \sqrt{8}^2 = 8
\end{equation*}

Infine dobbiamo trovare la circonferenza con centro $T$ e tangente esternamente a $\gamma_1$. Per 
farlo abbiamo molti modi, ecco il pi� semplice. Sappiamo gi� quanto deve valere il raggio perch� 
tocchi la circonferenza. Deve essere pari a $TC-r_{\gamma_1}$. Quindi
\begin{equation*}
r = TC-r_{\gamma_1} \rightarrow r = 4-2\sqrt{2}
\end{equation*}

Usando la formula per trovare il raggio possiamo scrivere
\begin{equation*}
r = \sqrt{\frac{a^2}{4}+\frac{b^2}{4}-c} \rightarrow 
4-2\sqrt{2} = \sqrt{\frac{a^2}{4}+\frac{b^2}{4}-c}
\end{equation*}
Abbiamo 3 variaibli quindi dobbiamo trovare un modo per toglierne 2. $a$ e $b$ sono utilizzate anche 
nella formula per trovare il centro della circonferenza. Si da il caso che noi abbiamo il centro!
Quindi
\begin{equation*}
\begin{dcases}
-\frac{a}{2} = -2\\
-\frac{b}{2} = 1
\end{dcases}\rightarrow
\begin{dcases}
a=4\\
b=-2
\end{dcases}
\end{equation*}
e ora possiamo trovare $c$
\begin{align*}
4-2\sqrt{2} = \sqrt{\frac{a^2}{4}+\frac{b^2}{4}-c} &\rightarrow 
4-2\sqrt{2} = \sqrt{\frac{16}{4}+\frac{4}{4}-c} \rightarrow\\
4-2\sqrt{2} = \sqrt{5-c} &\rightarrow
24-16\sqrt{2} = 5-c \rightarrow\\
c &= 16\sqrt{2}-19
\end{align*}
Quindi la nostra circonferenza sar�
\begin{equation*}
\boxed{\gamma:\, x^2+y^2+4x-2y+16\sqrt{2}-19=0}
\end{equation*}

\subsubsection*{\hyperref[subsec:geomanal:parabola]{Parabola}}\label{ex:parabola}
\paragraph{Esercizio 1}
Nel piano $xOy$ determinare
\begin{enumerate}
	\item l'equazione della parabola $\mathscr{P}_1$ avente asse parallelo all'asse $y$ e passante per
	$A(2,0)$, $B(6,0)$ e $C(0,6)$;\label{enum:ex:parabola:1:1}
	\item l'area del triangolo $ACH$ essendo $H$ l'ulteriore punto di intersezione di $\mathscr{P}_1$
	con la perpendicolare per $A$ alla retta $AC$;\label{enum:ex:parabola:1:2}
	\item l'equazione della circonferenza circoscritta al triangolo $CAH$;\label{enum:ex:parabola:1:3}
\end{enumerate}
\divisor

Per trovare l'equazione della parabola, possiamo sfruttare i 3 punti conosciuti e metterli a sistema
\begin{align*}
&\begin{cases}
4a+2b+c=0\\
36a+6b+c=0\\
c=6
\end{cases}\rightarrow
\begin{dcases}
a = \frac{-b-3}{2}\\
\cancelto{18}{36}\frac{-b-3}{\cancel{2}}+6b+6=0\\
c=6
\end{dcases}\rightarrow\\
&\begin{dcases}
a = \frac{-b-3}{2}\\
-18b-54_6b+6=0\\
c=6
\end{dcases}\rightarrow
\begin{dcases}
a = \frac{1}{2}\\
b = -4\\
c = 6
\end{dcases} \rightarrow\\ &\boxed{\mathscr{P}_1:\,y=\frac{1}{2}x^2-4x+6}
\end{align*}

E per concludere il punto \ref{enum:ex:parabola:1:1} disegnamo il grafico
\begin{center}
	\begin{tikzpicture}[scale=0.75]
		\coordinate (A) at (2,0);
		\coordinate (B) at (6,0);
		\coordinate (C) at (0,6);
		\coordinate (H) at (20/3,14/9);
		
		%\filldraw[orange, fill opacity = 0.3] (A) -- (H) -- (C) -- cycle;
		
		\tkzInit[xmin=-1,ymin=-2,xmax=9,ymax=7]
		\tkzGrid
		\tkzAxeXY
		
		\draw[thick, domain=-0.25:8.25] plot(\x, {0.5*\x*\x-4*\x+6});
		%\draw[red, domain=-0.3:2.7] plot(\x,{-3*\x+6});
		%\draw[red, thick, domain=-1:9] plot(\x, {\x/3-2/3});
		
		\filldraw (A) circle (0.05)
			node[below]{$A(2,0)$};
		\filldraw (B) circle (0.05)
			node[below]{$B(6,0)$};
		\filldraw (C) circle (0.05)
			node[below]{$C(0,6)$};
		%\filldraw (H) circle (0.05)
			%node[right]{$H(\dfrac{20}{3},\dfrac{14}{9})$};
	\end{tikzpicture}
\end{center}

Il punto \ref{enum:ex:parabola:1:2} richiede qualche passaggio intermedio. Per prima cosa troviamo
la retta passante per $AC$
\begin{equation*}
r_{AC}:\, \frac{y-y_1}{y_2-y_1} = \frac{x-x_1}{x_2-x_1} \rightarrow \frac{y}{6}=\frac{x-2}{-2}
\rightarrow r_{AC}:\,y=-3x+6
\end{equation*}

E ora dobbiamo trovare la perpendicolare passante per $A$.
\begin{align*}
r_{\perp AH}&:\,y=-\frac{1}{m}x+q \rightarrow y=\frac{x}{3}+q\rightarrow 0=\frac{2}{3}+q \rightarrow\\
r_{\perp AH}&:\,y=\frac{x}{3}-\frac{2}{3}
\end{align*}

E possiamo trovare $H$ facendo l'intersezione con la parabola $\mathscr{P}_1$
\begin{align*}
&\begin{dcases}
y = \frac{1}{2}x^2-4x+6\\
y=\frac{x}{3}-\frac{2}{3}
\end{dcases}\rightarrow
\begin{dcases}
\frac{x}{3}-\frac{2}{3} = \frac{1}{2}x^2-4x+6\\
y=\frac{x}{3}-\frac{2}{3}
\end{dcases}\rightarrow\\
&-\frac{1}{2}x^2+\frac{13}{3}x-\frac{20}{3}=0 \rightarrow 
x_{1/2} = \\
&\frac{-\frac{13}{3}\pm\sqrt{\frac{169}{9}-4\cdot-\frac{1}{2}\cdot-\frac{20}{3}}}{-1} 
\rightarrow\\ &-\frac{13}{3}\pm\frac{7}{3}\rightarrow\begin{dcases}
x_1 = 2\\
x_2 = \frac{20}{3}
\end{dcases}
\end{align*}
Il primo risultato ce lo aspettavamo in quanto � il punto $A$ che fa parte sia della retta che della
parabola.
\begin{equation*}
y = \frac{1}{2}x^2-4x+6 \, \text{con } x = \frac{20}{3} \rightarrow y = \frac{14}{9}
\end{equation*}
E quindi il nostro punto �
\begin{equation*}
H\left(\frac{20}{3},\frac{14}{9}\right)
\end{equation*}
Prima di proseguire, aggiorniamo il disegno
\begin{center}
	\begin{tikzpicture}[scale=0.75]
	\coordinate (A) at (2,0);
	\coordinate (B) at (6,0);
	\coordinate (C) at (0,6);
	\coordinate (H) at (20/3,14/9);
	\coordinate (M) at (1,3);
	\coordinate (N) at (13/3,7/9);
	\coordinate (K) at (10/3,34/9);
	
	\tkzInit[xmin=-1,ymin=-2,xmax=9,ymax=7]
	\tkzGrid
	\tkzAxeXY
	
	\filldraw[orange, fill opacity = 0.3] (A) -- (H) -- (C) -- cycle;
	
	\draw[thick, domain=-0.25:8.25] plot(\x, {0.5*\x*\x-4*\x+6});
	\draw[red, domain=-0.3:2.7] plot(\x,{-3*\x+6});
	\draw[red, thick, domain=-1:9] plot(\x, {\x/3-2/3});
	%\draw[blue, thick, domain=-1:9] plot(\x, {1/3*\x+8/3}); % M
	%\draw[blue, thick, domain=2.25:5.25] plot(\x, {-3*\x+124/9}); % N
	%\draw[blue, thick, domain=-0.5:5.5] plot(\x, {3/2*\x-11/9}); % K
	
	\filldraw (A) circle (0.05)
	node[below]{$A(2,0)$};
	\filldraw (B) circle (0.05)
	node[below]{$B(6,0)$};
	\filldraw (C) circle (0.05)
	node[below]{$C(0,6)$};
	\filldraw (H) circle (0.05)
	node[right]{$H(\dfrac{20}{3},\dfrac{14}{9})$};
	%\filldraw (M) circle (0.05)
	%node[above]{$M$};
	%\filldraw (K) circle (0.05)
	%node[above]{$K$};
	%\filldraw (N) circle (0.05)
	%node[above]{$N$};
	\end{tikzpicture}
\end{center}
L'area del triangolo � facilmente calcolabile con la formula
\begin{equation*}
\mathscr{A}(\mathscr{T}) = \frac{1}{2}\lvert
x_1y_2+y_1x_3+x_2y_3
-x_3y_2-y_3x_1-x_2y_1
\rvert
\end{equation*}
e quindi sostituendo
\begin{align*}
\mathscr{A}(\mathscr{T}) &= \frac{1}{2}\lvert
\cancel{x_1y_2}+y_1x_3+x_2y_3
\cancel{-x_3y_2}\cancel{-y_3x_1}-x_2y_1
\rvert \rightarrow\\
\mathscr{A}(\mathscr{T}) &= \frac{1}{2}\lvert 6\frac{20}{3}+2\frac{14}{9}-2\cdot6\rvert  = 
\frac{1}{2}\frac{280}{9} = \boxed{\frac{140}{9}}
\end{align*}

Il punto \ref{enum:ex:parabola:1:3} richiede qualche passaggio intermedio anch'esso. Per trovare la
circonferenza circoscritta al triangolo, dobbiamo innanzitutto trovare il centro. In un triangolo 
qualsiasi, il centro della circonferenza circoscritta � denominato \emph{circocentro} ed esso � il
punto di intersezione degli assi dei lati. Quindi per prima cosa si trovino i punti medi dei lati
utilizzando la formula
\begin{equation*}
\left(\frac{x_1+x_2}{2},\frac{y_1+y_2}{2}\right)
\end{equation*}
e otteniamo i seguenti risultati
\begin{equation*}
M(1,3) \qquad N\left(\frac{13}{3},\frac{7}{9}\right) \qquad K\left(\frac{10}{3},\frac{34}{9}\right)
\end{equation*}

Dobbiamo poi trovarci le rette dei lati per poi poter trovarne le perpendicolari. Avendo gi� fatto
il processo, riporto solo i risultati
\begin{align*}
r_{AC}&:\,y=-3x+6\\
r_{CH}&:\,y=-\frac{2}{3}x+6\\
r_{AH}&:\,y=\frac{x}{3}-\frac{2}{3}
\end{align*}
Per trovare le perpendicolari abbiamo una formula molto comoda
\begin{equation*}
y = -\frac{1}{m}(x-x_0)+mx_0+q
\end{equation*}
Essendo anche qui solo una questione di calcoli, riporto solo i risultati
\begin{align*}
r_{\perp AC}&:\,y=\frac{1}{3}x+\frac{8}{3}\\
r_{\perp CH}&:\,y=\frac{3}{2}x+\frac{65}{9}\\
r_{\perp AH}&:\,y=-3x+\frac{124}{9}
\end{align*}
E ora possiamo disegnare
\begin{center}
	\begin{tikzpicture}[scale=0.75]
	\coordinate (A) at (2,0);
	\coordinate (B) at (6,0);
	\coordinate (C) at (0,6);
	\coordinate (H) at (20/3,14/9);
	\coordinate (M) at (1,3);
	\coordinate (N) at (13/3,7/9);
	\coordinate (K) at (10/3,34/9);
	
	\def\R{4};
	
	\tkzInit[xmin=-1,ymin=-2,xmax=9,ymax=8]
	\tkzGrid
	\tkzAxeXY
	
	\filldraw[orange, fill opacity = 0.3] (A) -- (H) -- (C) -- cycle;
	
	\draw[thick, domain=-0.45:8.45] plot(\x, {0.5*\x*\x-4*\x+6});
	\draw[red, domain=-0.65:2.7] plot(\x,{-3*\x+6});
	\draw[red, thick, domain=-1:9] plot(\x, {\x/3-2/3});
	\draw[blue, domain=-1:9] plot(\x, {1/3*\x+8/3}); % M
	\draw[blue, domain=1.95:5.25] plot(\x, {-3*\x+124/9}); % N
	\draw[blue, domain=-0.5:6.15] plot(\x, {3/2*\x-11/9}); % K
	\draw[thick] (K) circle (\R);
	
	\filldraw (A) circle (0.05)
	node[below]{$A(2,0)$};
	\filldraw (B) circle (0.05)
	node[below]{$B(6,0)$};
	\filldraw (C) circle (0.05)
	node[below]{$C(0,6)$};
	\filldraw (H) circle (0.05)
	node[right]{$H(\dfrac{20}{3},\dfrac{14}{9})$};
	\filldraw (M) circle (0.05)
	node[above]{$M$};
	\filldraw (K) circle (0.05)
	node[above]{$K$};
	\filldraw (N) circle (0.05)
	node[above]{$N$};
	\end{tikzpicture}
\end{center}
Da questo disegno possiamo vedere che il punto di intersezione tra le tre rette � esattamente $K$.
Quindi per definire la circonferenza, basta solo trovare il raggio che equivale alla distanza $CK=KH$.
\begin{align*}
r &= CK = \sqrt{(x_C-x_K)^2+(y_C-y_K)^2} \rightarrow \\
r &= \sqrt{\left(0-\frac{10}{3}\right)^2+\left(6-\frac{34}{9}\right)^2} \rightarrow
r = \sqrt{\frac{400}{9}+\frac{1600}{81}} = \\&\frac{20\sqrt{13}}{9}
\end{align*}
E quindi la circonferenza diventa
\begin{equation*}
\boxed{
	\mathscr{C}:\,\left(x-\frac{10}{3}\right)^2+\left(y-\frac{34}{9}\right)^2=\frac{20\sqrt{13}}{9}
}
\end{equation*}

\subsubsection*{\hyperref[subsec:geomanal:ellisse]{Ellisse}}\label{ex:ellisse}
\paragraph{Esercizio 1}
Scritta l'equazione della parabola del tipo $x=ay^2+by+c$ avente il vertice $V$ sull'asse $x$ e 
passante per i punti $(6,2)$ e $(16,3)$, determinare l'equazione dell'ellisse avente un vertice in
$V$ e due altri vertici nei punti di intersezione della parabola con l'asse $y$. Determinare i punti
$P_1$ e $P_2$ dell'ellisse che hanno distanza $\dfrac{\sqrt{39}}{2}$ da $V$.
\divisor

Trovare l'equazione della parabola � estremamente semplice, infatti basta mettere a sistema le 
informazioni che si hanno.
\begin{equation*}
\begin{dcases}
-\frac{b}{2a} = 0\\
6=4a+2b+c\\
16=9a+3b+c
\end{dcases}\rightarrow
\begin{cases}
b = 0\\
4a+c-6=0\\
c=-9a+16
\end{cases}\rightarrow
\begin{cases}
b=0\\a=2\\c=-2
\end{cases}
\end{equation*}
Quindi la nostra parabola � $\boxed{\mathscr{P}:\,x=2y^2-2}$ e possiamo anche subito trovare il 
vertice
\begin{equation*}
-\frac{b^2-4ac}{4} \rightarrow -\frac{-4\cdot2\cdot-2}{4} = -2 \rightarrow V(-2,0)
\end{equation*}
Disegnamo ora ci� che abbiamo
\begin{center}
	\begin{tikzpicture}[scale=0.9]
		\coordinate (V) at (-2,0);
		\coordinate (L1) at (0,1);
		\coordinate (L2) at (0,-1);
		
		\tkzInit[xmin=-4,ymin=-3,xmax=4,ymax=3]
		\tkzGrid
		\tkzAxeXY
		
		\draw[thick, domain=-1.73:1.73] plot ({2*\x*\x-2},\x);
		
		\filldraw (V) circle (0.05)
		node[below left]{$V(-2,0)$};
		\filldraw (L1) circle (0.05)
		node[above left]{$L_1(0,1)$};
		\filldraw (L2) circle (0.05)
		node[below left]{$L_2(0,-1)$};
	\end{tikzpicture}
\end{center}
Troviamo subito gli altri due vertici dell'ellisse sostituendo $x=0$ nell'equazione della parabola
\begin{equation*}
y^2=1 \rightarrow y = \pm 1 \rightarrow L(0,\pm1)
\end{equation*}
E ora possiamo trovare l'ellisse sapendo che passa attraverso $V$ e $L_1$ (bastano solo questi due
vertici in quanto � simmetrica).
\begin{equation*}
\begin{dcases}
\frac{4}{a}=1\\
\frac{1}{b}=1
\end{dcases} \rightarrow
\begin{cases}
a=4\\
b=1
\end{cases} \rightarrow \boxed{\mathscr{E}:\,\frac{x^2}{16}+y^2=1}
\end{equation*}
Prima di disegnarla, osserviamo il punto successivo: ci chiede i punti dell'ellisse che si trovano ad
una certa distanza da $V$. Abbiamo un paio di modi, uno di questi � immaginare una circonferenza di
centro $V$ che abbia raggio pari alla distanza richiesta e vedere le intersezioni con l'ellisse.

\begin{center}
	\begin{tikzpicture}[scale=0.75]
	\coordinate (V) at (-2,0);
	\coordinate (L1) at (0,1);
	\coordinate (L2) at (0,-1);
	
	\def\R{sqrt(39)/2};
	
	\tkzInit[xmin=-6,ymin=-4,xmax=4,ymax=4]
	\tkzGrid
	\tkzAxeXY
	
	\draw[domain=-1.73:1.73] plot ({2*\x*\x-2},\x);
	\draw[thick] ellipse (2 and 1);
	\draw[red] (V) circle (\R);
	
	\filldraw (V) circle (0.05)
	node[below left]{$V(-2,0)$};
	\filldraw (L1) circle (0.05)
	node[above left]{$L_1(0,1)$};
	\filldraw (L2) circle (0.05)
	node[below left]{$L_2(0,-1)$};
	\end{tikzpicture}
\end{center}

La nostra circonferenza �
\begin{equation*}
(x-x_0)^2+(y-y_0)^2=r \rightarrow \mathscr{C}:\,(x+2)^2 + y^2 = \left(\frac{\sqrt{39}}{2}\right)^2
\end{equation*}

Per trovare i punti di intersezione, mettiamo a sistema le due equazioni
\begin{align*}
&\begin{dcases}
(x+2)^2+y^2=\frac{39}{4}\\
x^2+4y^2=4
\end{dcases}\rightarrow
\begin{dcases}
y^2 = \frac{-4x^2-16x-23}{4}\\
x^2+4\cdot\frac{-4x^2-16x-23}{4}=4
\end{dcases}\rightarrow\\
&\begin{dcases}
y^2 = \frac{-4x^2-16x-23}{4}\\
\begin{dcases}
x_1 = -\frac{19}{3}\\
x_2 = 1
\end{dcases}
\end{dcases}\rightarrow
\begin{dcases}
\begin{dcases}
y_1 = \pm\frac{5\sqrt{13}\imath}{6}\\
y_2 = \pm\frac{\sqrt{3}}{2}
\end{dcases}\\
\begin{dcases}
x_1 = -\frac{19}{3}\\
x_2 = 1
\end{dcases}
\end{dcases}
\end{align*}
Da queste soluzioni, eliminiamo quelle che non appartengono ad $\mathbb{R}$ e quindi otteniamo
i punti di intersezione
\begin{equation*}
\boxed{P_1\left(1,\frac{\sqrt{3}}{2}\right)\qquad P_2\left(1,-\frac{\sqrt{3}}{2}\right)}
\end{equation*}

\subsection*{\hyperref[sec:goniometria]{Goniometria}}\label{ex:goniometria}
\paragraph{Esercizio 1}
Risolvere la seguente equazione
\begin{equation*}
(\sqrt{3}+2)\cos x + \sin x + 1 =0
\end{equation*}
\divisor

Abbiamo gi� la fortuna che questa equazione � gi� stata semplificata ed organizzata. Notiamo 
osservandola che si tratta di un'equazione goniometrica lineare. Quindi procediamo con la risoluzione
\begin{align*}
\intertext{Poniamo}
&\cos x = X\,\text{e } \sin x = Y\\
&\begin{cases}
(\sqrt{3}+2)X + Y + 1 = 0\\
X^2+Y^2 = 1
\end{cases} \rightarrow\\
&\begin{cases}
Y = -1-(\sqrt{3}+2)X\\
X^2+1+(7+4\sqrt{3})X^2+2(2+\sqrt{3})X = 1
\end{cases}\\
\intertext{Quindi otteniamo le due possibili soluzioni}
&\begin{cases}
X = 0\\ Y=--1
\end{cases}\,\text{e }
\begin{dcases}
X = -\frac{1}{2}\\ Y = \frac{\sqrt{3}}{2}
\end{dcases}
\end{align*}
I due sistemi rappresentano le intersezioni con la circonferenza quindi ora non resta che trovare 
quali angoli (o archi) intersecano la circonferenza in quelle posizione. Ed essi sono
\begin{equation*}
x = \frac{2}{3}\pi + 2k\pi \qquad x = \frac{3}{2}\pi + 2k\pi
\end{equation*}

\paragraph{Esercizio 2}
Risolvere la seguente equazione
\begin{equation*}
\sin^2x + (1-\sqrt{3})\sin x\cos x-\sqrt{3}\cos^2 x = 0
\end{equation*}
\divisor

Notiamo che l'equazione � omogenea in quanto tutti i suoi termini sono di secondo grado. Dato che \\
contiene sia il termine di secondo grado in $\sin x$ sia in $\cos x$, possiamo scegliere per cosa 
dividere. Per preferenza personale, dividiamo per $\cos^2 x$.
\begin{align*}
&\frac{\sin^2x + (1-\sqrt{3})\sin x\cos x-\sqrt{3}\cos^2 x}{\cos^2 x} = 0 \rightarrow\\
&\tan^2 x + (1-\sqrt{3})\tan x - \sqrt{3} = 0
\intertext{che risolta d�}
(\tan x)_{1/2} &= \frac{\sqrt{3}-1\pm\sqrt{1+3-2\sqrt{3}+4\sqrt{3}}}{2} =\\
&\frac{\sqrt{3}-1\pm(1-\sqrt{3})}{2} = \begin{cases}
\tan x = -1\\\tan x = \sqrt{3}
\end{cases}
\intertext{che forniscono le soluzioni}
&x = -\frac{\pi}{4}+k\pi\,\text{e } x = \frac{\pi}{3}+k\pi
\end{align*}

\paragraph{Esercizio 3}
Nel settore circolare $AOB$ di raggio $r$, centro $O$ e angolo di apertura di $\ang{60}$ � inscritto
il rettangolo $MNPQ$ avente il vertice $M$ sull'arco $\arc{AB}$, il vertice $N$ sul raggio $OB$ e il
lato $PQ$ su $OA$. Determinare la posizione del vertice $M$ in modo che l'area di detto rettangolo
valga $\dfrac{\sqrt{3}}{6}r^2$.
\divisor

Per prima cosa, facciamo il disegno
\begin{center}
	\begin{tikzpicture}[scale=2]
		\coordinate (O) at (0,0);
		\coordinate (A) at (1,0);
		\coordinate (B) at (0.5,0.866);
		\coordinate (P) at (0.24,0);
		\coordinate (Q) at (0.923,0);
		\coordinate (N) at (0.24,0.4);
		\coordinate (M) at (0.923,0.4);
		
		\draw (A) -- (O) -- (B);
		\draw (A) arc(0:60:1);
		\markangle{O}{A}{B}{0.2}{2}{$\ang{60}$}
		\filldraw[cyan, fill opacity = 0.3] (P) -- (Q) -- (M) -- (N) -- cycle;
		\node (A1) at (A) [below right]{$A$};
		\node (B1) at (B) [above]{$B$};
		\node (P1) at (P) [below]{$P$};
		\node (Q1) at (Q) [below]{$Q$};
		\node (O1) at (O) [below left]{$O$};
		\node (N1) at (N) [above left]{$N$};
		\node (M1) at (M)[above right]{$M$};
	\end{tikzpicture}
\end{center}
Da questo possiamo dire che l'area di un rettangolo qualsiasi � definita come
$\text{base}\cdot\text{altezza}$, in questo caso come $\overline{PQ}\cdot\sin(\theta)r$ ($r$ � 
inserito per avere il seno corretto qualunque sia il raggio, $\theta = \arc{AOM}$). Il problema ora 
� trovare $\overline{PQ}$.\\
Definiamo $y_M$ e $y_N$ le due ordinate dei rispettivi punti.
\begin{equation*}
y_N = \overline{ON}\sin(\ang{60}) = y_M = r\sin(\theta)
\end{equation*} 
Questo lo vediaom chiaramente dal disegno.\\
$P$ si trova al piede di $N$, quindi la sua coordinata �
\begin{equation*}
\frac{y_N}{\tan(\ang{60})} = \frac{r\sin\theta}{\tan(\ang{60})}
\end{equation*}
Perch� da $y_N = \overline{ON}\sin(\ang{60})$ abbiamo isolato $\overline{ON}$ e moltiplicato per il
$\cos(\ang{60})$.\\
Infine $Q$ si trova al $\cos\theta$. Con queste informazioni, possiamo scrivere che
\begin{align*}
&\frac{\sqrt{3}}{6}r^2 = \left(r\cos\theta - \frac{r\sin\theta}{\tan(\ang{60})}\right)r\sin\theta\\
\intertext{Raccogliendo e semplificando $r$, risolvendo $\tan(\ang{60})$}
&\frac{1}{2\sqrt{3}} = \sin\theta\cos\theta -\frac{\sin^2\theta}{\sqrt{3}}\\
\intertext{Sostituiamo $\cos\theta = \sqrt{1-\sin^2\theta}$}
&\frac{1}{2\sqrt{3}} = \sin\theta\sqrt{1-\sin^2\theta}-\frac{\sin^2\theta}{\sqrt{3}}\\
\intertext{Poniamo $\sin\theta = t$}
&\frac{1}{2\sqrt{3}} = t\sqrt{1-t^2}-\frac{1}{\sqrt{3}}t^2\\
\intertext{Moltiplichiamo per $2\sqrt{3}$}
&1 = 2\sqrt{3}t\sqrt{1-t^2}-2t^2\\
\intertext{Spostiamo $t^2$}
&1+2t^2 = 2\sqrt{3}t\sqrt{1-t^2}\\
\intertext{Eleviamo al quadrato}
&1+4t^2+4t^4 = 12t^2-14t^4\\
\intertext{Semplifichiamo}
&-16t^4+8t^2-1=0\\
\intertext{Poniamo $u = t^2$}
&-16u^2-8u-1=0\\
\intertext{Risolviamo per $u$}
&u = \frac{1}{4}\\
\intertext{Torniamo a sostituire $t^2=u$}
&t = \sqrt{\frac{1}{4}} = \pm\frac{1}{2}\\
\intertext{Verifichiamo che solo $\dfrac{1}{2}$ � soluzione e torniamo a sostituire $t = \sin\theta$}
&\sin\theta = \frac{1}{2} \rightarrow \theta = \arcsin\left(\frac{1}{2}\right) = \boxed{\ang{30}}
\end{align*}

\subsection*{\hyperref[sec:logaritmi]{Logaritmi}}\label{ex:logaritmi}
\paragraph{Esercizio 1}
Risolvi
\begin{equation*}
50\left(\frac{4}{25}\right)^x-133\left(\frac{2}{5}\right)^x+20=0
\end{equation*}
\divisor

Per risolvere questo tipo di equazioni in modo semplice possiamo osservare attentamente e notare che
il primo termine tra parentesi ($\dfrac{4}{25}$) non � altro che il quadrato del secondo 
($\dfrac{2}{5}$)! Questo ci porta riscrivere l'equazione come
\begin{equation*}
50\left(\frac{2}{5}\right)^{2x}-133\left(\frac{2}{5}\right)+20=0
\end{equation*}
E ora possiamo risolvere semplicemente.
\begin{align*}
\intertext{Poniamo}
&t = \left(\frac{2}{5}\right)^x\\
\intertext{si ha quindi}
&50t^2-133t+20=0\\
&t_{1/2} = \frac{133\pm\sqrt{133-4\cdot50\cdot20}}{100} = \frac{133\pm117}{100}\\
&\begin{dcases}
t_1 = \frac{5}{2}\\
t_2 = \frac{4}{25}
\end{dcases}
\intertext{Torniamo a sostituire per $t_1$}
&\left(\frac{2}{5}\right)^x = \frac{5}{2}\rightarrow x = \log_{\frac{2}{5}}\frac{5}{2}
\intertext{Ricordando la propriet� $\log_{\frac{1}{a}} b = -\log_a b$}
&x = -\log_{\frac{5}{2}}\frac{5}{2} = \boxed{-1}
\intertext{Sostituiamo per $t_2$}
&\left(\frac{2}{5}\right)^x = \frac{4}{25}\rightarrow x = \log_{\frac{2}{5}}\frac{4}{25}
\intertext{Ricordando che $\log_a b^k = k\log_a b$}
&x = 2\log_{\frac{2}{5}}\frac{2}{5} = \boxed{2}
\end{align*}

\subsection*{\hyperref[sec:progressioni]{Progressioni}}\label{ex:progressioni}
\paragraph{Esercizio 1}
Trovare la somma dei primi $8$ termini di una progressione geometrica sapendo che il secondo termine
� $4$ e il quinto � $108$.

\divisor

Indichiamo come $a_1$ il primo termine e $q$ la ragione. Possiamo ora scrivere un sistema che ci
"matematizza" ci� che ci viene detto
\begin{align*}
&\begin{cases}
a_1q = 4\\
a_1q^4=108
\end{cases}
\intertext{da questo possiamo dividere membro a membro}
&\frac{a_1q}{a_1q^4}=q^3=\frac{108}{4}=27 \rightarrow q = 3
\intertext{Sostituendo nella prima equazione}
&a_1 = \frac{4}{3}
\intertext{E possiamo quindi trovare la somma di $8$ elementi}
&S_n = a_1\frac{1-q^n}{1-q} \rightarrow S_8 =
\frac{4}{3}\cdot\frac{1-3^8}{1-3}=\boxed{\frac{13120}{3}}
\end{align*}

\paragraph{Esercizio 2}
Una progressione aritmetica ha il primo termine $a_1=a$ e ragione $d=10$. La somma dei primi $n$ 
termini � pari a $10000$. Determinare l'espressione che fornisce $a_1$ in funzione di $n$ e calcolare
il valore di $a_{20}$.

\divisor

Ricordando la formula per la somma di una progressione
\begin{equation*}
S_n = \frac{a_1+a_n}{2}n
\end{equation*}
vediamo che per trovare $a_1$ ci manca solo $a_n$. Per trovarlo usiamo la formula per trovare
l'$n$-esimo elemento
\begin{align*}
&a_n = a_1+d(n-1) \rightarrow a_n = a+dn-d = a_n = a+10n-10
\intertext{A questo punto riscriviamo la formula della somma con tutti i dati}
&10000 = \frac{a+a+10n-10}{2}n\rightarrow 20000 = 2an+10n^2-10n
\intertext{Isoliamo $a$}
&\boxed{a = \frac{10000}{n}-5n+5}
\end{align*}

A questo punto troviamo l'elemento riapplicando la formula
\begin{equation*}
a_{20} = a_1 + d(20-1) = \frac{10000}{20}-5\cdot20+5+10\cdot(20-1) =\boxed{595} 
\end{equation*}

\subsection*{\hyperref[sec:calccomb]{Calcolo combinatorio}}\label{ex:calccomb}
\paragraph{Esercizio 1}
Determinare in quanti modi � possibile estrarre due carte da un mazzo di $52$ in modo che
\begin{itemize}
	\item le due carte estratte siano entrambe rosse
	\item una sia rossa e l'altra nera
	\item una almeno sia rossa
\end{itemize}
\divisor

Per il primo punto vediamo che ci chiedono 2 carte rosse. Un mazzo da $52$ contiene $26$ nere e 
altrettante rosse essendo un mazzo di carte francesi. Dato che l'ordine non ha importanza e che
tutti gli elementi sono distinti, le possibilit� sono le combinazioni semplici. Quindi
\begin{equation*}
\boxed{\binom{26}{2}}
\end{equation*}
$26$ perch� ci interessano solo le rosse, non tutto il mazzo.\\\\
Il prossimo punto chiede una carta rossa e una carta nera. Immaginiamo di avere quindi due spazi. Nel
primo mettiamo una carta rossa (quindi $26$ possibilit�), nel secondo una nera (sempre $26$). Quindi
le totali possibilit� si ottengono semplicemente moltiplicando
\begin{equation*}
26\cdot26 = \boxed{676}
\end{equation*}
Per l'ultimo punto, dobbiamo sottrarre da tutte le possibilit� quelle che non contengono una carta 
rossa. Dato che ci viene detto \emph{almeno} una rossa, possono essere anche entrambe. Quindi
\begin{equation*}
\boxed{\binom{52}{2}-\binom{26}{2}}
\end{equation*}

\paragraph{Esercizio 2}
Dimostrare
\begin{equation*}
\binom{n}{k+1} = \binom{n}{k}\frac{n-k}{k+1}
\end{equation*}
\divisor

\begin{align*}
\intertext{Partiamo sviluppando il lato sinistro}
&\binom{n}{k+1} = \frac{n!}{(k+1)!(n-k-1)!} = \frac{n!}{k!(k+1)(n-k-1)!}
\intertext{Se osserviamo attentamente notiamo che ci manca semplicemente un $n-k$ da aggiungere per
ottenere il desiderato. Quindi possiamo moltiplicare per $n-k$ e ottenere}
&\frac{n!}{k!(k+1)(n-k-1)!} = \binom{n}{k}\frac{n-k}{k+1}
\intertext{Q.E.D.}
\end{align*}

\subsection*{\hyperref[sec:prob]{Probabilit�}}\label{ex:prob}
\paragraph{Esercizio 1}
Tre macchine utensili producono lo stesso tipo di pezzi. La prima ne produce $150$ al giorno con il 
$2\%$ dei pezzi difettosi, la seconda ne produce $500$ con il $5\%$ di pezzi difettosi, la terza
$50$ con nessun pezzo difettoso. Supponiamo ora di prendere un pezzo a caso della produzione di un 
dato giorno, calcolare la probabilit� che
\begin{enumerate}
	\item il pezzo sia stato prodotto dalla prima macchina
	\item il pezzo sia stato prodotto dalla seconda macchina
	\item il pezzo sia stato prodotto dalla terza macchina
	\item il pezzo sia difettoso
\end{enumerate}
Supponendo che il pezzo sia difettoso, calcolare la probabilit� che
\begin{enumerate}
	\item sia stato prodotto dalla prima macchina
	\item sia stato prodotto dalla seconda macchina
	\item sia stato prodotto dalla terza macchina
\end{enumerate}
\divisor

Una cosa che ritorna estremamente utile nella risoluzione dei problemi di probabilit� � il grafico ad
albero per mostrare tutte le possibilit�. Come questo

\begin{center}
	\begin{tikzpicture}
		\tikzset{level 1/.style={level distance = 40pt}}
		\tikzset{level 2/.style={level distance = 40pt}}
		\Tree [.\node (Produzione) {Produzione};
			[.\node (Prima) {Prima};
				[.\node (Buono1) {Buono};
					$\dfrac{147}{600}$
				]
				[.\node (Difettoso1) {Difettoso};
					$\dfrac{3}{600}$
				]
			]
			[.\node (Seconda) {Seconda};
				[.\node (Buono2) {Buono};
					$\dfrac{380}{600}$
				]
				[.\node (Difettoso2) {Difettoso};
					$\dfrac{20}{600}$
				]
			]
			[.\node (Terza) {Terza};
				[.\node (Buono3) {Buono};
					$\dfrac{50}{600}$
				]
				[.\node (Difettoso3) {Difettoso};
					$0$
				]
			]
		]
		\node[above] at ($(Produzione) !.5! (Prima)$) {$\dfrac{150}{600}$};
		\node[left] at ($(Produzione) !.6! (Seconda)$) {$\dfrac{400}{600}$};
		\node[above] at ($(Produzione) !.5! (Terza)$) {$\dfrac{50}{600}$};
		\node[left] at ($(Prima) !.5! (Buono1)$) {$\dfrac{2}{100}$};
		\node[right] at ($(Prima) !.5! (Difettoso1)$) {$\dfrac{98}{100}$};
		\node[left] at ($(Seconda) !.5! (Buono2)$) {$\dfrac{5}{100}$};
		\node[right] at ($(Seconda) !.5! (Difettoso2)$) {$\dfrac{95}{100}$};
		\node[left] at ($(Terza) !.5! (Buono3)$) {$0$};
		\node[right] at ($(Terza) !.5! (Difettoso3)$) {$\dfrac{100}{100}$};
	\end{tikzpicture}
\end{center} 

I risultati che si vedono sono semplicemente ricavati dal prodotto delle due probabilit�
secondo la formula $p\left(\mathbb{E}_1\cap\mathbb{E}_2\right) = p(\mathbb{E}_1)\cdot p(\mathbb{E}_2)$
.\\\\
In totale vi sono $23$ pezzi difettosi in un giorno (infatti se andiamo ad osservare i numeratori e 
sommiamo i difettosi vediamo $3+20+0 = 23$). Quindi la probabilit� che un pezzo sia difettoso �
\begin{equation*}
p(\text{Difettoso}) = \frac{23}{600} \approx \boxed{3.8\%}
\end{equation*}

Allora, la probabilita che sia difettoso e dalla prima macchina �
\begin{equation*}
p\left(\text{Prima}\mid\text{Difettoso}\right) = \frac{\frac{3}{600}}{\frac{23}{600}} 
\approx\boxed{13\%}
\end{equation*}
e dalla seconda
\begin{equation*}
p\left(\text{Seconda}\mid\text{Difettoso}\right) = \frac{\frac{20}{600}}{\frac{23}{600}} 
\approx\boxed{8.7\%}
\end{equation*}

\subsection*{\hyperref[sec:aff]{Affinit�}}\label{ex:aff}
\paragraph{Esercizio 1}
Data l'affinit�
\begin{equation*}
T:\,\begin{cases}
x'=2x+y-1\\y'=x-y-2
\end{cases}
\end{equation*}
determinare
\begin{enumerate}
	\item il punto unito $U$
	\item i trasformati $O'$, $A'$ e $B'$ dei punti $O(0,0)$, $A(2,-1)$, $B(-3,4)$ e l'area del 
	triangolo $A'O'B'$
	\item la trasformazione inversa $T^{-1}$
	\item le trasformate delle curve
	\begin{equation*}
	y=3x+4\qquad x-y+5=0\qquad y=x^2
	\end{equation*}
\end{enumerate}
\divisor

Trovare il punto unito di $T$ � semplice, basta solo porre $x'=x$ e $y'=y$.
\begin{align*}
&\begin{cases}
x=2x+y-1\\
y=x-y-2
\end{cases}
\begin{cases}
y = -x+1\\
-x+1=x+x-4
\end{cases}
\begin{cases}
y=\frac{1}{3}\\
x=\frac{4}{3}
\end{cases}\rightarrow \\&U\left(\frac{1}{3},\frac{4}{3}\right)
\end{align*}

Trovare i trasformati � estremamente semplice: basta sostituire i valori di $x$ e $y$ dei punti
all'interno dell'affinit�
\begin{equation*}
\boxed{
O':\,\begin{cases}
x'=-1\\y'=-2
\end{cases}
A':\,\begin{cases}
x'=4-2=2\\
y'=2+1-2=1
\end{cases}
B':\,\begin{cases}
x'=-3\\
y'=-9
\end{cases}
}	
\end{equation*}

L'area del triangolo � facilmente trovabile usando la matrice

\begin{equation*}
\mathscr{A}(ABC) = \frac{1}{2}\left\lvert 
\begin{matrix}[1]
x_1 & y_1 & 1\\
x_2 & y_2 & 1\\
x_3 & y_3 & 1
\end{matrix}\right\rvert
\end{equation*}
e risolvendola con Sarrus. Quindi inserendo i valori e risolvendo otteniamo
\begin{align*}
&\mathscr{A}(A'O'B') = \frac{1}{2}\left\lvert 
-1+(-18)+6-(-3+9-4)\right\rvert = \\&\frac{1}{2}\left\lvert 
-1-18+6+3-9+4\right\rvert = \\&\frac{1}{2}\left\lvert-15\right\rvert = \boxed{\frac{15}{2}}
\end{align*}

Trovare la trasformazione inversa richiede solo di risolvere il sistema in $x$ e $y$.
\begin{align*}
&\begin{cases}
2x+y=x'+1\\
x-y=y'+2
\end{cases}
\begin{cases}
2y'+4+3y'=x'+1\\
x='+2+y
\end{cases}\\
&\boxed{\begin{dcases}
y = \frac{x'}{3}+\frac{2}{3}y'-1\\
x = \frac{x'}{3}+\frac{y'}{3}+1
\end{dcases}}
\end{align*}


Trovare le trasformate di un'equazione � relativamente semplice. Piuttosto tedioso per alcune formule
ma non complicato. Quello che si fa � applicare la trasformazione inversa alla nostra equazione.
Questo perch� immaginiamo che ci� che abbiamo sia un risultato e noi dobbiamo trovare l'equazione
che ha portato a quella data. Quindi torniamo indietro e dunque usiamo l'inversa.
\begin{align*}
&y=3x+4 \rightarrow\\
&\frac{x'}{3}-\frac{2}{3}y'-1=3\left(\frac{x'}{3}+\frac{y'}{3}+1\right)+4 \rightarrow\\
&\frac{x'}{3}-\frac{2}{3}y'-1=x'+y'+7\rightarrow\\\Aboxed{&2x'+5y'+24=0}
\end{align*}
\begin{align*}
&x-y+5=0\rightarrow\\
&\cancel{\frac{x'}{3}}+\frac{y'}{3}+1-\cancel{\frac{x'}{3}}+\frac{2}{3}y'+1+5 \rightarrow
\boxed{y'+5=0}
\end{align*}
\begin{align*}
&y=x^2\rightarrow\\
&0=-\frac{x'}{3}+\frac{2}{3}y'+1+\frac{x'}{9}+\frac{2}{9}xy+\frac{2}{9}x+\frac{y^2}{9}+\frac{2}{9}y+1
\\ &\boxed{x^{'2}+2x'y'+y^{'2}+3x'+12y'+18=0}
\end{align*}

\paragraph{Esercizio 2}
Considera nel piano $xOy$ la famiglia di curve di equazione
\begin{equation*}
y=\frac{mx-8}{x-2m}
\end{equation*}
determinare:
\begin{enumerate}
	\item per quali valori di $m$ l'equazione rappresenta un'iperbole equilatera traslata e il luogo
	di simmetria delle iperboli della famiglia \label{enum:ex:aff:2:1}
	\item le iperboli $\Phi_1$ e $\Phi_2$ della famiglia che sono tangenti, nel loro punto di ascissa
	nulla, alla retta con coefficiente angolare $\dfrac{1}{2}$. Sia $\Phi_1$ quella relativa al valore
	$m>0$ \label{enum:ex:aff:2:2}
	\item il luogo $\gamma$ dei centri delle circonferenze passanti per $O$ e tangenti al luogo
	di simmetria delle iperboli \label{enum:ex:aff:2:3}
	\item detta $\theta'$ la curva corrispondente di $\theta$ nella trasformazione
	\label{enum:ex:aff:2:4}
	\begin{equation*}
	\begin{dcases}
	x = \frac{\sqrt{2}}{2}x'-\frac{\sqrt{2}}{2}y'\\
	y = \frac{\sqrt{2}}{2}x'+\frac{\sqrt{2}}{2}y'
	\end{dcases}
	\end{equation*}
	determinare il punto $P$ di $\theta'$ nel primo quadrante in corrispondenza del quale � \\
	\textbf{massima} l'area del quadrilatero $OVPM$ essendo $V$ il punto di $\theta'$ di ordinata
	nulla e $M$ il punto di intersezione di $\theta'$ con i semiasse negativo delle ordinate.
	Sia $\theta$ la circonferenza con raggio pari a $1$ che si trova nella parte positiva di $y$.
\end{enumerate}
\divisor

Per il \hyperref[enum:ex:aff:2:1]{\textbf{punto uno}} dobbiamo vedere in che caso l'equazione non
descrive pi� un'iperbole. Per fare questo proviamo a semplificare e vediamo che dividendo
\begin{equation*}
\frac{mx}{x} \quad \text{e} \quad \frac{8}{2m}
\end{equation*}
otteniamo
\begin{equation*}
m = \frac{8}{2m} \rightarrow m^2 = 4 \rightarrow \boxed{m=\pm2}
\end{equation*}
Quindi i valori che noi dobbiamo esculdere sono $m\neq\pm2$ in quanto annullano l'equazione che
diventerebbe una retta passante per $y=\pm2$.\\
Il luogo dei centri di simmetria � quella retta su cui giaciono tutti i centri. Un centro � il punto
\begin{equation*}
C\left(-\frac{d}{c},\frac{a}{c}\right)
\end{equation*}
ed inserendo i nostri valori otteniamo
\begin{equation*}
C(2m,m)
\end{equation*}
quindi il luogo � dato dall'equazione $\boxed{y=\dfrac{x}{2}}$. Dobbiamo per� escludere i punti
\begin{equation*}
(4,2)\qquad(-4,-2)
\end{equation*}
in quanto abbiamo escluso $m=\pm2$.\\\\

Per il \hyperref[enum:ex:aff:2:2]{punto due} possiamo trovare i punti che distano $0$ dal fascio di
rette $y=\dfrac{1}{2}x+q$ e che appartengono alla nostra famiglia di iperboli che al tempo stesso
hanno $x=0$ ma non otterremmo nulla di significativo.\\
Perch�? Guardiamo un attimo il disegno
\begin{center}
	\begin{tikzpicture}
		\clip (-1,-1.5) rectangle (8,5.5); 
		\tkzInit[xmin=0,ymin=-1,xmax=7,ymax=5]
		\tkzGrid
		\tkzAxeXY
		\draw[red, thick, domain=-3:7, samples=500] plot({\x}, {(1.76*\x-8)/(\x-2*1.76)});
		\draw[blue, thick, domain=-1:6] plot({\x},{\x/2+1.92});
	\end{tikzpicture}
\end{center}
Notiamo che per due valori di $q$ e $m$ scelti appositamente il pi� vicino punto di intersezione tra 
la retta e l'iperbole ha come ascissa un valore leggermente inferiore a $2$. Andando a disegnare altre
iperboli e altre rette modificando i due parametri si andr� a notare che � impossibile che la retta
$y=\dfrac{x}{2}+q$ per qualunque $q$ sia tangente all'iperbole $\Phi$ in modo che $x=0$.\\\\

Per il \hyperref[enum:ex:aff:2:3]{punto tre} abbiamo un problema interessante. Abbiamo una 
circonferenza che passa per $O(0,0)$ e il cui centro abbia distanza dalla retta pari al raggio.
Scriviamo quindi il sistema
\begin{align*}
&\begin{dcases}
c = 0\\
\frac{\left\lvert -\frac{a}{4}+\frac{b}{2}+c\right\rvert}{\sqrt{a^2+b^2}}=
\sqrt{\frac{a^2}{4}+\frac{b^2}{4}-c}
\end{dcases}\rightarrow
\frac{\left\rvert a-2b\right\rvert}{2\sqrt{5}}=\sqrt{\frac{a^2}{4}+\frac{b^2}{4}}
\intertext{Semplifichiamo e otteniamo}
&b = -2a
\end{align*}
Cosa ci dice questo? Proviamo a inserire le informazioni all'interno della coordinata del centro
\begin{equation*}
C\left(-\frac{a}{2},-\frac{b}{2}\right)\rightarrow C\left(-\frac{a}{2},a\right)
\end{equation*}
Da questo vediamo che $y=-2x$. Quindi il nostro luogo dei centri �
\begin{equation*}
\boxed{\gamma:\,y = -2x}
\end{equation*}

Avremmo potuto farlo in un altro modo? Con un po' di ragionamento, s�, senza dubbio. Sappiamo che
le circonferenze devono passare per $O(0,0)$ e essere tangenti a $y=\dfrac{x}{2}$. Per� anche questa
retta passa per $O(0,0)$ in quanto ha $q=0$. Quindi il punto di tangenza � esattamente il centro degli
assi. Se sappiamo questo e sappiamo che le circonferenze devono essere tangenti, significa che 
sappiamo il raggio � perpendicolare alla retta. Quindi la retta passante per il centro e 
perpendicolare a $y=\dfrac{x}{2}$ � proprio $y=-2x$.\\\

Per il \hyperref[enum:ex:aff:2:4]{punto quattro} troviamo la circonferenza che ha il centro su quella
retta e che abbia il raggio pari a $1$.
\begin{align*}
\intertext{L'equazione della generica circonferenza � }
&x^2+y^2+ax-2ay=0
\intertext{perch� avendo visto prima la soluzione dell'equazione del luogo geometrico sappiamo che
$b=-2a$. Scriviamo tutto in funzione di $a$ per comodit�.
Con questo chiarito, troviamo il valore di $a$ sapendo che il raggio � pari a $1$}
&\sqrt{\frac{a^2}{4}+a^2} = 1 \rightarrow \frac{4}{5}a^2=1 \rightarrow a=\pm\frac{2\sqrt{5}}{5}
\intertext{Per scegliere quale dei due � accettabile, disegnamoli}
\end{align*}

\begin{center}
	\begin{tikzpicture}[scale=2]
		\coordinate (C) at (1/4.46,-1/2.23);
		\coordinate (C1) at (-1/4.46,1/2.23);
		\tkzInit[xmin=-1,ymin=-1,xmax=1,ymax=1]
		\tkzGrid
		\tkzAxeXY
		
		\draw[red, thick] (C) circle (1/2);
		\draw[blue, thick] (C1) circle (1/2);
		\filldraw[red] (C) circle (0.025);
		\filldraw[blue] (C1) circle (0.025);
		\node[red, below] at (C) {$a=\dfrac{2\sqrt{5}}{5}$};
		\node[blue, above] at (C1) {$a=-\dfrac{2\sqrt{5}}{5}$};
	\end{tikzpicture}
\end{center}
Quindi tra queste due quella da scegliere � quella con $a < 0$. Scriviamola
\begin{equation*}
\theta:\,x^2+y^2-\frac{2\sqrt{5}}{5}x+\frac{4\sqrt{5}}{5}y=0
\end{equation*}
La trasformazione che dobbiamo applicare �
\begin{equation*}
\begin{dcases}
x = \frac{\sqrt{2}}{2}x'-\frac{\sqrt{2}}{2}y'\\
y = \frac{\sqrt{2}}{2}x'+\frac{\sqrt{2}}{2}y'
\end{dcases}
\end{equation*}
che osservandola rappresenta una rotazione di $\alpha=\ang{45}$. Come mai? Beh, abbiamo la matrice dei
coefficienti che �
\begin{equation*}
\begin{bmatrix}[2.5]
\dfrac{\sqrt{2}}{2} & -\dfrac{\sqrt{2}}{2}\\
\dfrac{\sqrt{2}}{2} & \dfrac{\sqrt{2}}{2}
\end{bmatrix}
\end{equation*}
che messa accanto a quella della rotazione
\begin{equation*}
\begin{bmatrix}[1]
\cos\theta & -\sin\theta\\
\sin\theta & \cos\theta
\end{bmatrix}
\end{equation*}
risultano estremamente simili con $\theta = \dfrac{\pi}{4}$. Tanto simili da coincidere. Quindi 
sappiamo che abbiamo una rotazione di $\ang{45}$. Applichiamo la trasformazione inversa alla nostra
equazione e otteniamo
\begin{align*}
&\left(\frac{\sqrt{2}}{2}x'-\frac{\sqrt{2}}{2}y'\right)^2+
\left(\frac{\sqrt{2}}{2}x'+\frac{\sqrt{2}}{2}y'\right)^2+\\
&-\frac{2\sqrt{5}}{5}\left(\frac{\sqrt{2}}{2}x'-\frac{\sqrt{2}}{2}y'\right)+
\frac{4\sqrt{5}}{5}\left(\frac{\sqrt{2}}{2}x'+\frac{\sqrt{2}}{2}y'\right) = 0
\intertext{Che semplificata tramite calcoli che non riporto perch� eccessivamente lunghi}
&\boxed{\theta':\,x'^2+y^2+\sqrt{\frac{2}{5}}x+3\sqrt{\frac{2}{5}}y = 0}
\end{align*}
Che disegnata �
\begin{center}
	\begin{tikzpicture}[scale=2]
	\coordinate (C) at (-0.15,-0.46);
	\coordinate (C1) at (-1/4.46,1/2.23);
	
	\tkzInit[xmin=-1,ymin=-1,xmax=1,ymax=1]
	\tkzGrid
	\tkzAxeXY
	
	\draw[red, thick] (C) circle (1/2);
	\draw[blue, thick] (C1) circle (1/2);
	\filldraw[red] (C) circle (0.025);
	\filldraw[blue] (C1) circle (0.025);
	\end{tikzpicture}
\end{center}

Disegnamo ora i punti che ci vengono forniti
\begin{center}
	\begin{tikzpicture}[scale=2]
	\coordinate (C) at (-0.15,-0.46);
	\coordinate (V) at (-0.33,0);
	\coordinate (M) at (0,-0.9);
	\coordinate (O) at (0,0);
	\coordinate (P) at (-0.5,-0.8);
	
	\tkzInit[xmin=-1,ymin=-1,xmax=1,ymax=0.5]
	\tkzGrid
	\tkzAxeXY
	
	\draw[red, thick] (C) circle (1/2);
	\node[above] at (V) {$V$};
	\node[right] at (M) {$M$};
	\node[left] at (P) {$P$};
	\filldraw[orange, fill opacity = 0.3] (M) -- (P) -- (V) -- (O) -- cycle;
	\filldraw[red] (C) circle (0.025);
	\filldraw (V) circle (0.025);
	\filldraw (M) circle (0.025);
	\filldraw (O) circle (0.025);
	\filldraw (P) circle (0.025);
	\end{tikzpicture}
\end{center}
dove $P$ � un punto variabile. Per calcolare l'area massima di questo quadrilatero dobbiamo trovare
dove deve essere $P$ perch� sia massima. Ovviamente deve appartenere alla circonferenza, quindi
deve essere il pi� lontano possibile da $V$ e $M$. Quindi il punto pi� distante � quello in cui
la retta passante tra lui e il centro della circonferenza � perpendicolare alla retta passante
tra i due punti. Disegnamo per far capire.

\begin{center}
	\begin{tikzpicture}[scale=2]
	\coordinate (C) at (-0.15,-0.46);
	\coordinate (V) at (-0.33,0);
	\coordinate (M) at (0,-0.9);
	\coordinate (O) at (0,0);
	\coordinate (P) at (-0.6,-0.57);
	
	\tkzInit[xmin=-1,ymin=-1,xmax=1,ymax=0.5]
	\tkzGrid
	\tkzAxeXY
	
	\draw[red, thick] (C) circle (1/2);
	\node[above] at (V) {$V$};
	\node[right] at (M) {$M$};
	\node[left] at (P) {$P$};
	
	\filldraw[orange, fill opacity = 0.3] (M) -- (P) -- (V) -- (O) -- cycle;
	
	\draw[dashed, shorten >=-1cm] (P) -- (C);
	\draw[dashed] (V) -- (M);
	
	\filldraw[red] (C) circle (0.025);
	\filldraw (V) circle (0.025);
	\filldraw (M) circle (0.025);
	\filldraw (O) circle (0.025);
	\filldraw (P) circle (0.025);
	\end{tikzpicture}
\end{center}
L'area del quadrilatero � sempre ricavabile con la matrice
\begin{equation*}
\begin{vmatrix}[1]
x_1 & y_1 & 1\\
x_2 & y_2 & 1\\
x_3 & y_3 & 1\\
x_4 & y_4 & 1
\end{vmatrix}
\end{equation*}
usando la formula di Gauss. Prima troviamo i punti
\begin{equation*}
V:\, x^2+\sqrt{\frac{2}{5}}=0 \rightarrow x = -\sqrt{\frac{2}{5}} \rightarrow 
V\left(-\sqrt{\frac{2}{5}},0\right)
\end{equation*}
\begin{equation*}
M:\,y^2+3\sqrt{\frac{2}{5}}=0 \rightarrow y = -3\sqrt{\frac{2}{5}} \rightarrow 
M\left(0,-3\sqrt{\frac{2}{5}}\right)
\end{equation*}
La retta passante tra i due punti risulta essere
\begin{equation*}
r_{VM}:\,y=-3x-3\sqrt{\frac{2}{5}}
\end{equation*}
La tangente per $P$ invece
\begin{equation*}
t_{PC}:\,y=\frac{x}{3}-\frac{4}{3}\sqrt{\frac{2}{5}}
\end{equation*}
Il punto $P$ si trova facendo l'intersezione tra la tangente e la circonferenza. Si trova quindi
\begin{equation*}
P\left(-2\sqrt{\frac{2}{5}},\frac{1}{15}(5x-4\sqrt{10})\right)
\end{equation*}
Quindi l'area massima si trova risolvendo il determinante di
\begin{equation*}
\begin{bmatrix}[2.5]
0&0&1\\
-\sqrt{\dfrac{2}{5}}&0&1\\
0&-3\sqrt{\dfrac{2}{5}}&1\\
-2\sqrt{\dfrac{2}{5}}&\dfrac{1}{15}(5x-4\sqrt{10}&1
\end{bmatrix}
\end{equation*}
che risulta essere
\begin{equation*}
\frac{1}{2}\left\lvert-\sqrt{\frac{2}{5}}\cdot\left(-3\sqrt{\frac{2}{5}}\right) -
\left(-2\sqrt{\frac{2}{5}}\cdot\left(-3\sqrt{\frac{2}{5}}\right)\right)\right\rvert
\end{equation*}
che semplificato risulta in
\begin{equation*}
\boxed{\mathscr{A}(VOMP) = \frac{3}{5}}
\end{equation*}

\subsection*{\hyperref[sec:complex]{Numeri Complessi}}\label{ex:complex}
\paragraph{Esercizio 1}
Calcolare le radici quarte di $z=2+i2\sqrt{3}$
\divisor

Calcoliamo $\rho$ e $\theta$
\begin{align*}
&\rho = \sqrt{4+12} = \sqrt{16} = 4 \quad \theta = \frac{\pi}{3}
\intertext{Quindi si ha}
&z = 4\left(\cos\frac{\pi}{3}+i\sin\frac{\pi}{3}\right)
\intertext{da cui}
&w=\sqrt[4]{z} = 
\sqrt[4]{4}\left(\cos\frac{\frac{\pi}{3}+2k\pi}{4}+i\sin\frac{\frac{\pi}{3}+2k\pi}{4}\right) =\\ 
&\sqrt{2}\left(\cos\frac{\frac{\pi}{3}+2k\pi}{4}+i\sin\frac{\frac{\pi}{3}+2k\pi}{4}\right)
\end{align*}
Le radici quarte si ottengono sostituendo a $k$ i valori in ${0,1,2,3}$. Quindi otteniamo
\begin{align*}
\Aboxed{w_1 &= \sqrt{2}\left(\cos\frac{\pi}{12}+i\sin\frac{\pi}{12}\right)}\\
w_2 &= \sqrt{2}\left(\cos\frac{\frac{\pi}{3}+2\pi}{4}+i\sin\frac{\frac{\pi}{3}+2\pi}{4}\right) = \\
\Aboxed{&\sqrt{2}\left(\cos\frac{7}{12}\pi+i\sin\frac{7}{12}\pi\right)}\\
w_3 &= \sqrt{2}\left(\cos\frac{\frac{\pi}{3}+4\pi}{4}+i\sin\frac{\frac{\pi}{3}+4\pi}{4}\right) =\\ 
\Aboxed{&\sqrt{2}\left(\cos\frac{13}{12}\pi+i\sin\frac{13}{12}\pi\right)}\\
w_4 &= \sqrt{2}\left(\cos\frac{\frac{\pi}{3}+6\pi}{4}+i\sin\frac{\frac{\pi}{3}+6\pi}{4}\right) = \\
\Aboxed{&\sqrt{2}\left(\cos\frac{19}{12}\pi+i\sin\frac{19}{12}\pi\right)}
\end{align*}

\paragraph{Esercizio 2}
Risolvere nel campo complesso
\begin{equation*}
z^2+(1-4i)z-3-3i = 0
\end{equation*}
\divisor

Applichiamo la formula risolutiva delle equazioni di secondo grado che solitamente si usa per i reali
\begin{equation*}
z_{1/2} = \frac{1-4i+\sqrt{(1-4i)^2+12+12i}}{2} = \frac{1-4i+\sqrt{-3+4i}}{2}
\end{equation*}
avendo $\sqrt{-3+4i}$ ad indicare le radici del numero complesso $-3+4i$.\\
Per determinarle poniamo
\begin{align*}
&\sqrt{-3+4i}=a+ib
\intertext{con $a,b\in\mathbb{R}$. Eleviamo al quadrato}
&-3+4i=a^2-b^2+2iab
\intertext{e imponiamo che le due parti reali e immaginarie siano uguali}
&\begin{cases}
a^2-b^2=-3\\ab=2
\end{cases}
\intertext{Ricavando $b=\frac{2}{a}$ dalla seconda equazione e sostituendolo nella prima otteniamo
un'equazione biquadratica}
&a^4+3a^2-4=0\rightarrow a^2=-8\;\text{Non accettabile e }\; x^2=1 
\intertext{da cui}
&x_1=-1\quad\text{e}\quad x_2=1
\intertext{Ricavando i corrispondendi valori di $b$ otteniamo le radici cercate}
&1-2i\quad\text{e}\quad1+2i
\intertext{Sostituendo quei valori nella prima formula}
&z_1=\frac{1-4i+1+2i}{2}=1-i\quad\text{e}\quad z_2=\frac{1-4i-1-2i}{2}=-3i
\intertext{Di conseguenza l'equazione ammette due soluzioni}
\Aboxed{&z_1=1-i\quad\text{e}\quad z_2=-3i}
\end{align*}

\subsection*{\hyperref[sec:insiemi]{Insiemi numerici}}\label{ex:insiemi}
\paragraph{Esercizio 1}
Determinare l'estremo inferiore, superiore, gli eventuali punti di accumulazione del seguente insieme
numerico
\begin{equation*}
A=\left\{x_k\in\mathbb{R}\mid x_k=\frac{1}{1-2^{-k}},\forall k\in\mathbb{N}_0\right\}
\end{equation*}
\divisor

Come vedere se ha estremi? Beh, possiamo ad esempio provare a vedere il grafico di tale funzione
\begin{center}
	\begin{tikzpicture}
		\clip (-3.5,-3.5) rectangle (6,3.5); 
		\tkzInit[xmin=-3,ymin=-3,xmax=3,ymax=3]
		\tkzGrid
		\tkzAxeXY
		\draw[red, thick, domain=-3:3, samples=500] plot({\x}, {1/(1-2^(-\x))});
	\end{tikzpicture}
\end{center}
Da questo vediamo (e anche dalla formula) che descrive un'iperbole equilatera. Essendo l'iperbole una
curva che pu� continuare all'infinito e $k\in\mathbb{N}_0$ quindi ci sono infiniti elementi che 
proseguono in entrambi gli assi. Quindi non ci sono estremi n� superiori n� inferiori. O per scriverlo
in simboli
\begin{equation*}
{]{-\infty},{\infty}[}
\end{equation*}
I punti di accumulazione sono relativamente facili da trovare dal grafico. Dobbiamo trovare un punto
i cui intorni, qualunque essi siano, contengono almeno un elemento dell'insieme. Osservando il disegno
vediamo che $0$ � un punto di accumulazione in quanto sia da destra che da sinistra, diminuendo sempre
pi� la distanza, c'� sempre un elmento dell'insieme.

\paragraph{Esercizio 2}
Dato l'insieme di numeri reali $A=\{x\in\mathbb{R}: x= \arctan n, n\in\mathbb{N}\}$, quale delle
seguenti affermazioni � vera?
\begin{enumerate}
	\item � limitato, ma non ammette massimo
	\item � limitato e ammette sia massimo che minimo
	\item Non ha punti di accumulazione
	\item Tutte le precedenti sono false
\end{enumerate}
\divisor

Come prima cosa, facciamo il grafico.
\begin{center}
	\begin{tikzpicture}
	\tkzInit[xmin=-2,ymin=-2,xmax=2,ymax=2]
	\tkzGrid
	\tkzAxeXY
	\draw[red, thick, domain=-2:2, samples=500] plot({\x}, {rad(atan(\x))});
	\end{tikzpicture}
\end{center}
Come sappiamo, l'arcotangente tender� ad avvicinarsi a $\pm\frac{\pi}{2}$ senza per� effettivamente
raggiungerlo. Quindi non ammette n� un massimo e tantomeno un minimo in quanto per qualunque valore
di $x$ ci sar� sempre un elemento corrispondente dell'insieme. Di conseguenza \textbf{sia il punto 1 
e il punto 2 falso}.\\
Punti di accomulazione ci sono? Guardiamo nuovamente il grafico. Quale punto permette di essere
sicuri che sia da sinitra che da destra ci siano intorni sempre pi� vicini che contengono un punto
dell'insieme. Vediamo che $0$ � un punto di questi (l'unico fra l'altro). Perch� $0$ e non altro?
Prendiamo un altro punto e vediamo che per quanto ci allontaniamo o sforziamo, non garantisce che 
per	ogni punto ce ne sia uno relativo in $A$.

\subsection*{\hyperref[sec:limiti]{Limiti}}\label{ex:limiti}
\paragraph{Esercizio 1}
Verificare che
\begin{equation*}
\lim\limits_{x\to3}\frac{x^2-5x+6}{x^2-9}=\frac{1}{6}
\end{equation*}
\divisor

Consideriamo la disequazione
\begin{align*}
&\left\lvert\frac{x^2-5x+6}{x^2-9}-\frac{1}{6}<\varepsilon\right\rvert
\intertext{che per $x\neq3$ si riduce in}
&\left\lvert\frac{5x-15}{6(x+3)}\right\rvert<\varepsilon
\intertext{Da questa successivamente si ottiene}
&\begin{dcases}
\frac{5x-15}{6(x+3)}<\varepsilon\\\frac{5x-15}{6(x+3)}>-\varepsilon
\end{dcases}\rightarrow\\
&\begin{dcases}
\frac{(5-6\varepsilon)x-15-18\varepsilon}{x+3}<0\\
\frac{(5-6\varepsilon)x-15+18\varepsilon}{x+3}>0
\end{dcases}
\intertext{Risolvendo per $0<\varepsilon<\frac{5}{6}$ si ha}
&\begin{dcases}
-3<x<3\frac{5+6\varepsilon}{5-6\varepsilon}\\
x<-3;\,x>3\frac{5-6\varepsilon}{5+6\varepsilon}
\end{dcases}
\intertext{Che quindi diventa}
&
x\in{\left]3\frac{5-6\varepsilon}{5+6\varepsilon},3\frac{5+6\varepsilon}{5-6\varepsilon}\right[}-\{3\}
\end{align*}
Questo intervallo costituisce a tutti gli effetti un intorno di $3$, escluso $3$ stesso. Quindi il
limite � verificato. (Per verificare sia un intorno, si sotituiscano a $\varepsilon$ valori sempre
pi� vicini a $0$ (ad esempio $0.1$, $0.01$, \ldots) e si verifichi che si avvicinano sempre pi� ad
un valore.\\
La rappresentazione grafica di questo sistema �
\begin{center}
	\begin{tikzpicture}[yscale=0.75]
		\coordinate (A) at (1,0);
		\coordinate (B) at (3,0);
		\coordinate (C) at (4,0);
		
		\coordinate (A1) at (1,-2);
		\coordinate (A2) at (1,-1);
		\coordinate (B1) at (3,-2);
		\coordinate (C1) at (4,-1);
		
		\draw[thick, -stealth] (0,0) -- (5,0)
			node[pos=1,below right]{$x$};
		\draw[thick, cyan] (A1) -- (0,-2);
		\draw[thick, cyan] (A2) -- (C1);
		\draw[thick, cyan] (B1) -- ++(2,0);
		\draw[dashed] (A) -- (A1);
		\draw[dashed] (B) -- (B1);
		\draw[dashed] (C) -- (C1);
		
		\draw[decorate, decoration={zigzag}] (A) -- (A1);
		
		\node[above] at (A){$-3$};
		\node[above] at (B){$3\frac{5-6\varepsilon}{5+6\varepsilon}$};
		\node[above right] at (C){$3\frac{5+6\varepsilon}{5-6\varepsilon}$};
	\end{tikzpicture}
\end{center}

\paragraph{Esercizio 2}
Verificare che
\begin{equation*}
\lim\limits_{x\to0}\ln\abs{x}=-\infty
\end{equation*}
\divisor

Consideriamo per $x\neq0$ la disequazione
\begin{equation*}
\ln\abs{x}<M
\end{equation*}
Se essa risulter� soddisfatta in un intorno di $x_0=0$ per $M<0$, lo sar� certamente per $M\geq0$.\\
Sia dunque $M<0$; si ha
\begin{equation*}
\abs{x}<e^M\quad\text{cio�}\quad-e^M<x<e^M
\end{equation*}
Pertanto il limite risulta verificato perch� le soluzioni costituiscono un intorno privato dello $0$.
L'asse $y$ � asintoto verticale in questo caso.

\paragraph{Esercizio 3}
Determinare il limite di
\begin{equation*}
f(x)=\frac{2}{1-x^3}
\end{equation*}
\divisor

La funzione � definita in $\mathscr{D}_f=\mathbb{R}\setminus\{1\}$. Studiamo il segno per trovare
i limiti sinistri e destri di $x_0=1$ (in quanto � escluso dal dominio).
\begin{equation*}
1-x^3>0\rightarrow x^3<1\rightarrow x<1
\end{equation*}
Quindi quando $x<1$ � positiva, negativa altrimenti.\\
Poich� $\lim\limits_{x\to1}(1-x^3)=0$ si ha
\begin{equation*}
\boxed{\lim\limits_{x\to1^+}\frac{2}{1+x^3}=-\infty}\quad\text{e}\quad
\boxed{\lim\limits{x\to1^-}\frac{2}{1-x^3}=+\infty}
\end{equation*}

\paragraph{Esercizio 4}
Si deduca
\begin{equation*}
\lim\limits_{x\to+\infty}\arctan\ln x
\end{equation*}
\divisor

Deducendo il limite e spezzandolo parte per parte
\begin{align*}
\lim\limits_{x\to+\infty}\ln x \to +\infty
\intertext{Questo lo si scopre anche dal grafico se non ce lo si ricorda}
\lim\limits_{x\to+\infty}\arctan+\infty\to\frac{\pi}{2}
\end{align*}
Questo lo si capisce dal grafico
\begin{center}
	\begin{tikzpicture}
	\tkzInit[xmin=-2,ymin=-2,xmax=2,ymax=2]
	\tkzGrid
	\tkzAxeXY
	\draw[red, thick, domain=-2:2, samples=500] plot({\x}, {rad(atan(\x))});
	\end{tikzpicture}
\end{center}
La funzione � illimitata ma � contenuta all'interno di $\left]{-\frac{\pi}{2}},{\frac{\pi}{2}}\right[$
quindi per una $x$ che cresce sempre di pi�, la funzione tende a $\frac{\pi}{2}$.

\paragraph{Esercizio 5}
Si risolva
\begin{equation*}
\lim\limits_{x\to+\infty}\frac{3x^2-x+5}{4x^2-1}
\end{equation*}
\divisor

Se provassimo a sostituire $x = +\infty$ otterremmo
\begin{equation*}
\lim\limits_{x\to+\infty}\frac{3x^2-x+5}{4x^2-1} = 
\lim\limits_{x\to+\infty}\frac{3\infty^2-\infty+5}{4\infty^2-1} = 
\lim\limits_{x\to+\infty}\frac{\boxed{+\infty-\infty}+5}{\infty-1}
\end{equation*}
notiamo una forma indeterminata al numeratore. Quindi dobbiamo utilizzare qualche teorema. Possiamo
usare i limiti di funzioni razionali. Questo ci dice di guardare i gradi del numeratore e del 
denominatore. Notiamo che entrambi sono $n=m=2$. Quindi il valore di quel limite � pari al
quoziente dei coefficienti
\begin{equation*}
\lim\limits_{x\to+\infty}\frac{3x^2-x+5}{4x^2-1} = \boxed{\frac{3}{4}}
\end{equation*}

%!TEX ROOT=formularioMatematica.tex
\section{Dimostrazioni}
Qui verranno inserite alcune dimostrazioni di teoremi o formule che vengono usate nel formulario.

\begin{proof}[\protect\hyperlink{teor:tfa-ext}{Teorema fondamentale dell'Algebra esteso}]
  Il polinomio $P(x)$ in virtù del teorema fondamentale dell'Algebra, ha in $\mathbb{C}$ almeno uno 
  zero. Indicato con $\alpha_1$ tale zero, risulta:
  \begin{equation*}
    P(x) = (x-\alpha_1)P_1(x)
  \end{equation*}
  essendo il quoziente $P_1(x)$ un polinomio, a coefficienti in $\mathbb{C}$, di grado $(n-1)$.\\
  Se $n-1>0$, allora, per il teorema fondamentale dell'Algebra, anche il polinomio $P_1(x)$ ha in
  $\mathbb{C}$ almeno uno zero. Indicando tale zero con $\alpha_2$ avremo:
  \begin{equation*}
    P_1(x)=(x-\alpha_2)P_2(x)
  \end{equation*}
  essendo il quoziente $P_2(x)$ un polinomio, a coefficienti in $\mathbb{C}$, di grado $(n-2)$.\\
  Risulta quindi:
  \begin{align*}
    P(x)&=\underbrace{(x-\alpha_1)(x-\alpha_2)\dotsm(x-\alpha_n)P_n(x)}_{n\text{ fattori}} = \\
        &(x-\alpha_1)(x-\alpha_2)\dotsm(x-\alpha_n)c
  \end{align*}
  essendo l'ultimo termine di grado zero pari ad una costante $c$.\\
  Poiché la costante $c$ è il coefficiente del termine di grado massimo $x^n$, ne segue che $c=a_n$
  da cui
  \begin{equation*}
    P(x) = a_n(x-\alpha_1)(x-\alpha_2)\dotsm(x-\alpha_n)
  \end{equation*}
\end{proof}

\begin{proof}
  [\protect\hyperlink{teor:limiteInfinitoFunzRaz}{Limite di una funzione razionale}]
  Se
  \begin{equation*}
    P(x)=a_nx^n+a_{n-1}x^{n-1}+\dotsb+a_0
  \end{equation*}
  è un polinomio di grado $n>0$, si può scrivere per $x\neq0$
  \begin{equation*}
    P(x) = x^n\left(a_n+\frac{a_{n-1}}{x}+\dotsb+\frac{a_0}{x^n}\right)
  \end{equation*}
  e quindi, poiché $\lim\limits_{x\to+\infty}\frac{1}{x^n}=0,\,\forall n\in\mathbb{N}_0$, risulta
  \begin{equation*}
    \lim\limits_{x\to\infty}x^n\left(a_n+\frac{a_{n-1}}{x}+\dotsb+\frac{a_0}{x^n}\right) = a_n
  \end{equation*}
  Si ha
  \begin{align*}
    \lim\limits_{x\to\infty}P(x)&=\lim\limits_{x\to\infty}=
    \lim\limits_{x\to\infty}\left(a_nx^n+a_{n-1}x^{n-1}+\dotsb+a_0\right)=\\
    &\lim\limits_{x\to\infty}\left(a_nx^n\right)
  \end{align*}
\end{proof}

\begin{proof}[\protect\hyperlink{teor:uniLim}{Unicità del limite}]
  Supponiamo per assurdo che la funzione $f$ per $x\to x_0$ ammetta due limiti distinti $l_1$ e 
  $l_2$, cioè che si abbia
  \begin{equation*}
    \lim\limits_{x\to x_0}f(x)=l_1\quad\lim\limits_{x\to x_0}f(x)=l_2
  \end{equation*}
  In base alla definizione di limite, preso comunque un numero $\varepsilon>0$, è possibile 
  determinare due numeri positivi $\delta_\varepsilon'$ e $\delta_\varepsilon''$ tali che, per ogni
  $x\in\mathscr{D}_f$, verificante la condizione
  \begin{align*}
    0&<\abs{x-x_0}<\delta_\varepsilon' \quad&\text{risulti}&\quad\abs{f(x)-l_1}<\varepsilon\\
    0&<\abs{x-x_0}<\delta_\varepsilon'' &\text{risulti}&\abs{f(x)-l_2}<\varepsilon\\
  \end{align*}
  Ora, sia $\delta_\varepsilon$ il minore tra i due numeri $\delta_\varepsilon',\delta_\varepsilon''$
  per
  \begin{equation*}
    0<\abs{x-x_0}<\delta_\varepsilon
  \end{equation*}
  risulteranno verificate entrambe le disequazioni precedenti e potremo scrivere
  \begin{equation*}
    \abs{l_1-l_2}=\abs{l_1-f(x)+f(x)-l_2}\leq\abs{f(x)-l_1}+\abs{f(x)-l_2}<\varepsilon+\varepsilon=
    \varepsilon2
  \end{equation*}
  Data l'arbitrarietà di $\varepsilon$, la condizione $\abs{l_1-l_2}<2\varepsilon$ implica che sia
  $\abs{l_1-l_2}=0$ cioè $l_1=l_2$.
\end{proof}

\begin{proof}[\protect\hyperlink{teor:confLim}{Teorema del confronto}]
  In base alla definizione di limite, preso comunque un numero $\varepsilon>0$, è possibile 
  determinare due numeri positivi $\delta_\varepsilon'$ e $\delta_\varepsilon''$ tali che, per ogni
  $x\in\mathscr{D}_f$, verificante la condizione
  \begin{alignat*}{2}
    0&<\abs{x-x_0}<\delta_\varepsilon' &\quad\text{risulti}\quad \abs{f(x)-l_1}<\varepsilon\\
    0&<\abs{x-x_0}<\delta_\varepsilon'' &\text{risulti} \abs{f(x)-l_2}<\varepsilon\\
  \end{alignat*}
  Ora, sia $\delta_\varepsilon$ il minore tra i due numeri $\delta_\varepsilon',\delta_\varepsilon''$
  per
  \begin{equation*}
    0<\abs{x-x_0}<\delta_\varepsilon
  \end{equation*}
  saranno verificate entrambe le disequazioni precedenti quindi
  \begin{equation*}
    l-\varepsilon<f(x)\leq g(x)\leq h(x)<l+\varepsilon
  \end{equation*}
  cioè
  \begin{equation*}
    \abs{g(x)-l}<\varepsilon
  \end{equation*}
\end{proof}

\begin{proof}[\protect\hyperlink{teor:segno}{Teorema della permanenza del segno}]
  Dimostriamo innanzitutto la prima parte del teorema.\\
  Sia $\varepsilon=\frac{\abs{l}}{2}$; per la definizione di limite è possibile determinare in 
  corrispondenza di tale $\varepsilon$, un numero $\delta_\varepsilon>0$ tale che se 
  $x\in\mathscr{D}_f$
  \begin{equation*}
    0<\abs{x-x_0}<\delta_\varepsilon\quad\text{implichi}\quad
    l-\frac{\abs{l}}{2}<f(x)<l+\frac{\abs{l}}{2}
  \end{equation*}
  Ne consegue la tesi non appena si osservi che
  \begin{equation*}
    \text{se}\, l<0\quad l+\frac{\abs{l}}{2}<0\quad\text{quindi}\quad f(x)<0
  \end{equation*}
  \begin{equation*}
    \text{se}\, l>0\quad l-\frac{\abs{l}}{2}>0\quad\text{quindi}\quad f(x)>0
  \end{equation*}
  Dimostriamo ora la seconda parte.\\
  Sia per esempio $f(x)>0$. Dalla definizione di limite è possibile determinare in 
  corrispondenza di tale $\varepsilon$, un numero $\delta_\varepsilon>0$ tale che se 
  $x\in\mathscr{D}_f$
  \begin{equation*}
    0<\abs{x-x_0}<\delta_\varepsilon\quad\text{implichi}\quad
    l-\varepsilon<f(x)<l+\varepsilon
  \end{equation*}
  Supponiamo ora per assurdo che sia $l<0$, scegliendo $\varepsilon=-\frac{l}{2}>0$; si avrebbe
  \begin{equation*}
    f(x)<\frac{l}{2}<0
  \end{equation*}
  contro l'ipotesi che sia $f(x)>0$. Sarà dunque $l\geq0$
\end{proof}

\begin{proof}[\protect\hyperlink{teor:sommaLimiti}{Limite di una somma}]
  Si ha intanto
  \begin{equation*}
    \abs{[f(x)+g(x)]}-(l_1+l_2)\leq\abs{f(x)-l_1}+\abs{g(x)-l)2}
  \end{equation*}
  e quindi, preso $\varepsilon>0$, se si vuol provare che il primo membro è più piccolo di 
  $\varepsilon$, basta verificare che ciascuno dei due addendi a secondo membro è più piccolo di
  $\frac{\varepsilon}{2}$.\\
  Ma questo è evidente per le definizioni stesse di limiti. Il primo addendo sarà minore di 
  $\frac{\varepsilon}{2}$ se
  \begin{equation*}
    0<\abs{x-x_0}<\delta_\varepsilon'
  \end{equation*}
  e il secondo se
  \begin{equation*}
    0<\abs{x-x_0}<\delta_\varepsilon''
  \end{equation*}
  ove i due numeri $\delta_\varepsilon'$ e $\delta_\varepsilon''$ possono essere diversi in quanto 
  si riferiscono a funzioni diverse.\\
  Detto allora $\delta_\varepsilon$ il minore dei due, scegliendo $x$ tale che
  \begin{equation*}
    0<\abs{x-x_0}<\delta_\varepsilon
  \end{equation*}
  si soddisfano entrambe le condizioni; quindi per valori di $x$ così scelti si avrà
  \begin{equation*}
    \abs{f(x)-l_1}<\frac{\varepsilon}{2}\quad\abs{g(x)-l_2}<\frac{\varepsilon}{2}
  \end{equation*}
  e di conseguenza
  \begin{equation*}
    \abs{[f(x)+g(x)]}-(l_1+l_2)<\varepsilon
  \end{equation*}
\end{proof}

\begin{proof}[\protect\hyperlink{teor:prodottoLimiti}{Limite di un prodotto}]
  Si ha
  \begin{align*}
    &\abs{f(x)\cdot g(x)-l_1\cdot l_2} = \\
    &\abs{f(x)\cdot g(x)+l_1\cdot g(x)-l_1\cdot g(x)-l_1\cdot l_2} =\\
    &\abs{g(x)\cdot(f(x)-l_1)+l_1\cdot(g(x)-l_2)}\leq\\
    &\abs{g(x)}\cdot\abs{f(x)-l_1}+\abs{l_1}\cdot
    \abs{g(x)-l_2}
  \end{align*}
  Fissato allora $\varepsilon'$ in modo che sia $0<\varepsilon'<1$, esiste in corrispondenza di esso 
  un numero positivo $\delta_{\varepsilon'}$ tale che, $\forall x\in I$ verificante la condizione
  $0<\abs{x-x_0}<\delta_{\varepsilon'}$, risulti
  \begin{equation*}
    \abs{f(x)-1}<\varepsilon'\quad\abs{g(x)-l_2}<\varepsilon'\quad\abs{g(x)}<\abs{l_2}+\varepsilon'
  \end{equation*}
  Si ricava quindi
  \begin{equation*}
    \abs{f(x)\cdot g(x)-l_1\cdot l_2}<(\abs{l_2}+\varepsilon')\varepsilon'+\abs{l_2}\varepsilon'<
    (\abs{l_2}+\abs{l_1}+1)\varepsilon'
  \end{equation*}
  poiché $\varepsilon'^2<\varepsilon'$, essendo $0<\varepsilon'<1$, se scegliamo $\varepsilon'$ non
  solo positivo e minore di $1$ ma anche minore di
  \begin{equation*}
    \frac{\varepsilon}{\abs{l_1}+\abs{l_2}+1}
  \end{equation*}
  si ottiene, per $x$ appartenente ad un opportuno intorno di $x_0$
  \begin{equation*}
    \abs{f(x)\cdot g(x)-l_1\cdot l_2}<\varepsilon
  \end{equation*}
\end{proof}

\begin{proof}[\protect\hyperlink{teor:rolle}{Teorema di Rolle}]
  \textbf{Ipotesi}:
  \begin{itemize}
    \item $f$ definita e continua in $[a,b]$
    \item $f$ derivabile in $]a,b[$
    \item $f(a)=f(b)$
  \end{itemize}
  \textbf{Tesi}:
  \begin{equation*}
    \exists\,c\in{]a,b[}\suchthat f'(c)=0
  \end{equation*}
  \divisor

  Per il teorema di Weistrass la $f$ ammette $\max$ e $\min$ assoluti.
  \begin{equation*}
    m = \min_{x\in{[a,b]}} f \quad M = \max_{x\in{[a,b]}}f
  \end{equation*}
  Si distinguono due casi
  \begin{enumerate}
    \item I punti di $\max$ e/o $\min$ coincidono con gli estremi
      \begin{equation*}
        m\leq f(x)\leq M
      \end{equation*}
      Visto che
      \begin{equation*}
        f(a) = f(b)
      \end{equation*}
      si ha che
      \begin{equation*}
        m = f(x)
      \end{equation*}
      Quindi
      \begin{equation*}
        f'(x) = 0\quad\forall x \in{[a,b]}
      \end{equation*}
    \item Almeno uno fra $\max$ e $\min$ sono interni all'intervallo ${[a,b]}$
      \begin{equation*}
        f(c) = M
      \end{equation*}
      Visto che in $c$, $\exists\,\max f$ e $f$ è derivabile,
      \begin{equation*}
        f'(c) = 0
      \end{equation*}
  \end{enumerate}
\end{proof}

\begin{proof}[\protect\hyperlink{teor:lagrange}{Teorema di Lagrange}]
  \textbf{Ipotesi}:
  \begin{itemize}
    \item $f$ definita e continua in $[a,b]$
    \item $f$ derivabile in $]a,b[$
  \end{itemize}
  \textbf{Tesi}:
  \begin{equation*}
    \exists\,x\in[a,b]\suchthat f'(x) = \frac{f(b)-f(a)}{b-a}
  \end{equation*}
  \divisor

  Sia $\phi(x) = f(x)-kx$ per una costante $k$ in modo che abbia le stesse caratteristiche di $f$,
  ovvero è continua ($f(x)$ è continua per ipotesi, $kx$ è continua perché polinomiale) e 
  derivablie ($\phi'(x) = f'(x)-k$).\\
  Per ricondursi a Rolle
  \begin{equation*}
    \phi(a) = \phi(b)
  \end{equation*}
  quindi
  \begin{align*}
    f(a)-ak &= f(b)-kb\\
    kb-ka &= f(b)-f(a)\\
    k(b-a) &= f(b)-f(a)\\
    k &= \frac{f(b)-f(a)}{b-a}
  \end{align*}
  Dato che il Teorema di Rolle ci dice
  \begin{equation*}
    \exists\,x_0\in[a,b]\suchthat f'(x)=0
  \end{equation*}
  si ha che
  \begin{equation*}
    \phi'(x) = f'(x)-\frac{f(b)-f(a)}{b-a} = 0
  \end{equation*}
  da cui si deriva che
  \begin{equation*}
    f'(x) = \frac{f(b)-f(a)}{b-a}
  \end{equation*}
\end{proof}

\begin{proof}
  [\protect\hyperlink{teor:lagrange:1}{Monotonia $\leftrightarrow$ Crescenza/Decrescenza}]
  \textbf{Ipotesi}:
  \begin{itemize}
    \item $f$ definita e continua in $[a,b]$
    \item $f$ derivabile in $]a,b[$
  \end{itemize}
  \textbf{Tesi}:
  \begin{itemize}
    \item Se $f'(x) > 0$, allora la funzione è crescente
    \item Se $f'(x) < 0$, allora la funzione è decrescente
  \end{itemize}
  \divisor

  Preso un intorno $I=[a,b]$ possiamo prendere un altro intorno $[x_1,x_2]\in I$. Per il teorema
  di Lagrange,
  \begin{equation*}
    \exists\,c\in]x_1,x_2[\suchthat \frac{f(x_2)-f(x_1)}{x_2-x_1}=f'(c)
  \end{equation*}
  Si distinguono i due casi
  \begin{description}
    \item[$f'(c)>0$] allora $x_2-x_1>0 \implies f(x_2)-f(x_1)>0 \implies\,\text{crescente}$ 
    \item[$f'(c)<0$] allora $x_2-x_1<0 \implies f(x_2)-f(x_1)<0\implies\,\text{decrescente}$
  \end{description}
\end{proof}

\begin{proof}
  [\protect\hyperlink{teor:lagrange:2}{Costanza}]
  \textbf{Ipotesi}: 
  \begin{itemize}
    \item $f$ definita e continua in $[a,b]$
    \item $f$ con derivata nulla in $]a,b[$
  \end{itemize}
  \textbf{Tesi}:
  \begin{equation*}
    f(x) = k
  \end{equation*}
  \divisor

  Si prenda un intorno $]x_1,x_2[\in I$, allora per il teorema di Lagrange
  \begin{equation*}
    \exists\,c\in]x_1,x_2[\suchthat \frac{f(x_2)-f(x_1)}{x_2-x_1}=f'(c)=0
  \end{equation*}
  per ipotesi. Quindi si ha che
  \begin{equation*}
    f(x_2) - f(x_1) = 0 \implies f(x_2) = f(x_1)
  \end{equation*}
\end{proof}

\begin{proof}
  [\protect\hyperlink{teor:der:Sec:1}{Massimi e minimi e flessi con derivata seconda}]
  \textbf{Ipotesi}:
  \begin{itemize}
    \item $f$ sia continua e derivabile almeno 3 volte
  \end{itemize}
  \textbf{Tesi}:
  \begin{equation*}
    f'(x_0)=f''(x_0)=\dotsb=f^{(n-1)}(x_0)=0\land f^{(n)}(x_0)\neq0
  \end{equation*}
  \begin{enumerate}
    \item Se $n$ pari è un massimo o un minimo a seconda del segno
    \item Se $n$ dispari è un flesso a tangente orizzontale
  \end{enumerate}
  \divisor

  Dimostriamo il punto 1). Se $f''(x)$ è continua in $x_0$ allora
  \begin{equation*}
    \lim\limits_{x\to x_0} f''(x) = f''(x_0)[>0]
  \end{equation*}
  Per il teorema della permanenza del segno
  \begin{equation*}
    \exists\,I(x_0)\suchthat f''(x)>0
  \end{equation*}
  Possiamo riscrivere
  \begin{equation*}
    f''(x) = \Dif[f'(x)]
  \end{equation*}
  Dato che sappiamo che $f''(x)$ è maggiore di zero, significa che $f'(x)$ è crescente nell'intorno
  per il primo Lemma del teorema di Lagrange. Se $f'(x_0) = 0$ come da ipotesi significa che questa
  funzione interseca l'asse delle $x$. Essendo crescente significa che il segno è
  \begin{center}
    \begin{tikzpicture}
      \drawSign{\x}{-1}{1}{0.25}{}%
      \end{tikzpicture}
  \end{center}
  e che quindi la funzione $f(x)$ di cui è derivata è prima decrescente e poi crescente in modo
  da avere una tangente orizzontale a $x_0$. Ciò significa che in $x_0$ c'è un minimo relativo. Per
  il massimo, si dimostra in modo analogo.\\ [\baselineskip]
  Dimostriamo il punto 2). Se $f''(x_0)=0\land f'''(x_0)>0$ si ha che 
  $\Dif[f'(x_0)]=0\land\Dif[f''(x_0)]>0$. Da ciò si deduce che in $x_0$, $f'(x)$ ha un punto di
  minimo relativo. Dato che $f'$ è sempre positivo, significa che $f$ è crescente e in $x_0$ ha
  una tangente orizzontale. Quindi il punto $x_0$ è un punto di flesso per $f$.
\end{proof}

\begin{proof}
  [\protect\hyperlink{teor:derSec:2}{Concavità con derivata seconda}]
  \textbf{Ipotesi}:
  \begin{itemize}
    \item $f$ sia continua e derivabile in $]a,b[$
    \item Esiste $f''(x)$
  \end{itemize}
  \textbf{Tesi}:
  \begin{description}
    \item[Se $f''(x_0)>0$] ha concavità verso l'alto
    \item[Se $f''(x_0)<0$] ha concavità verso il basso
    \item[Se $f''(x_0)=0$ e $f'''(x_0)\neq0$] in $x_0$ ha un flesso
  \end{description}
  \divisor

  Dimostriamo il primo punto, gli altri due si fanno in modo analogo. Avendo concavità verso l'alto
  significa che
  \begin{equation*}
    \exists\,I(x_0)\suchthat\,\forall x\in I(x_0) \Rightarrow f(x)-t(x)\geq0
  \end{equation*}
  Definiamo ora $\varphi(x) = f(x)-t(x)$, ovvero
  \begin{equation*}
    \varphi(x) = f(x)-t(x) = f(x)-f(x_0)-f'(x_0)(x-x_0)
  \end{equation*} 
  Andando a calcolare le prime derivate vediamo che
  \begin{equation*}
    \varphi'(x) = f'(x)-f'(x_0)\quad\varphi''(x)=f''(x)
  \end{equation*}
  Ponendo ora $\varphi(x_0) = 0$, vediamo che
  \begin{equation*}
    \varphi'(x_0)=0\quad\varphi''(x_0)=f''(x_0)>0
  \end{equation*}
  Dato che $\varphi'(x_0)=0$, significa che $x_0$ è un punto di minimo, di massimo o di flesso a
  tangente orizzontale per $\varphi$. Dato che la derivata seconda è positiva, significa che è
  un punto di minimo, perciò
  \begin{equation*}
    \exists\,I(x_0)\suchthat\varphi(x)\geq\varphi(x_0)\geq0 \Rightarrow f(x)-t(x)\geq 0
  \end{equation*}
\end{proof}

\begin{proof}
  [\protect\hyperlink{teor:media}{Teorema del valor medio}]
  \textbf{Ipotesi}:
  \begin{itemize}
    \item $f$ continua e definita in $[a,b]$
  \end{itemize}
  \textbf{Tesi}:
  \begin{itemize}
    \item $\exists\,c\in[a,b]\suchthat f(c)=\frac{1}{b-a}\int\limits_{a}^{b} f(x)\,\dif x$
  \end{itemize}
  \divisor

  Se $f$ è continua in $[a,b]$, per il teorema di Weistrass ammette $m=\min f$ e $M=\max f$.
  \begin{align*}
    m\leq &f(x) \leq M\\
    \int\limits_{a}^{b} m\,\dif x\leq
    &\int\limits_{a}^{b}f(x)\,\dif x\leq\int\limits_{a}^{b}M\,\dif x\\
    \left.mx\right|_a^b\leq &\int\limits_{a}^{b} f(x)\,\dif x\leq \left.Mx\right|_a^b\\
    m(b-a)\leq &\int\limits_{a}^{b} f(x)\,\dif x\leq M(b-a)\\
    m\leq &\frac{1}{b-a}\int\limits_{a}^{b} f(x)\,\dif x\leq M
  \end{align*}
  Ora per il teorema dei valori intermedi
  \begin{equation*}
    \exists\,c\in[a,b]\suchthat f(c)=\frac{1}{b-a}\int\limits_{a}^{b} f(x)\,\dif x
  \end{equation*}
\end{proof}

\begin{proof}
  [\protect\hyperlink{teor:tfci}{Teorema fondamentale del calcolo integrale}]
  \textbf{Ipotesi}:
  \begin{itemize}
    \item $f$ continua in $[a,b]$
  \end{itemize}
  \textbf{Tesi}:
  \begin{itemize}
    \item $F(x)$ derivabile per ogni $x\in[a,b]$
    \item $F'(x)=f(x)$ e $F(a)=0$
  \end{itemize}
  \divisor

  Se $F(x)$ è derivabile, significa che
  \begin{equation*}
    \exists\,\lim\limits_{h\to0} \frac{F(x+h)-F(x)}{h}
  \end{equation*}
  Abbiamo quindi che
  \begin{equation*}
    F(x) = \int\limits_{a}^{x} f(x)\,\dif x
  \end{equation*}
  e
  \begin{equation*}
    F(x+h) = \int\limits_{a}^{x+h} f(x)\,\dif x = \int\limits_{a}^{x} f(x)\,\dif x + 
    \int\limits_{x}^{x+h} f(x)\,\dif x
  \end{equation*}
  Il numeratore quindi diventa
  \begin{equation*}
    F(x+h)-F(x) = \cancel{\int\limits_{a}^{x} f(x)\,\dif x}+\int\limits_{x}^{x+h} f(x)\,\dif x
    -\cancel{\int\limits_{a}^{x} f(x)\,\dif x}
  \end{equation*}
  Possiamo quindi scrivere
  \begin{equation*}
    \frac{1}{h}\int\limits_{x}^{x+h} f(x)\,\dif x
  \end{equation*}
  Se è continua in $[a,b]$, deve esserlo anche in $[x,x+h]\subseteq[a,b]$.\\
  Per il teorema della media
  \begin{equation*}
    \exists c\in[x,x+h]\suchthat f(c)=\frac{1}{h}\int\limits_{x}^{x+h} f(x)\,\dif x
  \end{equation*}
  e quindi
  \begin{equation*}
    \frac{1}{h}\int\limits_{x}^{x+h} f(x)\,\dif x=\frac{1}{h}h f(c) = f(c)
  \end{equation*}
  Quindi infine possiamo scrivere
  \begin{equation*}
    \overbrace{\lim\limits_{h\to0} \frac{F(x+h)-F(x)}{h}}^{\mathclap{F'(x)}}=
    \lim\limits_{h\to0}f(x)=\overbrace{\lim\limits_{c\to x}f(c)}^{\mathclap{\text{Continua}}}=f(x)
  \end{equation*}
  E quindi
  \begin{equation*}
    F'(x) = f(x)
  \end{equation*}
\end{proof}

\begin{proof}
  [\protect\hyperlink{teor:deriv}{Teorema del criterio di derivabilità}]
  \textbf{Ipotesi:}
  \begin{itemize}
    \item $f(x)$ continua in $x_0$
    \item $f(x)$ derivabile in $U\setminus\{x_0\}$
  \end{itemize}
  \textbf{Tesi:}
  \begin{itemize}
    \item $f'(x_0)=\lim\limits_{x\to x_0} f'(x)$
  \end{itemize}
  \divisor

  Per definizione
  \begin{equation*}
    f'(x_0) = \lim\limits_{x\to x_0} \frac{f(x)-f(x_0)}{x-x_0}
  \end{equation*}
  Dato che $f$ è continua si può dire che
  \begin{equation*}
    \lim\limits_{x\to x_0} f(x) = f(x_0)
  \end{equation*}
  quindi, risolvendo la forma indeterminata $^0\!/_0$ usando l'Hôpital
  \begin{equation*}
    \lim\limits_{x\to x_0} \frac{f(x)-f(x_0)}{x-x_0}\Heq{\frac{0}{0}}\lim\limits_{x\to x_0} f'(x)
  \end{equation*}
  Se questo limite esiste ed è finito, si può scrivere l'equivalenza
  \begin{equation*}
    f'(x_0)\coloneqq\lim\limits_{x\to x_0} \frac{f(x)-f(x_0)}{x-x_0}=\lim\limits_{x\to x_0} f'(x)
  \end{equation*}
\end{proof}


% List of TODOs
\newpage
\listoftodos[Note]

\end{document}
