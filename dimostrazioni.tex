%!TEX ROOT=formularioMatematica.tex
\section{Dimostrazioni}
Qui verranno inserite alcune dimostrazioni di teoremi o formule che vengono usate nel formulario.

\begin{proof}[Teorema fondamentale dell'Algebra esteso]
	Il polinomio $P(x)$ in virt� del teorema fondamentale dell'Algebra, ha in $\mathbb{C}$ almeno uno 
	zero. Indicato con $\alpha_1$ tale zero, risulta:
	\begin{equation*}
	P(x) = (x-\alpha_1)P_1(x)
	\end{equation*}
	essendo il quoziente $P_1(x)$ un polinomio, a coefficienti in $\mathbb{C}$, di grado $(n-1)$.\\
	Se $n-1>0$, allora, per il teorema fondamentale dell'Algebra, anche il polinomio $P_1(x)$ ha in
	$\mathbb{C}$ almeno uno zero. Indicando tale zero con $\alpha_2$ avremo:
	\begin{equation*}
	P_1(x)=(x-\alpha_2)P_2(x)
	\end{equation*}
	essendo il quoziente $P_2(x)$ un polinomio, a coefficienti in $\mathbb{C}$, di grado $(n-2)$.\\
	Risulta quindi:
	\begin{align*}
	P(x)&=\underbrace{(x-\alpha_1)(x-\alpha_2)\dotsm(x-\alpha_n)P_n(x)}_{n\text{ fattori}} = \\
	&(x-\alpha_1)(x-\alpha_2)\dotsm(x-\alpha_n)c
	\end{align*}
	essendo l'ultimo termine di grado zero pari ad una costante $c$.\\
	Poich� la costante $c$ � il coefficiente del termine di grado massimo $x^n$, ne segue che $c=a_n$
	da cui
	\begin{equation*}
	P(x) = a_n(x-\alpha_1)(x-\alpha_2)\dotsm(x-\alpha_n)
	\end{equation*}
\end{proof}