%!TEX ROOT=formularioMatematica.tex

\section{Distribuzioni}
Il concetto di fondo che sta alla base delle distribuzioni sono le \textbf{variabili casuali}. Ci 
sono due tipi di variabili casuali: discrete e continue e con esse due tipi di distribuzioni. Le
variabili si indicano con una $X$, maiuscola.

\subsection{Distribuizioni discrete}
Una variabile discreta è una variabile che può assumere un numero finito o numerabile di valori. Ad
ogni variabile si associa una certa probabilità. Spesso si impostano le variabili sottoforma di 
tabella in modo da evidenziare ciascuna probabilità
\begin{center}
  \begin{tabular}{c c c c c}
    $X:$ & $X_1$ & $X_2$ & $\cdots$ & $X_n$\\midrule
    $p:$ & $p_1$ & $p_2$ & $\cdots$ & $p_n$
  \end{tabular}
\end{center}
È ovvio che $\sum p_i = 1$.\\ [\baselineskip]
Si definiscono due funzioni: \textbf{funzione di distribuzione di probabilità} e di i
\textbf{ripartizione di probabilità}, rispettivamente
\begin{equation*}
  f(x_i) = P(X=x_i)
\end{equation*}
e
\begin{equation*}
  F(x) = P(X\leq x)\quad\forall x\in\mathbb{R}
\end{equation*}
È da notare che per il modo in cui sono definite, si hanno i seguenti due grafici
\begin{center}
  \begin{tikzpicture}[scale=0.75]
    \begin{axis}[xmin=0,ymin=0,xmax=4,ymax=4,ticks=none,ylabel=$f(x)$]
      \draw[thick] (1,0) -- (1,1);
      \draw[thick] (2,0) -- (2,3);
      \draw[thick] (3,0) -- (3,2);
    \end{axis}
  \end{tikzpicture}
\end{center}
e così via per tutti gli $x_i$.
\begin{center}
  \begin{tikzpicture}[scale=0.75]
    \begin{axis}[xmin=0,ymin=0,xmax=4,ymax=4,ticks=none,ylabel=$F(x)$]
      \draw[thick] (1,1) circle (0.05);
      \draw[thick] (1,1) -- (2,1); 
      \draw[thick] (2,1) circle (0.05);
      
      \draw[thick] (2,2) circle (0.05);
      \draw[thick] (2,2) -- (3,2);
      \draw[thick] (3,2) circle (0.05);

      \draw[thick] (3,3) circle (0.05);
      \draw[thick] (3,3) -- (4,3);
      \draw[thick] (4,3) circle (0.05);
    \end{axis}
  \end{tikzpicture}
\end{center}
Da questi grafici si evincono alcune cose:
\begin{enumerate}
  \item $F(x)=P(X\leq x)=\sum\limits^{n}_{x_i<x} f(x_i)$
  \item $f(x_i)\geq0\quad\forall x_i$
  \item $\sum f(x_i)=1$
  \item $F(x)$ è monotona non decrescente
  \item $0\leq F(x_i)\leq1$
\end{enumerate}
Si definiscono anche altre due funzioni estremamente usate: il valore medio (o indice di media o
valore aspettato) che è definita come
\begin{equation*}
  E(x) = \sum\limits^{n}_{i=1} x_if(x_i)
\end{equation*}
e la varianza
\begin{equation*}
  \sigma^2(x) = \sum\limits^{n}_{i=1} [x_i-E(x_i)]^2f(x_i) = E(X^2)-E^2(X)
\end{equation*}
