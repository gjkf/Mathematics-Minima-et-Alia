%!TEX ROOT=formularioMatematica.tex

\section{Funzioni continue}\label{sec:funzCont}

Una funzione $f$ si dice continua o in un punto ($x_0$) o in un intervallo ($I$). La continuit�in
un punto si ha se
\begin{equation*}
  \lim\limits_{x\to x_0}f(x) = f(x_0) 
\end{equation*}
La definizione formale quindi diventa
\begin{equation*}
  \forall\varepsilon>0,\exists\,\delta_\varepsilon>0\mid\forall x:\,\abs{x-x_0}<\delta_\varepsilon
  \Rightarrow \abs{f(x)-f(x_0)}<\varepsilon
\end{equation*}
che, se confrontata con la definizione di limite manca di un $0<$ in quanto la funzione � continua,
quindi $x_0\in\mathscr{C}(f)$.\\
La continuit� in un intervallo � una generalizzazione di quella di un punto, ovvero una funzione
� continua in un intervallo se
\begin{equation*}
  \forall x_0\in I\,f\text{ � continua in }x_0\Rightarrow f\,\text{� continua in } I
\end{equation*}

\subsection{Propriet� delle funzioni continue}
Nelle funzioni continue si mantengono i teoremi dei limiti. Ovvero siano $f$ e $g$ due funzioni.
Se entrambe sono continue in $x_0$
\begin{align*}
  f\pm g\,&\text{� continua in }x_0\\
  f\circ g\,&\text{� continua in }x_0\\
  \frac{f}{g}\,&\text{� continua in }x_0\quad \big(g(x_0)\big)\neq0
\end{align*}
Da queste propriet� ricaviamo subito che qualsiasi funzione polinomiale � continua. Questo perch�,
sia $P(x)$ una funzione polinomiale del tipo $a_nx^n+\dotsb+a_0$. La funzione costante ($y=a$) �
continua in quanto non dipende da alcuna variabile. Per $x^n$ possiamo pensarlo come $x\cdot
x\underbrace{\cdot\dotsb\cdot}_{n-\text{volte}}x$. Considerato le precedenti relazioni ed essendo
$y=x$ continua in quanto il suo dominio � tutto $\mathbb{R}$, il prodotto di funzioni
continue � un'altra funzione continua. La somma di funzioni continue � un'altra funzione continua.
Quindi \textbf{ogni funzione polinomiale � continua in ogni $x_0$}.

\subsection{Punti di discontinuit�}
Una funzione pu� essere continua solo se
\begin{equation*}
  \lim\limits_{x\to x_0}f(x)=f(x_0) 
\end{equation*}
e questo limite esiste, la funzione � definita in $x_0$ e $l=f(x_0)$. Ovviamente ci sono casi in
cui queste tre caratteristiche non si verificano. Ecco che si classificano quindi i punti di
discontinuit�, ovvero quei $x_0$ in cui la funzione pecca di queste particolarit�.

\subsubsection{Prima specie}
Si ha quando
\begin{equation*}
  \lim\limits_{x\to x_0^-}f(x) \neq \lim\limits_{x\to x_0^+}f(x) 
\end{equation*}
Questa specie � anche definita "salto" in quanto visualmente si ha un salto della funzione. Ad
esempio
\begin{center}
  \begin{tikzpicture}
    \tkzInit[xmin=-2,ymin=-2,xmax=2,ymax=2]
    \tkzGrid
    \tkzAxeXY
    
    \draw[thick, red] plot[domain=-2:0] (\x, \x+1);
    \draw[thick, red] plot[domain=0:2] (\x,\x-2);
  \end{tikzpicture}
\end{center}
ha una discontinuit� di prima specie in quanto ha un "salto" nella funzione e in $x_0=0$ la 
funzione non � continua.

\subsubsection{Seconda specie}
La seconda specie si ha quando
\begin{equation*}
  \lim\limits_{x\to x_0^-}f(x) = \pm\infty \lor \lim\limits_{x\to x_0^+}f(x) = \pm\infty
\end{equation*}
Un tipico esempio pu� essere la funzione tangente o un'iperbole equilatera.

\subsubsection{Terza specie}
Avviene quando
\begin{equation*}
  \not\exists f(x_0) \lor \lim\limits_{x\to x_0} f(x) \neq f(x_0)
\end{equation*}
Questa specie viene anche definita "eliminabile" in quanto si pu� trovare una funzione 
$\tilde{f}$ che contenga al suo interno anche un punto. Ad esempio
\begin{equation*}
  \tilde{f} =
  \begin{cases}
    x_0,\,& x = x_0\\
    f(x),& x\neq x_0
  \end{cases}
\end{equation*}
