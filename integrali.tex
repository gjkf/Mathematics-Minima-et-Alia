%!TEX ROOT=formularioMatematica.tex

\section{Integrali}
\subsection{Integrali indefiniti}
Sia $f$ una funzione continua in $[{a,b}]$. Allora $f'(x)$ � la sua derivata, $F(x)$ � una primitiva.
Bisogna quindi definire una primitiva di una funzione
\begin{primitiva}
  Una funzione $F(x)$ si dice primitiva della funzione $f(x)$ continua in $[{a,b}]$ se
  \begin{equation*}
    F'(x) = f(x)
  \end{equation*}
\end{primitiva}
Sia ad esempio $f(x) = \cos x$, $F(x)$ sar� allora quella funzione la cui derivata � $f(x)$. Quindi
$F(x)=\sin x$. Ma � soltanto questa? No, infatti anche $\sin x +1$ o $\sin x -\frac{e}{4}$ o 
qualsiasi altra funzione che abbia una costante. Cos� possiamo definire un insieme denominato
\textbf{totalit� delle primitive} che le raccoglie tutte. L'operatore che permette di trovare questo
insieme � \textbf{l'integrale indefinito}
\begin{equation*}
  \int f(x)\,\dif x = \{F(x)+c\}
\end{equation*}
Scrivere $\dif x$ � necessario perch�, come nella scrittura di Leibniz per le derivate, indica per
quale lettera si deve integrare.

\subsubsection{Propriet� dell'integrale indefinito}
Per la definizione stessa di integrale si ha che
\begin{equation*}
  \Dif\int f(x)\,\dif x = f(x)
\end{equation*}
e
\begin{equation*}
  \int\Dif f(x)\,\dif x = f(x)+c
\end{equation*}
Se $f(x)$ � una funzione continua e $k$ una costante, si ha
\begin{equation*}
  \int kf(x)\,\dif x = k\int f(x)\,\dif x
\end{equation*}
Se si ha una una somma di funzioni $\sum f$,
\begin{equation*}
  \int\sum\limits^{n}_{i=1} f_i(x)\,\dif x = \sum\limits^{n}_{i=1} \int f_i(x)\,\dif x
\end{equation*}

\subsubsection{Integrali indefiniti immediati}
Di seguito verr� riportata una tabella con i principali integrali indefiniti immediati e le 
principali funzioni composte
\tablefirsthead{\midrule}
\tablehead{\midrule}
\tablelasttail{\bottomrule}
\renewcommand*{\arraystretch}{3.3}
\begin{center}
  \begin{xtabular}{M{4cm}|M{4cm}}
    $\int {[f(x)]}^\alpha\,\dif x = \frac{{[f(x)]}^{\alpha+1}}{\alpha+1}+c$ & 
      $\int\frac{f'(x)}{\sin^2 f(x)}\,\dif x = -\cot f(x) +c$\\ 
    $\int \frac{f'(x)}{f(x)}\,\dif x = \ln\abs{f(x)}+c$ &
      $\int f'(x)a^{f(x)}\,\dif x = a^{f(x)}\log_a e + c$\\ 
    $\int f'(x)\sin f(x)\,\dif x = -\cos f(x)+c$ &
      $\int f'(x)\cos f(x)\,\dif x = \sin f(x) + c$\\ 
    $\int \frac{f'(x)}{\sqrt{1-{[f(x)]}^2}}\,\dif x = \arcsin f(x) + c$ &
      $\int \frac{f'(x)}{\cos^2 f(x)}\,\dif x = \tan f(x) + c$\\ 
    $\int \frac{f'(x)}{1+{[f(x)]}^2}\,\dif x = \arctan f(x) + c$ & \\ 
  \end{xtabular}
\end{center}
\renewcommand*{\arraystretch}{2.4}
